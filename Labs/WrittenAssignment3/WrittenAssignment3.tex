\documentclass[titlepage]{article}

\usepackage[margin=1in]{geometry}
\usepackage{csquotes}
\usepackage{fancyhdr}
\usepackage{marginnote}
\usepackage{enumitem}
\usepackage{xr}
\usepackage[style=chem-acs]{biblatex}
\usepackage{float,subcaption}
\usepackage{graphicx}
\usepackage{tikz}
\usepackage{siunitx}
\usepackage{amsmath,amssymb}
\usepackage{bm}
\usepackage{mhchem,chemfig}
\usepackage[hidelinks]{hyperref}

\MakeOuterQuote{"}

\fancypagestyle{main}{
    \fancyhf{}
    \fancyhead[L]{\leftmark}
    \fancyhead[R]{CHEM 22100}
    \fancyfoot[R]{Labalme\ \thepage}
}
\fancypagestyle{plain}{
    \fancyhead{}
    \renewcommand{\headrulewidth}{0pt}
}

\reversemarginpar

\setitemize[3]{label={\scriptsize$\blacksquare$}}

\DefineBibliographyStrings{english}{bibliography={References}}

\usetikzlibrary{decorations,arrows.meta,bending,calc}
\colorlet{rex}{magenta}
\colorlet{gax}{gray!50}
\colorlet{orx}{orange!90!yellow!95!black!90}

\sisetup{range-phrase=-,range-units=single}
\DeclareSIUnit{\calorie}{cal}
\DeclareSIUnit{\atomicmassunit}{amu}

\setchemfig{atom sep=2em,fixed length=true,bond offset=3pt,cram width=3pt}
\setcharge{extra sep=3pt}
\pgfdeclaredecoration{ddbond}{initial}{
    \state{initial}[width=3.5pt]{
        \pgfpathlineto{\pgfpoint{4pt}{0pt}}
        \pgfpathmoveto{\pgfpoint{0pt}{2pt}}
        \pgfpathlineto{\pgfpoint{1pt}{2pt}}
        \pgfpathmoveto{\pgfpoint{3pt}{2pt}}
        \pgfpathlineto{\pgfpoint{4pt}{2pt}}
        \pgfpathmoveto{\pgfpoint{4pt}{0pt}}
    }
    \state{final}{
        \pgfpathlineto{\pgfpointdecoratedpathlast}
    }
}
\tikzset{
    lddbond/.style={decorate,decoration=ddbond},
    rddbond/.style={decorate,decoration={ddbond,mirror}}
}

\newcommand{\e}[1][]{\text{e}^{#1}}

\usepackage{subfiles}

\begin{document}




\pagestyle{main}
\renewcommand{\leftmark}{Written Assignment 3}
\setitemize{label={--}}
\section*{Abstract}
\begin{enumerate}
    \item \textbf{Purpose}: N/a.
    \begin{itemize}
        \item It is explained in the introduction that calixarenes are a versatile and useful chemical, but a general method for functionalizing all of the methylene groups at the same time has yet to be discovered, that is, until this paper. However, this purpose is not alluded to at all in any of the sentences in the abstract.
    \end{itemize}
    \item \textbf{Method}: "The radical bromination of \emph{p}-\emph{tert}-butylcalixarene tetramethyl ether was reinvestigated and the product (\textbf{2b}) characterized by X-ray crystallography."
    \begin{itemize}
        \item The authors approached the problem described above by reinvestigating the properties of compound \textbf{2b}, which has bromines at each methylene group as a result of radical bromination of the unbrominated calixarene. They confirmed its structure via X-ray crystallography before seeing if it was reactive.
    \end{itemize}
    \item \textbf{Findings}: "The tetrabromo calixarene derivatives \textbf{2a} or \textbf{2b} react with alcohols (TFE, \ce{EtOH}), azide, and 2-methylfuran under solvolytic conditions affording calixarene derivatives functionalized at the four bridges."
    \begin{itemize}
        \item They found that the tetrabromo calixarene was a good starting material in a variety of reactions leading to calixarene derivatives functionalized at all four methylenes.
    \end{itemize}
    \item \textbf{Conclusions}: "The reaction with alcohols and the aromatic compound proceeds in stereoselective fashion and affords the \emph{rccc} isomer of the tetrasubstituted product."
    \begin{itemize}
        \item The above findings mean that there is now a general and predictable synthetic route for forming calixarenes that are tetrasubstituted with alcohols, azide, and 2-methylfuran.
    \end{itemize}
\end{enumerate}



\section*{Conclusion}
\begin{enumerate}
    \item \textbf{Findings}: "With the use of S\textsubscript{N}1 reaction conditions, \ce{C-O}, \ce{C-N}, and \ce{C-C} bonds can be readily formed under mild conditions."
    \begin{itemize}
        \item This is a direct result that the authors observed.
    \end{itemize}
    \item \textbf{Context}: "The tetrabromo derivatives can be used as starting materials for the preparation of calixarenes monosubstituted at the four bridges."
    \begin{itemize}
        \item As mentioned in connection with the purpose above, this sentence makes it clear that the gap in the literature was how to prepare calixarenes that are monosubstituted at the four bridges, and the authors' tetrabromo derivatives turned out to be the needed intermediate.
    \end{itemize}
\end{enumerate}




\end{document}