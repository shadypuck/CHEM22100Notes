\documentclass{article}

\usepackage[margin=1in]{geometry}
\usepackage{csquotes}
\usepackage{fancyhdr}
\usepackage{marginnote}
\usepackage{enumitem}
\usepackage{xr}
\usepackage{scrextend}
\usepackage[bottom]{footmisc}
\usepackage[style=chem-acs]{biblatex}
\usepackage{float,subcaption}
\usepackage{graphicx}
\usepackage{tikz}
\usepackage{siunitx}
\usepackage{amsmath,amssymb}
\usepackage{bm}
\usepackage{mhchem,chemfig,chemmacros}
\usepackage[hidelinks]{hyperref}

\MakeOuterQuote{"}

\fancypagestyle{main}{
    \fancyhf{}
    \fancyhead[L]{\leftmark}
    \fancyhead[R]{CHEM 22100}
    \fancyfoot[R]{Labalme\ \thepage}
}
\fancypagestyle{plain}{
    \fancyhead{}
    \renewcommand{\headrulewidth}{0pt}
}

\reversemarginpar

\setitemize[3]{label={\scriptsize$\blacksquare$}}

\deffootnotemark{\textsuperscript{\textup{[}\thefootnotemark\textup{]}}}
\deffootnote[2.1em]{0em}{0em}{\textsuperscript{\thefootnote}}

\DefineBibliographyStrings{english}{bibliography={References}}

\usetikzlibrary{decorations,arrows.meta,bending,calc,decorations.pathreplacing}
\colorlet{rex}{magenta}
\colorlet{gax}{gray!50}
\colorlet{orx}{orange!90!yellow!95!black!90}
\colorlet{blx}{cyan}
\colorlet{grx}{green!60!cyan}

\sisetup{range-phrase=-,range-units=single}
\DeclareSIUnit{\calorie}{cal}
\DeclareSIUnit{\atomicmassunit}{amu}

\setchemfig{atom sep=2em,fixed length=true,bond offset=3pt,cram width=3pt}
\setcharge{extra sep=3pt}
\pgfdeclaredecoration{ddbond}{initial}{
    \state{initial}[width=3.5pt]{
        \pgfpathlineto{\pgfpoint{4pt}{0pt}}
        \pgfpathmoveto{\pgfpoint{0pt}{2pt}}
        \pgfpathlineto{\pgfpoint{1pt}{2pt}}
        \pgfpathmoveto{\pgfpoint{3pt}{2pt}}
        \pgfpathlineto{\pgfpoint{4pt}{2pt}}
        \pgfpathmoveto{\pgfpoint{4pt}{0pt}}
    }
    \state{final}{
        \pgfpathlineto{\pgfpointdecoratedpathlast}
    }
}
\tikzset{
    lddbond/.style={decorate,decoration=ddbond},
    rddbond/.style={decorate,decoration={ddbond,mirror}}
}

\newcommand{\N}{\mathbb{N}}
\newcommand{\e}[1][]{\text{e}^{#1}}

\usepackage{subfiles}

\begin{document}




\pagestyle{main}
\renewcommand{\leftmark}{Written Assignment 4}
\setitemize{label={--}}
\section*{Abstract Categorization}
\begin{enumerate}
    \item \textbf{Purpose}: N/a.
    \item \textbf{Method}: "Selective monobromination of aromatic heterocyclic compounds through in-situ acid activation of $2:1$ bromide/bromate couple as a benign brominating reagent is described."
    \item \textbf{Findings}: "The in-situ acid activation of a bromide-bromate couple generates the reactive species \ce{BrOH}, and is proved to be an efficient for the bromination of various heterocycles under mild aqueous conditions without the use of any catalyst. Heterocycles that had electron-rich substituents provided good yields."
    \item \textbf{Conclusions}: N/a.
\end{enumerate}

\begin{center}
    \renewcommand{\arraystretch}{1.2}
    \begin{longtable}{|>{\raggedright}p{0.22\linewidth}|>{\raggedright}p{0.22\linewidth}|>{\raggedright}p{0.22\linewidth}|>{\raggedright\arraybackslash}p{0.22\linewidth}|}
        \hline
        Components of an abstract & \textbf{Present in abstract?} & \textbf{If so, how could the writing be improved?} & \textbf{If not, summarize the portions from the full text.}\\
        \hline
        Purpose of the study is outlined in a clear and concise way&
            No.&
            N/a.&
            The authors need to make it more clear, as they do in the introduction, that bromination of heteroarenes is an important and widely used process in pharmaceuticals and academia, but there is a lack of a safe and environmentally friendly way to run the reaction, and \emph{this} is why it's significant that they've found a "benign brominating reagent."\\
        \hline
        Methods of the study are covered&
            Yes.&
            A bit more detail could be useful (i.e., noting that they mainly focused on, and achieved their most selective results with, aminopyridines).&
            N/a.\\
        \hline
        Findings of the study are summarized&
            Yes.&
            There's a typo --- "an efficient" what? Other than that, the experimental results are fairly well summarized, and in good detail (though, again, they failed to note the significance of aminopyridines to their paper).&
            N/a.\\
        \hline
        Conclusions are written clearly&
            No.&
            N/a.&
            The take-away message is basically just that they filled the gap in the purpose: There wasn't an environmentally friendly way to brominate heteroarenes under mild conditions, and now there is one (for some heteroarenes). More specifically, there is a good way for aminopyridines and a passable way for some other types of heteroarenes.\\
        \hline
    \end{longtable}
    \vspace{-1em}
    \begin{table}[h!]
        \caption{Review of abstract.}
    \end{table}
    \vspace{-3em}
\end{center}



\section*{General Comments}
\begin{itemize}
    \item Strengths: The abstract is largely clear and succinct. It successfully conveys how the bromination was achieved both on a macroscale ("in-situ acid activation of [\emph{sic}] a $2:1$ bromide/bromate couple") and a microscale (a highly reactive \ce{BrOH} intermediate is formed that can then brominate an arene via EAS).
    \item Weaknesses: Despite the generally good use of language, there are some instances of grammatically incorrect and unclear phrasing, e.g., "it is proved to be an efficient." Additionally, the abstract does not address in sufficient detail why this research was done and what the take-aways are --- it needs to be more explicit that bromination is widely used in pharmaceutical applications, and this paper reports a method of doing it in a sustainable way under mild conditions for certain types (which should be specified) of heteroarenes.
\end{itemize}




\end{document}