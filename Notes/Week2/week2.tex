\documentclass[../notes.tex]{subfiles}

\pagestyle{main}
\renewcommand{\chaptermark}[1]{\markboth{\chaptername\ \thechapter\ (#1)}{}}
\stepcounter{chapter}

\begin{document}




\chapter{Spectrometry}
\section{Office Hours (Snyder)}
\begin{itemize}
    \item \marginnote{1/17:}Does cyclohexane only have one \ce{{}^13C} NMR signal, and only one \ce{{}^1H} NMR signal?
    \begin{itemize}
        \item 1 singlet for \ce{{}^13C}.
        \item 1 singlet for \ce{{}^1H}.
        \item We don't integrate carbon.
        \item We only integrate to compare things.
        \item We won't have to deal with cyclohexane conformations wrt. NMR on any test.
    \end{itemize}
    \item What do we need to know about the Karplus correlation?
    \begin{itemize}
        \item We won't need it for problems.
        \item It's useful, but we've got other things to worry about.
    \end{itemize}
    \item Do chemists/when do chemists run \ce{{}^13C} NMR experiments with all carbons isotopically carbon-13?
    \item Is the reason we don't integrate carbon because the placing of the carbon-13s is random? Would the proportions not still be representative?
    \item For \ce{{}^1H} NMR, feel free to draw in the hydrogen atoms on the line-angle structure.
    \item Multiplying $n+1$ of different types of neighbors (e.g., if a hydrogen has 3 neighboring hydrogens to one side and 2 neighboring hydrogens to the other side, it has a maximum of $(3+1)(2+1)=12$ peaks in its signal).
    \begin{itemize}
        \item The multiplication analysis applies only to chains that are completely different.
    \end{itemize}
\end{itemize}




\end{document}