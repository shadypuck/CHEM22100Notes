\documentclass[../notes.tex]{subfiles}

\pagestyle{main}
\renewcommand{\chaptermark}[1]{\markboth{\chaptername\ \thechapter\ (#1)}{}}
\setcounter{chapter}{2}

\begin{document}




\chapter{More Types of Reactions}
\section{Radical Chemistry}
\begin{itemize}
    \item \marginnote{1/25:}Reviews mass spectroscopy.
    \item Radical chemistry allows us to do some reactions that we cannot do in a two-electron manifold.
    \begin{itemize}
        \item If we want to attach a nucleophile to the C2 position of propane, heat alone will not make the hydrogen on that position leave (hydrides are terrible leaving groups).
    \end{itemize}
    \item Presents how easy (in terms of $\Delta H$) it is to homolytically cleave various \ce{C-H} bonds in alkanes.
    \item Radical stability is the same as carbocation stability.
    \begin{itemize}
        \item In terms of decreasing stability,
        \begin{equation*}
            \text{benzylic} \approx \text{allylic}
            > \text{tertiary}
            > \text{secondary}
            > \text{primary}
            > \text{methyl}
        \end{equation*}
        \item Note that a benzylic or allylic \emph{primary} radical is still more stable than a tertiary radical with no resonance stabilization.
    \end{itemize}
    \item Three steps (initiation, propagation, and termination).
    \begin{itemize}
        \item Initiation is either started by light ($h\nu$) or heat ($\Delta$).
    \end{itemize}
    \item You can lose \ce{CO2} in a radical mechanism.
    \begin{figure}[h!]
        \centering
        \footnotesize
        \schemestart
            \chemfig{*6(-=-(-(=[2]O)-[:-30]@{O1a}O-[@{sb1}:30]@{O1b}O-[:-30](=[6]O)-[:30]*6(=-=-=-))=-=)}
            \arrow{->[$h\nu$][$\Delta$]}
            \large\chemfig{2}\footnotesize\arrow{0}[,0.2]
            \chemfig{*6(-=-@{C2}(-[@{sb2a}](=[2]O)-[@{sb2b}:-30]@{O2}\charge{90=\.,0=\:,-90=\:}{O})=-=)}
            \arrow
            \large\chemfig{2}\footnotesize\arrow{0}[,0.2]
            \chemfig{*6(-=-\charge{45=\.}{}=-=)}
            \arrow{0}[,0]\+{1.5em}
            \large\chemfig{2}\footnotesize\arrow{0}[,0.2]
            \chemfig{O=C=O}
        \schemestop
        \chemmove{
            \draw [rex,semithick,shorten <=2pt,shorten >=2pt,arrows={-Stealth[harpoon,swap]}] (sb1) to[bend right=90,looseness=3] (O1a);
            \draw [rex,semithick,shorten <=2pt,shorten >=2pt,arrows={-Stealth[harpoon,swap]}] (sb1) to[bend right=90,looseness=3] (O1b);
            \draw [rex,semithick,shorten <=2pt,shorten >=2pt,arrows={-Stealth[harpoon,swap]}] (sb2a) to[bend right=90,looseness=3] (C2);
            \draw [rex,semithick,shorten <=2pt,shorten >=2pt,arrows={-Stealth[harpoon,swap]}] (sb2a) to[bend right=60,looseness=2] (sb2b);
            \draw [rex,semithick,shorten <=6pt,shorten >=2pt,arrows={-Stealth[harpoon,swap]}] (O2) to[out=90,in=70,looseness=2.5] (sb2b);
        }
        \caption{Losing \ce{CO2} in a radical mechanism.}
        \label{fig:CO2radical}
    \end{figure}
    \begin{itemize}
        \item The second step is strongly favored by entropy ($\Delta S$).
    \end{itemize}
    \item Chlorination of alkanes.
    \begin{figure}[h!]
        \centering
        \footnotesize
        \begin{subfigure}[b]{\linewidth}
            \centering
            \schemestart
                \chemfig{@{Cl1a}Cl-[@{sb1}]@{Cl1b}Cl}
                \arrow{->[$h\nu$]}
                \chemfig{\charge{0=\.}{Cl}}
                \+{1em,1em}
                \chemfig{\charge{180=\.}{Cl}}
            \schemestop
            \chemmove{
                \draw [rex,semithick,shorten <=2pt,shorten >=2pt,arrows={-Stealth[harpoon,swap]}] (sb1) to[bend right=60,looseness=2] (Cl1a);
                \draw [rex,semithick,shorten <=2pt,shorten >=2pt,arrows={-Stealth[harpoon]}] (sb1) to[bend left=60,looseness=2] (Cl1b);
            }
            \caption{Initiation.}
            \label{fig:alkaneChlorinationa}
        \end{subfigure}\\[2em]
        \begin{subfigure}[b]{\linewidth}
            \centering
            \schemestart
                \chemfig{-[:30](-[:70])(-[:110])-[:-30]@{C1}-[@{sb1}:30]H}
                \arrow{0}[,0.6]
                \chemfig{@{Cl2}\charge{180=\.}{Cl}}
                \arrow
                \chemfig{-[:30](-[:70])(-[:110])-[:-30]\charge{0=\.}{}}
                \arrow{0}[,0]\+{1.5em,0.8em}
                \chemfig{HCl}
            \schemestop
            \chemmove{
                \draw [rex,semithick,shorten <=2pt,shorten >=2pt,arrows={-Stealth[harpoon]}] (sb1) to[out=-80,in=-90,looseness=5] (C1);
                \draw [rex,semithick,shorten <=2pt,shorten >=2pt,arrows={-Stealth[harpoon,swap]}] (sb1) to[out=100,in=90,looseness=2] ++(0.79,0.15);
                \draw [rex,semithick,shorten <=2pt,shorten >=2pt,arrows={-Stealth[harpoon]}] ([xshift=-1mm]Cl2.west) to[bend right=80,looseness=2.5] ++(-0.49,-0.15);
            }
            \rule{0.9em}{0pt}\\[2em]
            \schemestart
                \chemfig{-[:30](-[:70])(-[:110])-[:-30]@{C1}\charge{0=\.}{}}
                \arrow{0}[,0.6]
                \chemfig{Cl-[@{sb2}]@{Cl2}Cl}
                \arrow
                \chemfig{-[:30](-[:70])(-[:110])-[:-30]-[:30]Cl}
                \arrow{0}[,0]\+{1em,1.2em}
                \chemfig{\charge{180=\.}{Cl}}
            \schemestop
            \chemmove{
                \draw [rex,semithick,shorten <=2pt,shorten >=2pt,arrows={-Stealth[harpoon,swap]}] ([xshift=1mm]C1.east) to[bend left=80,looseness=3] ++(0.49,0.1);
                \draw [rex,semithick,shorten <=2pt,shorten >=2pt,arrows={-Stealth[harpoon]}] (sb2) to[bend right=80,looseness=2] ++(-0.9,-0.32);
                \draw [rex,semithick,shorten <=2pt,shorten >=2pt,arrows={-Stealth[harpoon]}] (sb2) to[out=80,in=110,looseness=2.8] (Cl2);
            }
            \caption{Propagation.}
            \label{fig:alkaneChlorinationb}
        \end{subfigure}\\[2em]
        \begin{subfigure}[b]{\linewidth}
            \centering
            \schemestart
                \chemfig{-[:30](-[:70])(-[:110])-[:-30]@{C1}\charge{0=\.}{}}
                \arrow{0}[,0.7]
                \chemfig{@{Cl2}\charge{180=\.}{Cl}}
                \arrow
                \chemfig{-[:30](-[:70])(-[:110])-[:-30]-[:30]Cl}
            \schemestop
            \chemmove{
                \draw [rex,semithick,shorten <=2pt,shorten >=2pt,arrows={-Stealth[harpoon,swap]}] ([xshift=1mm]C1.east) to[bend left=90,looseness=3] ++(0.42,0.2);
                \draw [rex,semithick,shorten <=2pt,shorten >=2pt,arrows={-Stealth[harpoon]}] ([xshift=-1mm]Cl2.west) to[bend right=90,looseness=3] ++(-0.4,-0.21);
            }
            \caption{Termination.}
            \label{fig:alkaneChlorinationc}
        \end{subfigure}
        \caption{Chlorination of alkanes.}
        \label{fig:alkaneChlorination}
    \end{figure}
    \begin{itemize}
        \item If multiple types of \ce{C-H} bonds are present, they will all be functionalized but in differing amounts.
        \begin{itemize}
            \item The mechanism is sensitive both to the number of available hydrogens of each type, how sterically accessible hydrogens are, and (most importantly) radical stability.
        \end{itemize}
        \item You can also get polychlorinated products.
        \item Take-home message: If we use this, we only do so when all hydrogens are symmetric and we use excess starting material.
    \end{itemize}
    \item Bromination of alkanes is basically the same.
    \begin{itemize}
        \item One difference is that bromination is incredibly sensitive to radical stability, so whatever is the most stable radical will be the brominated one.
    \end{itemize}
    \item Multistep synthesis example.
    \begin{itemize}
        \item Propane to propane-1,2-diol.
        \item Use radical bromination to put a bromine on C2, then $\beta$-elimination, then dihydroxylation.
    \end{itemize}
    \item Allylic/benzylic halogenation.
    \item General form.
    \begin{equation*}
        \ce{=- ->[Br2][h\nu] =--Br}
    \end{equation*}
    % \begin{figure}[h!]
    %     \centering
    %     \footnotesize
    %     \begin{subfigure}[b]{\linewidth}
    %         \centering
    %         \schemestart
    %             \chemfig{@{Br1a}Br-[@{sb1}]@{Br1b}Br}
    %             \arrow{->[$h\nu$]}
    %             \chemfig{\charge{0=\.}{Br}}
    %             \+{1em,1em}
    %             \chemfig{\charge{180=\.}{Br}}
    %         \schemestop
    %         \chemmove{
    %             \draw [rex,semithick,shorten <=2pt,shorten >=2pt,arrows={-Stealth[harpoon,swap]}] (sb1) to[bend right=60,looseness=2] (Br1a);
    %             \draw [rex,semithick,shorten <=2pt,shorten >=2pt,arrows={-Stealth[harpoon]}] (sb1) to[bend left=60,looseness=2] (Br1b);
    %         }
    %         \caption{Initiation.}
    %         \label{fig:allylicHalogenationa}
    %     \end{subfigure}\\[2em]
    %     \begin{subfigure}[b]{\linewidth}
    %         \centering
    %         \schemestart
    %             \chemfig{=_[:30]-[:-30]@{C1}-[@{sb1}:30]H}
    %             \arrow{0}[,0.6]
    %             \chemfig{@{Br2}\charge{180=\.}{Br}}
    %             \arrow
    %             \chemfig{=_[:30]-[:-30]\charge{0=\.}{}}
    %             \arrow{0}[,0]\+{1.5em,0.8em}
    %             \chemfig{HBr}
    %         \schemestop
    %         \chemmove{
    %             \draw [rex,semithick,shorten <=2pt,shorten >=2pt,arrows={-Stealth[harpoon]}] (sb1) to[out=-80,in=-90,looseness=5] (C1);
    %             \draw [rex,semithick,shorten <=2pt,shorten >=2pt,arrows={-Stealth[harpoon,swap]}] (sb1) to[out=100,in=90,looseness=2] ++(0.8,0);
    %             \draw [rex,semithick,shorten <=2pt,shorten >=2pt,arrows={-Stealth[harpoon]}] ([xshift=-1mm]Br2.west) to[bend right=80,looseness=2.5] ++(-0.5,-2pt);
    %         }
    %         \rule{0.9em}{0pt}\\[2em]
    %         \schemestart
    %             \chemfig{=_[:30]-[:-30]@{C1}\charge{0=\.}{}}
    %             \arrow{0}[,0.6]
    %             \chemfig{Br-[@{sb2}]@{Br2}Br}
    %             \arrow
    %             \chemfig{=_[:30]-[:-30]-[:30]Br}
    %             \arrow{0}[,0]\+{1em,1.2em}
    %             \chemfig{\charge{180=\.}{Br}}
    %         \schemestop
    %         \chemmove{
    %             \draw [rex,semithick,shorten <=2pt,shorten >=2pt,arrows={-Stealth[harpoon,swap]}] ([xshift=1mm]C1.east) to[bend left=80,looseness=3] ++(0.5,0);
    %             \draw [rex,semithick,shorten <=2pt,shorten >=2pt,arrows={-Stealth[harpoon]}] (sb2) to[bend right=80,looseness=2] ++(-0.91,-0.12);
    %             \draw [rex,semithick,shorten <=2pt,shorten >=2pt,arrows={-Stealth[harpoon]}] (sb2) to[out=80,in=110,looseness=2.8] (Br2);
    %         }
    %         \caption{Propagation.}
    %         \label{fig:allylicHalogenationb}
    %     \end{subfigure}\\[2em]
    %     \begin{subfigure}[b]{\linewidth}
    %         \centering
    %         \schemestart
    %             \chemfig{=_[:30]-[:-30]@{C1}\charge{0=\.}{}}
    %             \arrow{0}[,0.7]
    %             \chemfig{@{Br2}\charge{180=\.}{Br}}
    %             \arrow
    %             \chemfig{=_[:30]-[:-30]-[:30]Br}
    %         \schemestop
    %         \chemmove{
    %             \draw [rex,semithick,shorten <=2pt,shorten >=2pt,arrows={-Stealth[harpoon,swap]}] ([xshift=1mm]C1.east) to[bend left=90,looseness=3] ++(0.41,0.06);
    %             \draw [rex,semithick,shorten <=2pt,shorten >=2pt,arrows={-Stealth[harpoon]}] ([xshift=-1mm]Br2.west) to[bend right=90,looseness=3] ++(-0.41,-0.05);
    %         }
    %         \caption{Termination.}
    %         \label{fig:allylicHalogenationc}
    %     \end{subfigure}
    %     \caption{Allylic/benzylic halogenation.}
    %     \label{fig:allylicHalogenation}
    % \end{figure}
    \begin{itemize}
        \item A possible side reaction is bromination of the alkene, but this requires a high temperature and low concentration.
        \item The mechanism is entirely analogous to that of chlorination.
    \end{itemize}
    \item \ce{HBr} addition to alkenes.
    \begin{itemize}
        \item The hydrohalogenation mechanism produces the Markovnikov product.
        \item Morris Kharasch at UChicago in 1933 proposed that a radical mechanism produced the anti-Markovnikov product.
        \begin{itemize}
            \item In particular, when run in the presence of air, it proceeds quickly even at low temperatures and with the help of an organic peroxide.
        \end{itemize}
    \end{itemize}
    \item Mechanism.
    \begin{figure}[h!]
        \centering
        \footnotesize
        \begin{subfigure}[b]{\linewidth}
            \centering
            \schemestart
                \chemfig{R@{O1a}O-[@{sb1}]@{O1b}OR}
                \arrow{->[$h\nu$]}
                \chemfig{R\charge{0=\.}{O}}
                \+{1em,1em}
                \chemfig{\charge{180=\.}{O}R}
            \schemestop
            \chemmove{
                \draw [rex,semithick,shorten <=2pt,shorten >=2pt,arrows={-Stealth[harpoon,swap]}] (sb1) to[bend right=60,looseness=2] (O1a);
                \draw [rex,semithick,shorten <=2pt,shorten >=2pt,arrows={-Stealth[harpoon]}] (sb1) to[bend left=60,looseness=2] (O1b);
            }
            \caption{Initiation.}
            \label{fig:radicalHydrohalogenationa}
        \end{subfigure}\\[2em]
        \begin{subfigure}[b]{\linewidth}
            \centering
            \schemestart
                \chemfig{R@{O1}\charge{0=\.}{O}}
                \arrow{0}[,0.6]
                \chemfig{H-[@{sb2}]@{Br2}Br}
                \arrow
                \chemfig{ROH}
                \arrow{0}[,0]\+{1.5em,0.8em}
                \chemfig{\charge{180=\.}{Br}}
            \schemestop
            \chemmove{
                \draw [rex,semithick,shorten <=2pt,shorten >=2pt,arrows={-Stealth[harpoon,swap]}] ([xshift=1mm]O1.east) to[bend left=80,looseness=2.5] ++(0.45,-0.1);
                \draw [rex,semithick,shorten <=2pt,shorten >=2pt,arrows={-Stealth[harpoon]}] (sb2) to[bend right=80,looseness=1.7] ++(-0.82,-0.1);
                \draw [rex,semithick,shorten <=2pt,shorten >=2pt,arrows={-Stealth[harpoon]}] (sb2) to[out=80,in=110,looseness=2.5] (Br2);
            }\\[2em]
            \schemestart
                \chemfig{@{Br1}\charge{0=\.}{Br}}
                \arrow{0}[,0.6]
                \chemfig{=_[@{db2}:30]@{C2}-[:-30]}
                \arrow
                \chemfig{Br-[:-30]-[:30]\charge{90=\.}{}-[:-30]}
            \schemestop
            \chemmove{
                \draw [rex,semithick,shorten <=2pt,shorten >=2pt,arrows={-Stealth[harpoon,swap]}] ([xshift=1mm]Br1.east) to[bend left=90,looseness=3] ++(0.4,0);
                \draw [rex,semithick,shorten <=2pt,shorten >=2pt,arrows={-Stealth[harpoon]}] (db2) to[out=120,in=80,looseness=2.2] ++(-0.62,-0.03);
                \draw [rex,semithick,shorten <=2pt,shorten >=2pt,arrows={-Stealth[harpoon]}] (db2) to[out=100,in=100,looseness=4] (C2);
            }
            \caption{Propagation.}
            \label{fig:radicalHydrohalogenationb}
        \end{subfigure}\\[2em]
        \begin{subfigure}[b]{\linewidth}
            \centering
            \schemestart
                \chemfig{Br-[:-30]-[:30]@{C1}\charge{90=\.}{}-[:-30]}
                \arrow{0}[,0.6]
                \chemfig{H-[@{sb2}]@{Br2}Br}
                \arrow
                \chemfig{Br-[:-30]-[:30](-[2]H)-[:-30]}
                \arrow{0}[,0]\+{1em,1em}
                \chemfig{\charge{0=\.}{Br}}
            \schemestop
            \chemmove{
                \draw [rex,semithick,shorten <=6pt,shorten >=2pt,arrows={-Stealth[harpoon,swap]}] (C1) to[bend left=80,looseness=2] ++(0.93,0);
                \draw [rex,semithick,shorten <=2pt,shorten >=2pt,arrows={-Stealth[harpoon]}] (sb2) to[bend right=80,looseness=2] ++(-1.0,0.05);
                \draw [rex,semithick,shorten <=2pt,shorten >=2pt,arrows={-Stealth[harpoon]}] (sb2) to[out=70,in=110,looseness=2] (Br2);
            }
            \caption{Termination/propagation.}
            \label{fig:radicalHydrohalogenationc}
        \end{subfigure}
        \caption{Radical hydrohalogenation.}
        \label{fig:radicalHydrohalogenation}
    \end{figure}
    \begin{itemize}
        \item In hydrohalogenation, the hydrogen adds into the double bond to form the most stable carbocation.
        \item In this mechanism, the bromine adds into the double bond to form the most stable radical.
    \end{itemize}
\end{itemize}




\end{document}