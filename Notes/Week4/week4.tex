\documentclass[../notes.tex]{subfiles}

\pagestyle{main}
\renewcommand{\chaptermark}[1]{\markboth{\chaptername\ \thechapter\ (#1)}{}}
\setcounter{chapter}{3}

\begin{document}




\chapter{Exam 1 Materials}
\section{Office Hours (Snyder)}
\begin{itemize}
    \item \marginnote{1/31:}Sterics vs. an EDG on the diene?
    \begin{center}
        \footnotesize
        \chemfig{[:-120]*6((-[:-30]OMe)=-=-)}
    \end{center}
    \begin{itemize}
        \item Even if you have a strong EDG, if sterics prevents your diene from achieving the s-cis conformation, the reaction will be very slow and/or not proceed.
    \end{itemize}
    \item PSet 2, 1f: Why is $t$-BuOH listed?
    \begin{itemize}
        \item We need the Zaitsev product here; we ignore the bulky base --- it's just used to favor E2 over S\textsubscript{N}2.
    \end{itemize}
    \item PSet 2, 5b: ?
    \begin{itemize}
        \item Think Diels-Alder here with the given SM as the dienophile and then ozonolysis.
    \end{itemize}
\end{itemize}



\section{Office Hours (Salinas)}
\begin{itemize}
    \item E1 2020, 3b: Distinction between carbons 3 and 6?
    \begin{itemize}
        \item 3 is shifted higher because it's next to two functional groups, whereas 6 is only next to one functional group.
    \end{itemize}
    \item E1 2020, 4: Ordering of the last 2/3 steps? Shouldn't we take advantage of the allylic stability to make the process even more selective before hydrogenating?
    \begin{itemize}
        \item Both are right.
    \end{itemize}
    \item PSet 2, 1e: Is there reactivity with the alkene that's not next to the EWG in the dienophile?
    \begin{itemize}
        \item Not enough reactivity to care about.
    \end{itemize}
    \item PSet 2, 4(i): Is the diene too unreactive?
    \begin{itemize}
        \item Yes
    \end{itemize}
    \item PSet 2, 6: Are we using benzoyl peroxide to pull the bromine off the starting material and leave a radical behind at that site? A radical which we can either quench with \ce{H*} or wind back around to form a ring and then quench?
    \begin{itemize}
        \item Yes.
    \end{itemize}
\end{itemize}



\section{Review}
\begin{itemize}
    \item \marginnote{2/1:}Esters bonded to the diene/dienophile through their single oxygen will be donor groups, but worse than groups like \ce{OMe}.
    \begin{itemize}
        \item This is because they \emph{can} push electrons toward the diene/dienophile, but they also have the option to withdraw electrons through resonance.
        \item This effect is enough to strongly deactivate a dienophile.
    \end{itemize}
    \item Substituents that aren't on the double bond count as alkyl groups. Their inductive effect will vary based on other groups further down the chain, but they will have no resonance effects.
    \item When you have two groups on the diene in the "B" position, the diene will \emph{never} be viable.
    \item He is going to ask us to use the reactions from last quarter, but not know the mechanisms.
    \begin{itemize}
        \item Write "magic powder" over your arrow if you forget the reagents.
        \item Alkene reactions to know: hydrogenation (\ce{H2} \ce{Pd}/\ce{C}), dihydroxylation (\ce{OsO4}), ozonolysis (\ce{O3} \ce{Me2S}), hydrobromination (\ce{HBr}), and bromination (\ce{Br2}).
    \end{itemize}
    \item We use both \ce{KO^{$t$}Bu} and \ce{{}^{$t$}BuOH} to establish a buffer, an equilibrium that will allow us to both grab and release a proton.
    \begin{itemize}
        \item This is not so important for E2 chemistry, but is important for other chemistry.
    \end{itemize}
    \item MS: para-dimethylbenzene vs. ethylbenzene.
    \begin{itemize}
        \item For para-dimethylbenzene, we can only lose one methyl group (losing the other would lead to a $2+$ ion, which we will not observe). This gives a $m/z=91$ peak.
        \item For ethylbenzene, we can lose just a methyl radical \emph{or} the entire ethyl chain. This gives a $m/z=91$ \emph{and} a $m/z=77$ peak.
        \item Rule: If you have ortho/meta/para substituents, you can lose \emph{at most one} substituent at a time.
    \end{itemize}
    \item For \ce{{}^13C} NMR, he's not above giving us cyclobutane with a ketone attached.
    \item If you need to form \ce{C-C} bonds, that's probably going to be Diels-Alder for this exam.
    \item \marginnote{2/2:}PSet 1/2 review takeaways.
    \begin{itemize}
        \item $4^\circ$/allylic \ce{{}^13C} peaks are higher than $3^\circ$ peaks.
        \item When asked how you can distinguish two molecules based on NMR spectra, answer in terms of the number of peaks/shapes of peaks, not the shift of peaks (that's not generally intuitively characteristic of a molecule).
        \item Beware Diels-Alder products drawn with stereochemistry opposite to the way we've practiced.
        \item Show the \ce{H-SnBu3} bond homolytically cleaving if necessary.
    \end{itemize}
\end{itemize}
\newpage



\section{Exam 1 Cheat Sheet}
\begin{table}[h!]\marginnote{2/3:}
    \centering
    \small
    \renewcommand{\arraystretch}{1.2}
    \begin{tabular}{|l|l|}
        \hline
        \multicolumn{2}{|c|}{\textbf{COMMON ABSORPTIONS}}\\ \hline
        Aromatic \ce{C-C} & Two peaks usually in the range of $\SIrange{1500}{1600}{\per\centi\meter}$\\ \hline
        \ce{C=C} & $\sim\SI{1650}{\per\centi\meter}$\\ \hline
        \ce{C=O} & $\sim\SI{1710}{\per\centi\meter}$ (shifts to $\sim\SI{1735}{\per\centi\meter}$ for esters)\\ \hline
        \ce{C#C} & $\SIrange{2100}{2300}{\per\centi\meter}$\\ \hline
        \ce{C#N} & $\SIrange{2100}{2300}{\per\centi\meter}$\\ \hline
        \ce{C-H} (aldehyde) & Two peaks at $\SI{2170}{\per\centi\meter}$ and $\SI{2810}{\per\centi\meter}$\\ \hline
        $sp^3$ \ce{C-H} & Just to the right of $\SI{3000}{\per\centi\meter}$\\ \hline
        $sp^2$ \ce{C-H} & Just to the left of $\SI{3000}{\per\centi\meter}$\\ \hline
        $sp$ \ce{C-H} & $\sim\SI{3300}{\per\centi\meter}$\\ \hline
        \ce{N-H} & $\sim\SI{3300}{\per\centi\meter}$ (one peak for \ce{-NH-}, two peaks for \ce{-NH2})\\ \hline
        \ce{O-H} (alcohol) & $\sim\SI{3400}{\per\centi\meter}$ (a broad, smooth peak)\\ \hline
        \ce{O-H} (acid) & $\sim\SIrange{2500}{3500}{\per\centi\meter}$ (a very broad, ugly [not smooth] peak)\\ \hline
    \end{tabular}
    \caption*{Common IR spectroscopy absorptions.}
\end{table}
\begin{table}[h!]
    \centering
    \small
    \renewcommand{\arraystretch}{1.4}
    \begin{tabular}{|lc|lc|}
        \hline
        \rule{0pt}{2em}\textbf{Type of Proton} & \textbf{\shortstack{Chemical Shift\\($\bm{\delta}$, ppm)}} & \textbf{Type of Proton} & \textbf{\shortstack{Chemical Shift\\($\bm{\delta}$, ppm)}}\\
        $\ang{1}$ Alkyl, {\sf\ce{RC{\color{rex}H}3}} & \numrange{0.8}{1.2} & Alkyl bromide, {\sf\ce{RC{\color{rex}H}2}Br} & \numrange{3.4}{3.6}\\
        $\ang{2}$ Alkyl, {\sf\ce{RC{\color{rex}H}2R}} & \numrange{1.2}{1.5} & Alkyl chloride, {\sf\ce{RC{\color{rex}H}2}Cl} & \numrange{3.6}{3.8}\\
        $\ang{3}$ Alkyl, {\sf\ce{R3C{\color{rex}H}}} & \numrange{1.4}{1.8} & Vinylic, {\sf\ce{R2C=C{\color{rex}H}2}} & \numrange{4.6}{5.0}\\
        Allylic, {\sf\ce{R2C=CR-C{\color{rex}H}3}} & \numrange{1.6}{1.9} & Vinylic, {\sf\ce{R2C=CR{\color{rex}H}}} & \numrange{5.2}{5.7}\\
        Ketone, {\sf\ce{RCOC{\color{rex}H}3}} & \numrange{2.1}{2.6} & Aromatic, {\sf\ce{Ar{\color{rex}H}}} & \numrange{6.0}{8.5}\\
        Benzylic, {\sf\ce{ArC{\color{rex}H}3}} & \numrange{2.2}{2.5} & Aldehyde, {\sf\ce{RCO{\color{rex}H}}} & \numrange{9.5}{10.5}\\
        Acetylenic, {\sf\ce{RC#C{\color{rex}H}}} & \numrange{2.5}{3.1} & Alcohol hydroxyl, {\sf\ce{RO{\color{rex}H}}} & \numrange{0.5}{6.0}\textsuperscript{*}\\
        Alkyl iodide, {\sf\ce{RC{\color{rex}H}2I}} & \numrange{3.1}{3.3} & Amino, {\sf\ce{R-N{\color{rex}H}2}} & \numrange{1.0}{5.0}\textsuperscript{*}\\
        Ether, {\sf\ce{ROC{\color{rex}H}2R}} & \numrange{3.3}{3.9} & Phenolic, {\sf\ce{ArO{\color{rex}H}}} & \numrange{4.5}{7.7}\textsuperscript{*}\\
        Alcohol, {\sf\ce{HOC{\color{rex}H}2R}} & \numrange{3.3}{4.0} & Carboxylic, {\sf\ce{RCOO{\color{rex}H}}} & \numrange{10}{13}\textsuperscript{*}\\
        \hline
        \multicolumn{4}{l}{\footnotesize\textsuperscript{*}The chemical shifts of these protons vary in different solvents and with temperature and concentration.}
    \end{tabular}
    \caption*{Approximate proton chemical shifts.}
\end{table}
\begin{table}[h!]
    \centering
    \small
    \renewcommand{\arraystretch}{1.4}
    \begin{tabular}{|lc|}
        \hline
        \rule{0pt}{2em}\textbf{Type of Carbon} & \textbf{\shortstack{Chemical Shift\\($\bm{\delta}$, ppm)}}\\
        $\ang{1}$ Alkyl, {\sf\ce{R{\color{rex}C}H3}} & \numrange{0}{40}\\
        $\ang{2}$ Alkyl, {\sf\ce{R{\color{rex}C}H2R}} & \numrange{10}{50}\\
        $\ang{3}$ Alkyl, {\sf\ce{R{\color{rex}C}HR2}} & \numrange{15}{50}\\
        Alkyl halide or amine, {\sf\ce{R3{\color{rex}C}X}} ($\ce{X}=\ce{Cl},\ce{Br},\ce{NR$'$2}$) & \numrange{10}{65}\\
        Alcohol or ether, {\sf\ce{R3{\color{rex}C}OR$'$}} & \numrange{50}{90}\\
        Alkyne, {\sf\ce{R{\color{rex}C}#R$'$}} & \numrange{60}{90}\\
        Alkene, {\sf\ce{R2{\color{rex}C}=R$'$}} & \numrange{100}{170}\\
        Aryl, {\renewcommand*\printatom[1]{\ensuremath{\mathsf{#1}}}\chemfig[atom sep=1.4em]{[:30]**6(--{\color{rex}C}(-R)----)}} & \numrange{100}{170}\\
        Nitrile, {\sf\ce{R{\color{rex}C}#N}} & \numrange{120}{130}\\
        Amide, {\sf\ce{R{\color{rex}C}ONR$'$2}} & \numrange{150}{180}\\
        Carboxylic acid or ester, {\sf\ce{R{\color{rex}C}OOR$'$}} & \numrange{160}{185}\\
        Aldehyde or ketone, {\sf\ce{R{\color{rex}C}OR$'$}} & \numrange{182}{215}\\
        \hline
    \end{tabular}
    \caption*{Approximate carbon-13 chemical shifts.}
\end{table}
\begin{figure}[h!]
    \centering
    \footnotesize
    \schemestart
        \chemfig{CH_2=CH-CH_2-R}
        \arrow{->[ionization][-$\e[-]$]}[,1.4]
        \chemfig{CH_2-[@{db2},,,,lddbond]CH-[@{sb2a}]CH_2-[@{sb2b}]@{R2}R}
        \arrow{->[fragmentation]}[,1.8]
        \chemleft{[}
            \subscheme{
                \chemfig{\charge{90={\tiny$+$}}{C}H_2-CH=CH_2}
                \arrow(.south--.north){<->}[-90]
                \chemfig{CH_2=CH-\charge{90={\tiny$+$}}{C}H_2}
            }
        \chemright{]}
        \arrow{0}[,0]\+
        \chemfig{\charge{180=\.}{R}}
    \schemestop
    \chemmove{
        \draw [rex,semithick,shorten <=4pt,shorten >=2pt,arrows={-Stealth[harpoon,swap]}] (db2) to[bend left=60,looseness=1.5] (sb2a);
        \draw [rex,semithick,shorten <=2pt,shorten >=2pt,arrows={-Stealth[harpoon]}] (sb2b) to[bend right=60,looseness=1.5] (sb2a);
        \draw [rex,semithick,shorten <=2pt,shorten >=2pt,arrows={-Stealth[harpoon,flex]}] (sb2b) to[bend left=60,looseness=1.5] (R2);
    }
    \caption*{Resonance fragmentation: Alkenes.}
\end{figure}
\begin{figure}[h!]
    \centering
    \footnotesize
    \schemestart
        \chemfig{R-\charge{90=\:}{Z}-CH_2-CH_3}
        \arrow{->[ionization][-$\e[-]$]}[,1.4]
        \chemfig{R-@{Z2}\charge{73=\.,107={\tiny$+$}}{Z}-[@{sb2a}]CH_2-[@{sb2b}]@{C2}CH_3}
        \arrow{->[fragmentation]}[,1.8]
        \chemleft{[}
            \subscheme{
                \chemfig{R-\charge{90={\tiny$+$}}{Z}=CH_2}
                \arrow(.south--.north){<->}[-90]
                \chemfig{R-\charge{90=\:}{Z}-\charge{90={\tiny$+$}}{C}H_2}
            }
        \chemright{]}
        \arrow{0}[,0]\+
        \chemfig{\charge{180=\.}{C}H_3}
    \schemestop
    \chemmove{
        \draw [rex,semithick,shorten <=6pt,shorten >=2pt,arrows={-Stealth[harpoon,swap]}] (Z2) to[bend left=80,looseness=3] (sb2a);
        \draw [rex,semithick,shorten <=2pt,shorten >=2pt,arrows={-Stealth[harpoon]}] (sb2b) to[bend right=60,looseness=1.5] (sb2a);
        \draw [rex,semithick,shorten <=2pt,shorten >=2pt,arrows={-Stealth[harpoon,flex]}] (sb2b) to[bend left=60,looseness=1.5] (C2);
    }
    \caption*{Resonance fragmentation: Lone pairs.}
\end{figure}
\begin{figure}[h!]
    \centering
    \footnotesize
    \schemestart
        \chemfig{C(-[:120]R)(-[:-120]R')=\charge{[extra sep=1.5pt]45=\:,-45=\:}{O}}
        \arrow{->[ionization][-$\e[-]$]}[,1.4]
        \chemfig{C(-[@{sb2}:120]@{R2}R)(-[:-120]R')=[@{db2}]@{O2}\charge{107=\.,73={\tiny$+$},0=\:}{O}}
        \arrow{->[fragmentation]}[,1.8]
        \chemname{
            \chemleft{[}
                \subscheme{
                    \chemfig{R'-C~\charge{90={\tiny$+$},0=\:}{O}}
                    \arrow(.south--.north){<->}[-90]
                    \chemfig{R'-\charge{90={\tiny$+$}}{C}=\charge{[extra sep=1.5pt]45=\:,-45=\:}{O}}
                }
            \chemright{]}
        }{Acylium ion}
        \arrow{0}[,0]\+
        \chemfig{\charge{0=\.}{R}}
    \schemestop
    \chemmove{
        \draw [rex,semithick,shorten <=2pt,shorten >=2pt,arrows={-Stealth[harpoon,swap,flex]}] (sb2) to[bend right=60,looseness=1.5] (R2);
        \draw [rex,semithick,shorten <=2pt,shorten >=2pt,arrows={-Stealth[harpoon,swap]}] (sb2) to[bend left=50,looseness=1.3] (db2);
        \draw [rex,semithick,shorten <=6pt,shorten >=2pt,arrows={-Stealth[harpoon,flex]}] (O2) to[bend right=70,looseness=2.5] (db2);
    }
    \caption*{Resonance fragmentation: Carbonyls.}
\end{figure}
\begin{figure}[h!]
    \centering
    \footnotesize
    \schemestart
        \chemfig{R-CH(-[@{sb1a}2]H)-[@{sb1b}]CH_2-[@{sb1c,0.2}2]@{O1}\charge{90=\:,163={\tiny$+$},197=\.}{O}H}
        \arrow(.-18--)
        \chemleft{[}
            \chemfig{R-CH-[,,,,lddbond]CH_2}
        \chemright{]^+}
        \+
        \chemfig{H-\charge{90=\:,-90=\:}{O}-H}
    \schemestop
    \chemmove{
        \draw [rex,semithick,shorten <=2pt,shorten >=6pt,arrows={-Stealth[harpoon]}] (sb1a) to[out=135,in=150,out looseness=3] (O1);
        \draw [rex,semithick,shorten <=2pt,shorten >=2pt,arrows={-Stealth[harpoon]}] (sb1a) to[bend left=60,looseness=1.5] (sb1b);
        \draw [rex,semithick,shorten <=2pt,shorten >=2pt] (sb1c) to[out=30,in=-60,looseness=1.5] (O1);
    }
    \caption*{Fragmentation: Loss of \ce{H2O}.}
\end{figure}
\begin{figure}[h!]
    \centering
    \footnotesize
    \schemestart
        \chemfig{Y-[:30]C(=[@{db1}2]\charge{[extra sep=1.5pt]45=\:,118:1pt={\tiny$+$},152=\.}{O})-[@{sb1a}:-30]\chembelow{C}{\hspace{5pt}H_2}-[@{sb1b}:30]CH_2-[@{sb1c}2]CH(-[@{sb1d}:150]@{H1}H)-[:30,,1]R}
        \arrow
        \chemfig{Y-[:30]C(-[2]\charge{90=\:,197={\tiny$+$},163=\.}{O}-[:30]H)=[:-30]CH_2}
        \arrow{0}[,0]\+{1em,1em,2em}
        \chemfig{CH_2=[2]CH-[:30,,1]R}
    \schemestop
    \chemmove{
        \draw [rex,semithick,shorten <=3pt,shorten >=2pt] (db1) to[bend right=50,looseness=1.4] (H1);
        \draw [rex,semithick,shorten <=2pt,shorten >=2pt] (sb1b) to[bend right=60,looseness=1.5] (sb1a);
        \draw [rex,semithick,shorten <=2pt,shorten >=2pt] (sb1d) to[bend right=60,looseness=1.5] (sb1c);
    }
    \vspace{1em}
    \caption*{Fragmentation: McLafferty rearrangement.}
\end{figure}
\begin{itemize}
    \item Alkene reactions to know: hydrogenation (\ce{H2} \ce{Pd}/\ce{C}), dihydroxylation (\ce{OsO4}), ozonolysis (\ce{O3} \ce{Me2S}), hydrobromination (\ce{HBr}), and bromination (\ce{Br2}).
\end{itemize}




\end{document}