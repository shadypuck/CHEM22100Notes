\documentclass[../notes.tex]{subfiles}

\pagestyle{main}
\renewcommand{\chaptermark}[1]{\markboth{\chaptername\ \thechapter\ (#1)}{}}
\setcounter{chapter}{4}

\begin{document}




\chapter{Aromaticity}
\section{Aromaticity 1}
\begin{itemize}
    \item \marginnote{2/8:}Office hours: Tuesday and Friday at 4:00 PM.
    \item PSet 3 is due 2/17.
    \item Aromatic compounds are called such because they're often fragrant. They're heavily associated with biological systems.
    \item History of aromatic comounds.
    \begin{itemize}
        \item 1825: Michael Faraday isolated a compound from his oil lamp having a $\ce{C}:\ce{H}$ ratio of $1:1$.
        \item 1834: Benzoic acid plus heat makes \ce{(CH)_{$n$} + CO2}.
        \begin{itemize}
            \item Even hex-1,3,5-triene still has more hydrogens than carbons.
        \end{itemize}
        \item Benzene.
        \begin{itemize}
            \item There are about 60 possible structures for \ce{C6H6}.
            \item \textbf{Dewer benzene} is two fused 4-member rings with alkenes on opposing sides.
            \item But benzene is highly unreactive in alkene reactions\dots
        \end{itemize}
        \item 1865: Kekul\'{e} proposed a "cyclohexatriene" structure.
        \begin{figure}[h!]
            \centering
            \footnotesize
            \schemestart
                \chemfig{*6(-=-=-=)}
                \arrow{->[\ce{Br2}]}
                \chemfig{*6(-(-[,,,,white]\phantom{Br})=-(-Br)=(-Br)-=)}
                \arrow{0}[,0]\+{,,1em}
                \chemfig{*6(-(-[,,,,white]\phantom{Br})=(-Br)-(-Br)=(-[,,,,white]\phantom{Br})-=)}
            \schemestop
            \vspace{-2em}
            \caption{Bromination of cyclohexatriene.}
            \label{fig:brominationCyclohexatriene}
        \end{figure}
        \begin{itemize}
            \item Evidence: You would expect bromination of cyclohexatriene to produce two products, but it only produces one (the two molecules must be rapidly interconverting, i.e., via resonance).
        \end{itemize}
        \item Chemists began looking for more similar compounds.
        \item 1911: Cyclooctatriene was made.
        \begin{itemize}
            \item It can be hydrogenated, so not consistent with the low reactivity of benzene.
        \end{itemize}
        \item Cyclobutadiene was impossible to isolate due to a self-Diels-Alder reaction at any temperature greater than \SI{-260}{\celsius}.
    \end{itemize}
    \item Enthalpies of hydrogenation.
    \begin{itemize}
        \item Hydrogenation of cyclohexene has $\Delta H=\SI[per-mode=symbol]{-28.6}{\kilo\calorie\per\mole}$.
        \item Hydrogenation of cyclohex-1,4-diene has $\Delta H=\SI[per-mode=symbol]{-57.2}{\kilo\calorie\per\mole}$.
        \item Hydrogenation of cyclohex-1,3-diene has $\Delta H=\SI[per-mode=symbol]{-55.4}{\kilo\calorie\per\mole}$.
        \begin{itemize}
            \item The \SI[per-mode=symbol]{1.8}{\kilo\calorie\per\mole} difference between the previous two comes from conjugation as predicted by resonance.
        \end{itemize}
        \item Hydrogenation of benzene has $\Delta H=\SI[per-mode=symbol]{-49.3}{\kilo\calorie\per\mole}$.
        \begin{itemize}
            \item That is a huge stabilization effect.
        \end{itemize}
    \end{itemize}
    \item The bond lengths in benzene are all equally $\SI{1.39}{\angstrom}$.
    \item MO theory: We need a method to draw the MOs for flat, cyclic conjugated compounds. We will use the \textbf{Frost method}.
    \item For hexa-1,3,5-triene, six $p$-orbitals combine to make six MOs.
    \begin{figure}[h!]
        \centering
        \begin{tikzpicture}
            \LARGE
            \draw [ultra thick]
                (0,0) -- node{$\upharpoonleft$} ++(0.5,0)
                ++(0.1,0) -- node{$\upharpoonleft$} ++(0.5,0)
                ++(0.1,0) -- node{$\upharpoonleft$} ++(0.5,0)
                ++(0.1,0) -- node{$\upharpoonleft$} ++(0.5,0)
                ++(0.1,0) -- node{$\upharpoonleft$} ++(0.5,0)
                ++(0.1,0) -- node{$\upharpoonleft$} ++(0.5,0)
            ;
            \draw [loosely dashed] (3.7,0) -- ++(4,0);
    
            \draw [ultra thick]
                (5.45,3)  -- ++(0.5,0)
                (5.45,2)  -- ++(0.5,0)
                (5.45,1)  -- ++(0.5,0)
                (5.45,-1) -- node{$\upharpoonleft$\hspace{-1mm}$\downharpoonright$} ++(0.5,0)
                (5.45,-2) -- node{$\upharpoonleft$\hspace{-1mm}$\downharpoonright$} ++(0.5,0)
                (5.45,-3) -- node{$\upharpoonleft$\hspace{-1mm}$\downharpoonright$} ++(0.5,0)
            ;
    
            \begin{scope}[yshift=3cm]
                \draw [semithick] (7,0) -- ++(2.5,0);
                \filldraw [semithick,fill=grx]
                    (7,0)   to[out=120,in=60,looseness=150] ++(0.01,0)
                    (7.5,0) to[out=-120,in=-60,looseness=150] ++(0.01,0)
                    (8,0)   to[out=120,in=60,looseness=150] ++(0.01,0)
                    (8.5,0) to[out=-120,in=-60,looseness=150] ++(0.01,0)
                    (9,0)   to[out=120,in=60,looseness=150] ++(0.01,0)
                    (9.5,0) to[out=-120,in=-60,looseness=150] ++(0.01,0)
                ;
                \draw [semithick]
                    (7,0)   to[out=-120,in=-60,looseness=150] ++(0.01,0)
                    (7.5,0) to[out=120,in=60,looseness=150] ++(0.01,0)
                    (8,0)   to[out=-120,in=-60,looseness=150] ++(0.01,0)
                    (8.5,0) to[out=120,in=60,looseness=150] ++(0.01,0)
                    (9,0)   to[out=-120,in=-60,looseness=150] ++(0.01,0)
                    (9.5,0) to[out=120,in=60,looseness=150] ++(0.01,0)
                ;
                \draw [rex,thick,densely dashed]
                    (7.25,0.4) -- ++(0,-0.8)
                    (7.75,0.4) -- ++(0,-0.8)
                    (8.25,0.4) -- ++(0,-0.8)
                    (8.75,0.4) -- ++(0,-0.8)
                    (9.25,0.4) -- ++(0,-0.8)
                ;
            \end{scope}
            \begin{scope}[yshift=2cm]
                \draw [semithick] (7,0) -- ++(2.5,0);
                \filldraw [semithick,fill=grx]
                    (7,0)   to[out=120,in=60,looseness=150] ++(0.01,0)
                    (7.5,0) to[out=-120,in=-60,looseness=150] ++(0.01,0)
                    (8,0)   to[out=120,in=60,looseness=150] ++(0.01,0)
                    (8.5,0) to[out=120,in=60,looseness=150] ++(0.01,0)
                    (9,0)   to[out=-120,in=-60,looseness=150] ++(0.01,0)
                    (9.5,0) to[out=120,in=60,looseness=150] ++(0.01,0)
                ;
                \draw [semithick]
                    (7,0)   to[out=-120,in=-60,looseness=150] ++(0.01,0)
                    (7.5,0) to[out=120,in=60,looseness=150] ++(0.01,0)
                    (8,0)   to[out=-120,in=-60,looseness=150] ++(0.01,0)
                    (8.5,0) to[out=-120,in=-60,looseness=150] ++(0.01,0)
                    (9,0)   to[out=120,in=60,looseness=150] ++(0.01,0)
                    (9.5,0) to[out=-120,in=-60,looseness=150] ++(0.01,0)
                ;
                \draw [rex,thick,densely dashed]
                    (7.25,0.4) -- ++(0,-0.8)
                    (7.75,0.4) -- ++(0,-0.8)
                    (8.75,0.4) -- ++(0,-0.8)
                    (9.25,0.4) -- ++(0,-0.8)
                ;
            \end{scope}
            \begin{scope}[yshift=1cm]
                \draw [semithick] (7,0) -- ++(2.5,0);
                \filldraw [semithick,fill=grx]
                    (7,0)   to[out=120,in=60,looseness=150] ++(0.01,0)
                    (7.5,0) to[out=-120,in=-60,looseness=150] ++(0.01,0)
                    (8,0)   to[out=-120,in=-60,looseness=150] ++(0.01,0)
                    (8.5,0) to[out=120,in=60,looseness=150] ++(0.01,0)
                    (9,0)   to[out=120,in=60,looseness=150] ++(0.01,0)
                    (9.5,0) to[out=-120,in=-60,looseness=150] ++(0.01,0)
                ;
                \draw [semithick]
                    (7,0)   to[out=-120,in=-60,looseness=150] ++(0.01,0)
                    (7.5,0) to[out=120,in=60,looseness=150] ++(0.01,0)
                    (8,0)   to[out=120,in=60,looseness=150] ++(0.01,0)
                    (8.5,0) to[out=-120,in=-60,looseness=150] ++(0.01,0)
                    (9,0)   to[out=-120,in=-60,looseness=150] ++(0.01,0)
                    (9.5,0) to[out=120,in=60,looseness=150] ++(0.01,0)
                ;
                \draw [rex,thick,densely dashed]
                    (7.25,0.4) -- ++(0,-0.8)
                    (8.25,0.4) -- ++(0,-0.8)
                    (9.25,0.4) -- ++(0,-0.8)
                ;
            \end{scope}
            \begin{scope}[yshift=-1cm]
                \draw [semithick] (7,0) -- ++(2.5,0);
                \filldraw [semithick,fill=grx]
                    (7,0)   to[out=120,in=60,looseness=150] ++(0.01,0)
                    (7.5,0) to[out=120,in=60,looseness=150] ++(0.01,0)
                    (8,0)   to[out=-120,in=-60,looseness=150] ++(0.01,0)
                    (8.5,0) to[out=-120,in=-60,looseness=150] ++(0.01,0)
                    (9,0)   to[out=120,in=60,looseness=150] ++(0.01,0)
                    (9.5,0) to[out=120,in=60,looseness=150] ++(0.01,0)
                ;
                \draw [semithick]
                    (7,0)   to[out=-120,in=-60,looseness=150] ++(0.01,0)
                    (7.5,0) to[out=-120,in=-60,looseness=150] ++(0.01,0)
                    (8,0)   to[out=120,in=60,looseness=150] ++(0.01,0)
                    (8.5,0) to[out=120,in=60,looseness=150] ++(0.01,0)
                    (9,0)   to[out=-120,in=-60,looseness=150] ++(0.01,0)
                    (9.5,0) to[out=-120,in=-60,looseness=150] ++(0.01,0)
                ;
                \draw [rex,thick,densely dashed]
                    (7.75,0.4) -- ++(0,-0.8)
                    (8.75,0.4) -- ++(0,-0.8)
                ;
            \end{scope}
            \begin{scope}[yshift=-2cm]
                \draw [semithick] (7,0) -- ++(2.5,0);
                \filldraw [semithick,fill=grx]
                    (7,0)   to[out=120,in=60,looseness=150] ++(0.01,0)
                    (7.5,0) to[out=120,in=60,looseness=150] ++(0.01,0)
                    (8,0)   to[out=120,in=60,looseness=150] ++(0.01,0)
                    (8.5,0) to[out=-120,in=-60,looseness=150] ++(0.01,0)
                    (9,0)   to[out=-120,in=-60,looseness=150] ++(0.01,0)
                    (9.5,0) to[out=-120,in=-60,looseness=150] ++(0.01,0)
                ;
                \draw [semithick]
                    (7,0)   to[out=-120,in=-60,looseness=150] ++(0.01,0)
                    (7.5,0) to[out=-120,in=-60,looseness=150] ++(0.01,0)
                    (8,0)   to[out=-120,in=-60,looseness=150] ++(0.01,0)
                    (8.5,0) to[out=120,in=60,looseness=150] ++(0.01,0)
                    (9,0)   to[out=120,in=60,looseness=150] ++(0.01,0)
                    (9.5,0) to[out=120,in=60,looseness=150] ++(0.01,0)
                ;
                \draw [rex,thick,densely dashed] (8.25,0.4) -- ++(0,-0.8);
            \end{scope}
            \begin{scope}[yshift=-3cm]
                \draw [semithick] (7,0) -- ++(2.5,0);
                \filldraw [semithick,fill=grx]
                    (7,0)   to[out=120,in=60,looseness=150] ++(0.01,0)
                    (7.5,0) to[out=120,in=60,looseness=150] ++(0.01,0)
                    (8,0)   to[out=120,in=60,looseness=150] ++(0.01,0)
                    (8.5,0) to[out=120,in=60,looseness=150] ++(0.01,0)
                    (9,0)   to[out=120,in=60,looseness=150] ++(0.01,0)
                    (9.5,0) to[out=120,in=60,looseness=150] ++(0.01,0)
                ;
                \draw [semithick]
                    (7,0)   to[out=-120,in=-60,looseness=150] ++(0.01,0)
                    (7.5,0) to[out=-120,in=-60,looseness=150] ++(0.01,0)
                    (8,0)   to[out=-120,in=-60,looseness=150] ++(0.01,0)
                    (8.5,0) to[out=-120,in=-60,looseness=150] ++(0.01,0)
                    (9,0)   to[out=-120,in=-60,looseness=150] ++(0.01,0)
                    (9.5,0) to[out=-120,in=-60,looseness=150] ++(0.01,0)
                ;
            \end{scope}
        \end{tikzpicture}
        \caption{Hexa-1,3,5-triene MO diagram.}
        \label{fig:MOhexa-1,3,5-triene}
    \end{figure}
    \begin{itemize}
        \item Three are bonding; three are antibonding. We can guess at what the SALCs look like with a nodal analysis.
    \end{itemize}
    \item For benzene, six $p$-orbitals combine to make six different MOs.
    \begin{itemize}
        \item See Figures III.1 and III.2 in \textcite{bib:CHEM20100Notes}.
        \item For conjugated cyclic systems with an even number of atoms, there will always be a single high and single low MO energy level.
    \end{itemize}
    \item \textbf{Frost method}: The following procedure for drawing MOs for flat cyclic conjugated compounds.
    \begin{enumerate}
        \item Draw a polygon of the molecule without double bonds and with a vertex at the bottom.
        \item Draw a line halfway through the structure.
        \item Put an MO at each vertex.
    \end{enumerate}
    \item For example, if we want to find the MOs of butadiene, we do the following.
    \begin{figure}[H]
        \centering
        \begin{tikzpicture}
            \footnotesize
            \node [xshift=0cm] {\chemfig{*4(-=-=)}};
            \draw [xshift=1cm] [very thick,-stealth] (0.1,0) -- node[above]{1} ++(0.8,0);
            \draw [xshift=3cm] (0.4,0) -- (0,0.4) -- (-0.4,0) -- (0,-0.4) -- cycle;
            \draw [xshift=4cm] [very thick,-stealth] (0.1,0) -- node[above]{2} ++(0.8,0);
            \draw [xshift=6cm] [blx,thick,dashed] (-0.8,0) -- (0.8,0);
            \draw [xshift=6cm] (0.4,0) -- (0,0.4) -- (-0.4,0) -- (0,-0.4) -- cycle;
            \draw [xshift=7cm] [very thick,-stealth] (0.1,0) -- node[above]{3} ++(0.8,0);
            \draw [xshift=9cm] [blx,thick,dashed] (-0.8,0) -- (0.8,0);
            \draw [xshift=9cm] [ultra thick]
                (0.2,0) -- ++(0.4,0)
                (-0.2,0.4) -- ++(0.4,0)
                (-0.6,0) -- ++(0.4,0)
                (-0.2,-0.4) -- ++(0.4,0)
            ;
        \end{tikzpicture}
        \caption{Frost method: Butadiene.}
        \label{fig:frostButadiene}
    \end{figure}
    \item We can even apply this to cyclotetradecaheptaene.
    \begin{figure}[h!]
        \centering
        \begin{tikzpicture}
            \footnotesize
            \node{\chemfig{[:-60]*6(-=-=(*6(-=-(*6(=-=-(*6(=-=)))))))}};
    
            \draw [loosely dashed] (2,0) -- ++(3,0);
            \large
            \draw [ultra thick]
                (3.25,2) -- ++(0.5,0)
                (2.75,1.5) -- ++(0.5,0) ++(0.5,0) -- ++(0.5,0)
                (2.75,1) -- ++(0.5,0) ++(0.5,0) -- ++(0.5,0)
                (2.75,0.5) -- ++(0.5,0) ++(0.5,0) -- ++(0.5,0)
                (2.75,-0.5) -- node{$\upharpoonleft$\hspace{-1mm}$\downharpoonright$} ++(0.5,0) ++(0.5,0) -- node{$\upharpoonleft$\hspace{-1mm}$\downharpoonright$} ++(0.5,0)
                (2.75,-1) -- node{$\upharpoonleft$\hspace{-1mm}$\downharpoonright$} ++(0.5,0) ++(0.5,0) -- node{$\upharpoonleft$\hspace{-1mm}$\downharpoonright$} ++(0.5,0)
                (2.75,-1.5) -- node{$\upharpoonleft$\hspace{-1mm}$\downharpoonright$} ++(0.5,0) ++(0.5,0) -- node{$\upharpoonleft$\hspace{-1mm}$\downharpoonright$} ++(0.5,0)
                (3.25,-2) -- node{$\upharpoonleft$\hspace{-1mm}$\downharpoonright$} ++(0.5,0)
            ;
        \end{tikzpicture}
        \caption{Frost method: Cyclotetradecaheptaene.}
        \label{fig:frostCyclotetradecaheptaene}
    \end{figure}
    \item \textbf{$\bm{(4n+2)}$ rule}: If a system has $4n+2$ $\pi$-electrons for $n\in\N_0$, then it is aromatic.
    \begin{itemize}
        \item Alternatively, if all bonding orbitals are filled and there are no electrons in non-bonding or anti-bonding orbitals, then the compound is aromatic.
    \end{itemize}
    \item \textbf{Anti-aromatic} (molecule): A flat cyclic conjugated molecule with an uninterrupted flow of $p$-orbitals that does not satisfy the $(4n+2)$ rule.
    \begin{itemize}
        \item Alternatively, the molecule must have electrons in non-bonding or antibonding orbitals and six or fewer atoms in the cycle.
    \end{itemize}
    \item \textbf{Non-aromatic} (molecule): A molecule with electrons in non-bonding or antibonding orbitals and seven or more atoms in the cycle.
    \item Unpaired electrons in nonbonding orbitals can be very destabilizing.
    \begin{itemize}
        \item But since cyclooctatetraene is not flat (it's tub-shaped), it avoids the MO overlap that leads to anti-aromaticity.
    \end{itemize}
\end{itemize}



\section{Aromaticity 2}
\begin{itemize}
    \item \marginnote{2/8:}Note that the bond lengths of benzene are equal because the two resonance structures each contribute equally, and we can measure the bond length via x-ray crystallography.
    \item A magnetic field induces the $\pi$-electrons of aromatic compounds to circulate. This motion reinforces the magnetic field, leading to a substantial deshielding effect in NMR experiments.
    \begin{itemize}
        \item Indeed, NMR is one of our key tools for identifying aromatic compounds.
    \end{itemize}
    \item People in the 1820s thought that a compound had to smell to be aromatic. Of course, we now know that smell has nothing to do with chemical aromaticity.
    \item \textbf{H\"{u}ckel's rules}: A set of rules that determines whether or not a compound is aromatic; a shortcut to the Frost diagram method.
    \begin{itemize}
        \item To apply H\"{u}ckel's rule, the molecule in question must be flat, cyclic, and have a $p$-orbital on each atom.
        \item If one of these conditions does not apply, the molecule is simply non-aromatic.
        \item If the conditions do apply, $4n+2$ $\pi$ electrons implies aromaticity and $4n$ ($n\in\N$) electrons implies anti-aromaticity (the number of atoms is less than 6) or non-aromaticity (the number of atoms is greater than 6).
    \end{itemize}
    \item Note that if we chose the bottom vertex of cyclotetradecaheptaene to be any other vertex than the one shown in Figure \ref{fig:frostCyclotetradecaheptaene} (or the one directly opposite it), we would end up with multiple lowest energy MOs (which would be incorrect).
    \item Anti-aromatic molecules react in any way they can to avoid existing in such a state.
    \begin{itemize}
        \item Think of cyclobutadiene doing a self-Diels-Alder reaction to avoid being anti-aromatic.
        \item Cyclooctatetraene is sufficiently big such that it need not react; it can just bend.
    \end{itemize}
    \item Tougher Frost diagrams:
    \begin{itemize}
        \item 5-membered ring: 3 bonding orbitals and 2 anti-bonding orbitals.
        \item 7-membered ring: 3 bonding orbitals and 4 anti-bonding orbitals.
    \end{itemize}
    \item Based on first principles, a structure with a ring system and a number of electrons that makes reasonable sense for aromaticity is aromatic.
    \begin{itemize}
        \item However, in nature, 18 $\pi$-electrons tends to be the upper limit for aromaticity.
        \item Chemists have gone up to 34 $\pi$-electrons and you can go even higher.
        \item As long as the molecule is still flat, if everything else works, it is aromatic.
    \end{itemize}
    \item Today, we will discuss three new classes of molecules that have aromaticity.
    \item Class 1: Anions/cations.
    \begin{itemize}
        \item Treating cyclopentadiene with an appropriate base yields the cyclopentadienyl anion, which is aromatic.
        \begin{itemize}
            \item The cyclopentadienyl anion has five equivalent resonance structures.
            \item The Frost diagram analysis supports this claim, since the three bonding orbitals are completely filled and the two antibonding orbitals are empty.
        \end{itemize}
        \item Treating cycloheptatriene with the trityl cation abstracts a hydride leaving the tropylium ion, which is aromatic.
        \begin{figure}[h!]
            \centering
            \footnotesize
            \chemnameinit{\chemfig{\charge{45:3pt=$\oplus$}{}(-[:-30]*6(=-=-=-))(-[2]*6(-=-=-=))(-[:210]*6(=-=-=-))}}
            \schemestart
                \chemname{\chemfig{[:-64.29]*7(=-=-=-(-[@{sb1}:70]H)(-[:110]H)-)}}{Cycloheptatriene}
                \+{2em,2em}
                \chemname{\chemfig{@{C2}\charge{45:3pt=$\oplus$}{}(-[:-30]*6(=-=-=-))(-[2]*6(-=-=-=))(-[:210]*6(=-=-=-))}}{Trityl cation}
                \arrow(.mid east--.mid west){->[][-\ce{CHPh3}]}[,1.3]
                \chemname{\chemfig{[:-64.29]*7(=-=-=-\charge{90:3pt=$\oplus$}{}-)}}{Tropylium ion}
            \schemestop
            \chemnameinit{}
            \chemmove{
                \draw [rex,semithick,shorten <=2pt,shorten >=2pt] (sb1) to[out=-30,in=150,out looseness=0.6,in looseness=1.3] (C2);
            }
            \caption{Aromaticity in the tropylium ion.}
            \label{fig:aromaticityTropylium}
        \end{figure}
    \end{itemize}
    \item When we have substituted compounds, we only care about the $\pi$-electrons in the ring system.
    \begin{itemize}
        \item For example, in tropone, we only count seven $\pi$-electrons.
        \begin{figure}[H]
            \centering
            \footnotesize
            \schemestart
                \chemfig{[:38.57]*7(=-=-(=[@{db1}]@{O1}O)-=-)}
                \arrow{<->}
                \chemfig{[:38.57]*7(=-=-\charge{-90:3pt=$\oplus$}{}(-\charge{45=$\ominus$}{O})-=-)}
                \arrow{<->}
                \chemfig{[:38.57]**7(----(-\charge{45=$\ominus$}{O})---)}
            \schemestop
            \chemmove{
                \draw [rex,semithick,shorten <=3pt,shorten >=2pt] (db1) to[bend right=90,looseness=3] (O1);
                \node at (cyclecenter1) {$+$};
            }
            \caption{Aromaticity in tropone.}
            \label{fig:aromaticityTropone}
        \end{figure}
        \begin{itemize}
            \item However, if a molecule can become aromatic, it will. Thus, the actual structure of tropone is the right resonance structures above.
        \end{itemize}
    \end{itemize}
    \item We can indeed have $2\pi$-electron aromatic systems \parencite{bib:Breslow}.
    \begin{figure}[h!]
        \centering
        \footnotesize
        \setchemfig{atom sep=3em}
        \schemestart
            \chemfig{[:-30]*3(=(-[6,0.7,,,white]\phantom{Cl})-(-[,0.7]Cl)-)}
            \arrow{->[\ce{SbCl5}][-\ce{SbCl6-}]}[,1.3]
            \chemfig{[:-30]**3(---)}
        \schemestop
        \chemmove{
            \node at (cyclecenter1) {+};
        }
        \vspace{-2.5em}
        \caption{Aromaticity in the cyclopropenyl ion.}
        \label{fig:aromaticityCyclopropenyl}
    \end{figure}
    \begin{itemize}
        \item This was the crowning achievement of a push in the 1950s-60s by organic chemists to push the bounds of aromatic compounds. It was done by Ron Breslow of Columbia in 1967.
    \end{itemize}
    \item Joined rings can also rearrange \`{a} la tropone (Figure \ref{fig:aromaticityTropone}) into an aromatic system.
    \begin{figure}[h!]
        \centering
        \footnotesize
        \schemestart
            \chemfig{[:38.57]*7(=-=-(=[@{db1}]@{C1}*5(-=-=-))-=-)}
            \arrow{<->}
            \chemfig{[:38.57]*7(=-=-\charge{-90:3pt=$\oplus$}{}(-\charge{90:3pt=$\ominus$}{}*5(-=-=-))-=-)}
            \arrow{<->}
            \chemfig{[:38.57]**7(----(-**5(-----))---)}
        \schemestop
        \chemmove{
            \draw [rex,semithick,shorten <=3pt,shorten >=3pt] (db1) to[bend right=90,looseness=4] (C1);
            \node at (cyclecenter1) {$+$};
            \node at (cyclecenter2) {$-$};
        }
        \caption{Aromaticity in sesquifulvalene.}
        \label{fig:aromaticitySesquifulvalene}
    \end{figure}
    \begin{itemize}
        \item Although we can call this molecule aromatic overall, it would be better to say each ring is separately aromatic.
        \item In problems like this, get an initial electron count first (five for the top ring and seven for the bottom ring). This will then provide information about where you need to push electrons to create aromaticity. For example, seven is one too high and five is one too low, so we give one electron from the seven ring to the five ring to create two rings with six electrons.
        \item Not all such systems do, however: Fulvalene\footnote{Fulvalene looks exactly like sesquifulvalene, except that both rings have only five carbons.}, for instance would have to rearrange into an aromatic ring \emph{and} an anti-aromatic ring, so it foregoes any rearrangement and is actually non-aromatic.
        \item Take-away: If one ring becomes aromatic and one remains non-aromatic, that's fine. If both rings become aromatic, that's great. If one ring would have to become anti-aromatic for the other to become aromatic, that will not happen.
    \end{itemize}
    \item \textbf{Heterocyclic compound}: A cyclic compound containing atoms other than carbon and hydrogen. \emph{Also known as} \textbf{heterocycle}.
    \item \textbf{Heteroatom}: Any atom that is not carbon or hydrogen.
    \begin{itemize}
        \item Commonly oxygen, sulfur, or nitrogen.
    \end{itemize}
    \item Class 2: Heterocyclic compounds.
    \begin{figure}[H]
        \centering
        \footnotesize
        \begin{subfigure}[b]{0.15\linewidth}
            \centering
            \chemfig{*6(-N=-=-=)}
            \caption{Pyridine.}
            \label{fig:commonHeterocyclesa}
        \end{subfigure}
        \begin{subfigure}[b]{0.15\linewidth}
            \centering
            \chemfig{[:-18]*5(-O-=-=)}
            \caption{Furan.}
            \label{fig:commonHeterocyclesb}
        \end{subfigure}
        \begin{subfigure}[b]{0.15\linewidth}
            \centering
            \chemfig{[:-18]*5(-\chembelow{N}{H}-=-=)}
            \vspace{1em}
            \caption{Pyrrole.}
            \label{fig:commonHeterocyclesc}
        \end{subfigure}
        \begin{subfigure}[b]{0.15\linewidth}
            \centering
            \chemfig{[:-18]*5(-\chembelow{N}{H}-=N-=)}
            \vspace{1em}
            \caption{Imidazole.}
            \label{fig:commonHeterocyclesd}
        \end{subfigure}
        \caption{Common heterocyclic compounds.}
        \label{fig:commonHeterocycles}
    \end{figure}
    \item An analysis of pyridine.
    \begin{figure}[h!]
        \centering
        \begin{tikzpicture}[scale=1.3]
            \footnotesize
            \begin{scope}[rotate=2.8,yscale=0.4,rotate=-7]
                \draw (0:1) node(N)[fill=white]{\ce{N}}
                    -- (60:1)  coordinate (C1)
                    -- (120:1) coordinate (C2)
                    -- (180:1) coordinate (C3)
                    -- (240:1) coordinate (C4)
                    -- (300:1) coordinate (C5)
                    -- cycle
                ;
                \draw [blx,shorten <=4pt] (0:0.9) -- (60:0.9);
                \draw [blx]
                    (120:0.85) -- (180:0.85)
                    (240:0.8) -- (300:0.8)
                ;
            \end{scope}
    
            \draw
                (N.70) to[out=70,in=110,looseness=10] (N.110) (N.-70) to[out=-70,in=-110,looseness=10] (N.-110)
                (C1) to[out=120,in=60,looseness=150] ++(0.01,0) to[out=-60,in=-120,looseness=150] cycle
                (C2) to[out=120,in=60,looseness=150] ++(0.01,0) to[out=-60,in=-120,looseness=150] cycle
                (C3) to[out=120,in=60,looseness=150] ++(0.01,0) to[out=-60,in=-120,looseness=150] cycle
                (C4) to[out=120,in=60,looseness=150] ++(0.01,0) to[out=-60,in=-120,looseness=150] cycle
                (C5) to[out=120,in=60,looseness=150] ++(0.01,0) to[out=-60,in=-120,looseness=150] cycle
            ;
            \node [above=4mm,circle,fill=blx,inner sep=0.5pt] at (N) {};
            \node [above=3mm,circle,fill=blx,inner sep=0.5pt] at (C1) {};
            \node [above=3mm,circle,fill=blx,inner sep=0.5pt] at (C2) {};
            \node [above=3mm,circle,fill=blx,inner sep=0.5pt] at (C3) {};
            \node [above=3mm,circle,fill=blx,inner sep=0.5pt] at (C4) {};
            \node [above=3mm,circle,fill=blx,inner sep=0.5pt] at (C5) {};
    
            \draw (N.30) to[out=30,in=-30,looseness=10] (N.-30);
            \node [right=6mm,yshift=0.5mm,circle,fill=blx,inner sep=0.5pt] at (N) {};
            \node [right=6mm,yshift=-0.5mm,circle,fill=blx,inner sep=0.5pt] at (N) {};
        \end{tikzpicture}
        \caption{The structure of pyridine.}
        \label{fig:pyridineStructure}
    \end{figure}
    \begin{itemize}
        \item The three double bonds in pyridine contribute the six $\pi$-electrons necessary for it to be aromatic.
        \item Importantly, this means that the lone pair of nitrogen is \emph{not} needed for aromaticity, so it sits outside the compound in an $sp^2$ orbital.
        \item The fact that this lone pair is free implies that pyridine is an excellent base.
    \end{itemize}
    \item An analysis of pyrrole.
    \begin{itemize}
        \item The two double bonds \emph{plus} the lone pair of the nitrogen constitute the six $\pi$-electrons necessary for it to be aromatic.
        \begin{itemize}
            \item Although VSEPR theory suggests that the nitrogen would be $sp^3$ hybridized so as to get all electron pairs as far away as possible, the increase in energy by rehybridizing to $sp^2$ is more than compensated for by the aromatic stabilization energy.
        \end{itemize}
        \item For this reason, pyrrole is \emph{not} a good base.
        \item Indeed, if the nitrogen picks up another hydrogen, you lose aromaticity and introduce a $+1$ formal charge on the nitrogen. Thus, since the hydrogen adduct is so unstable, it is a strong acid on the order of \ce{HCl} ($\pKa=0.9$).
    \end{itemize}
    \item An analysis of furan.
    \begin{itemize}
        \item The two double bonds plus one of the lone pairs of the oxygen constitute the six $\pi$-electrons necessary for it to be aromatic.
        \item However, there is still a lone pair left over on the oxygen, so furan can still act like a base.
    \end{itemize}
    \item Adenine is a heterocyclic aromatic compound with 10 $\pi$-electrons. Some of its nitrogens contribute their lone pair electrons to the $\pi$-system, and others have them free to act as bases.
    \item Heterocycles can be anti-aromatic as well.
    \begin{figure}[h!]
        \centering
        \footnotesize
        \chemfig{[:-18]*5(-B(-CH_3)-=-=)}
        \caption{An anti-aromatic heterocycle.}
        \label{fig:antiaromatiHeterocycle}
    \end{figure}
    \begin{itemize}
        \item Boron is happy with three bonds, so it has an empty $p$-orbital.
        \item Thus, this is a flat cyclic molecule with an uninterrupted chain of $p$ orbitals and $4n$ ($n=1$) $\pi$-electrons. But this implies that it is anti-aromatic.
    \end{itemize}
    \item Degrees of aromaticity.
    \begin{itemize}
        \item Benzene is "the most" aromatic compound.
        \item All molecules with heteroatoms will have slightly different bond lengths and thus a lesser stabilization energy.
        \item Indeed, under forcing enough conditions, we can make some of the heteroaromatics actually do reactions.
        \begin{itemize}
            \item For example, we can make furan do a Diels-Alder reaction at very high temperatures; this is never something we would see with benzene.
        \end{itemize}
    \end{itemize}
    \item Class 3: Polycyclic aromatic hydrocarbons (PAHs).
    \begin{figure}[h!]
        \centering
        \footnotesize
        \begin{subfigure}[b]{0.24\linewidth}
            \centering
            \chemfig{*6(=-(*6(-=-=-))=-=-)}
            \caption{Naphthalene.}
            \label{fig:commonPAHsa}
        \end{subfigure}
        \begin{subfigure}[b]{0.24\linewidth}
            \centering
            \chemfig{*6(=-(*6(-=(*6(-=-=-))-=-))=-=-)}
            \caption{Anthracene.}
            \label{fig:commonPAHsb}
        \end{subfigure}
        \begin{subfigure}[b]{0.24\linewidth}
            \centering
            \chemfig{*6(=-(*6(-=-(*6(-=-=-))=-))=-=-)}
            \caption{Phenanthrene.}
            \label{fig:commonPAHsc}
        \end{subfigure}
        \begin{subfigure}[b]{0.24\linewidth}
            \centering
            \chemfig{*6(-=(*6(-=-(*6(=-=-(*6(=-=))-))-=))---=)}
            \caption{Pyrene.}
            \label{fig:commonPAHsd}
        \end{subfigure}
        \caption{Common PAHs.}
        \label{fig:commonPAHs}
    \end{figure}
    \begin{itemize}
        \item Naphthalene is used in mothballs.
        \item These names won't be tested, but they're useful to know.
    \end{itemize}
    \item An analysis of naphthalene.
    \begin{figure}[h!]
        \centering
        \footnotesize
        \chemfig{*6()}
        \setchemfig{autoreset cntcycle=false}
        \schemestart
            \chemfig{*6(-=(*6(-=-=-))-=-=)}
            \arrow{<->}
            \chemfig{*6(=-(*6(-=-=-))=-=-)}
            \arrow{<->}
            \chemfig{*6(=-(*6(=-=-=))--=-)}
        \schemestop
        \chemfig{*6()}
        \chemmove{
            \draw [orx,semithick,dashed]
                ([xshift=7mm,yshift=7mm]cyclecenter2) rectangle ([xshift=-7mm,yshift=-7mm]cyclecenter2)
                ([xshift=7mm,yshift=7mm]cyclecenter4) rectangle ([xshift=-7mm,yshift=-7mm]cyclecenter4)
                ([xshift=7mm,yshift=7mm]cyclecenter5) rectangle ([xshift=-7mm,yshift=-7mm]cyclecenter5)
                ([xshift=7mm,yshift=7mm]cyclecenter7) rectangle ([xshift=-7mm,yshift=-7mm]cyclecenter7)
            ;
        }
        \caption{The structure of naphthalene.}
        \label{fig:naphthaleneStructure}
    \end{figure}
    \begin{itemize}
        \item Although Figure \ref{fig:commonPAHsa} shows that naphthalene is a 10 $\pi$-electron system, it can be useful to think of it as two separate benzene rings.
        \item Doing so and drawing all resonance structures reveals that each ring only appears as benzene (as opposed to a diene) $2/3$ of the time.
        \begin{itemize}
            \item Every occurrence of a ring as benzene is boxed in Figure \ref{fig:naphthaleneStructure}. Notice how each ring is boxed twice and not boxed once (across the three resonance structures).
        \end{itemize}
        \item Thus, the aromatic stabilization of naphthalene is not twice benzene's $\SI[per-mode=symbol]{-36.5}{\kilo\calorie\per\mole}$ but rather $\frac{2}{3}\cdot 2\approx 1.33$ times benzene's $\SI[per-mode=symbol]{-36.5}{\kilo\calorie\per\mole}$.
        \begin{itemize}
            \item If we assign benzene an aromaticity value of 1, we would assign naphthalene 1.33.
        \end{itemize}
        \item The bonds in naphthalene alternate between $\SI{1.36}{\angstrom}$ and $\SI{1.42}{\angstrom}$, a $\SI{0.03}{\angstrom}$ perturbation from the bond lengths in benzene.
    \end{itemize}
    \item \textbf{Rocks of Gibraltar}: The molecules benzene, naphthalene, and pyrene, which in general will not undergo further chemical reactions due to the extent of their aromatic stabilization.
    \item An analysis of pyrene.
    \begin{figure}[H]
        \centering
        \footnotesize
        \chemfig{@{C1}*6(-@{C2}=@{C3}(*6(-@{C4}=@{C5}-@{C6}(*6(=@{C7}-@{C8}=@{C9}-@{C10}(*6(=@{C11}-@{C12}=@{C13}))-))-=))---@{C14}=)}
        \chemmove{
            \draw [blx,-,line width=6mm,opacity=0.3] plot[smooth cycle] coordinates{(C1) (C2) (C3) (C4) (C5) (C6) (C7) (C8) (C9) (C10) (C11) (C12) (C13) (C14)};
        }
        \caption{The structure of pyrene.}
        \label{fig:pyreneStructure}
    \end{figure}
    \begin{itemize}
        \item Pyrene is aromatic, but it appears to have $16=4n$ $\pi$-electrons.
        \item However, since one of the criteria for aromaticity is a \emph{cyclic} chain of $\pi$-orbitals, it is actually only the $14=4n+2$ $\pi$-electrons around the periphery that constitute the aromatic system. The $\pi$-bond in the center of the molecule is just a lone alkene with no aromatic stabilization.
        \item You can hydrogenate the central double bond at very high pressures, but essentially for all intents and purposes, pyrene is a nonreactive molecule.
    \end{itemize}
    \item Diels-Alder reactivity of anthracene.
    \begin{figure}[h!]
        \centering
        \footnotesize
        \begin{tikzpicture}[scale=1.4]
            \node at (-3,0.5) {
                \schemestart
                    \chemfig{*6(-=(*6(-=(*6(-=-=-))-=-))-=-=)}
                    \arrow{0}[,0]\+{1.5em,1em,3.3em}
                    \chemfig{CO_2Me-[2]~[2]-[2]CO_2Me}
                    \arrow{->[$\Delta$]}
                \schemestop
            };
    
            \begin{scope}[yscale=0.5,rotate=-10]
                \draw (30:0.6) coordinate (C1a) -- (90:0.6) -- (150:0.6) -- (210:0.6) -- (270:0.6) -- (330:0.6) coordinate (C1b) -- cycle;
                \draw
                    (30:0.47) -- (90:0.47)
                    (150:0.5) -- (210:0.5)
                    (270:0.47) -- (330:0.47)
                ;
            \end{scope}
            \begin{scope}[xshift=2.15cm,yshift=-0.1cm,yscale=0.5,rotate=-10]
                \draw (30:0.6) -- (90:0.6) -- (150:0.6) coordinate (C2a) -- (210:0.6) coordinate (C2b) -- (270:0.6) -- (330:0.6) -- cycle;
                \draw
                    (30:0.47) -- (90:0.47)
                    (150:0.5) -- (210:0.5)
                    (270:0.47) -- (330:0.47)
                ;
            \end{scope}
    
            \draw
                (C1a) -- ++(0.6,0.2) -- (C2a)
                (C1b) -- ++(0.6,0.2) -- (C2b)
            ;
    
            \draw [white,very thick,double=black,double distance=0.4pt]
                (C1a) ++(0.6,0.2) -- ++(0,0.6)
                (C1b) ++(0.6,0.2) -- ++(0,0.6)
            ;
            \draw ([xshift=6mm,yshift=8mm]C1b) -- ([xshift=6mm,yshift=8mm]C1a);
            \draw [shorten >=10pt] ([xshift=6.5mm,yshift=7.5mm]C1b) -- ([xshift=6.5mm,yshift=7.5mm]C1a);
    
            \draw
                ([xshift=6mm,yshift=8mm]C1a) -- ++(0.07,0.4) node[above,xshift=4mm]{\ce{CO2Me}}
                ([xshift=6mm,yshift=8mm]C1b) -- ++(-0.07,0.4) node[above,xshift=-5mm]{\ce{MeO2C}}
            ;
        \end{tikzpicture}
        \caption{Diels-Alder reactivity of anthracene.}
        \label{fig:dielsAlderAnthracene}
    \end{figure}
    \begin{itemize}
        \item Anthracene is finally destabilized enough to react in a Diels-Alder reaction.
        \item When we look to predict products, we want to maximize the amount of aromaticity left over after the reaction (because this will be the most stable product).
        \item If we perform the Diels-Alder reaction with the central diene, the product will have two benzene rings.
    \end{itemize}
    \item The char marks on grilled meat contain a number of PAHs, notably benzopyrene.
    \begin{itemize}
        \item Benzopyrene is one ring too far to be stable.
        \item Since it is flat, it can intercolate in our DNA and cause a lot of issues, notably with regulating the cell cycle.
        \item Thus, our bodies want to get rid of it, so it sends enzymes to epoxidize the benzene hanging off the pyrene.
        \item Now that the molecule is polar, it can be excreted, but this is a risky strategy because epoxides are highly reactive and can damage other things.
    \end{itemize}
    \item Vioxx vs. Celebrex.
    \begin{itemize}
        \item It is possible that the reason that Vioxx is harmful and Celebrex is not is that Vioxx has a ring that is not aromatic whereas all of Celebrex's rings are aromatic.
    \end{itemize}
    \item Special considerations.
    \begin{figure}[H]
        \centering
        \footnotesize
        \begin{subfigure}[b]{0.3\linewidth}
            \centering
            \chemfig{[:-30]*6(---(-~-*6(-----(-~-*6(-----(-~-)=))=))=--)}
            \caption{}
            \label{fig:specialConsiderationsa}
        \end{subfigure}
        \begin{subfigure}[b]{0.3\linewidth}
            \centering
            \chemfig{[:-180]*6(-=-=-(*6(=-=-=(-[7,0.7]H)))(-[3,0.7]H))}
            \caption{}
            \label{fig:specialConsiderationsb}
        \end{subfigure}
        \caption{Special considerations for determining aromaticity.}
        \label{fig:specialConsiderations}
    \end{figure}
    \begin{itemize}
        \item Triple bonds: A triple bond in the $\pi$-system still only contributes two electrons. This is because its other two $\pi$ electrons are perpendicular to the $\pi$-system in question.
        \begin{itemize}
            \item The compound in Figure \ref{fig:specialConsiderationsa} has 12 $\pi$-electrons but is still non-aromatic because the number of atoms is greater than six (it can bend).
        \end{itemize}
        \item Sterics: If sterics prevent a molecule from being flat, it cannot be aromatic.
        \begin{itemize}
            \item The compound in Figure \ref{fig:specialConsiderationsb} cannot lie flat due to the steric clashing of the two indicated hydrogens. Note that this clashing is unavoidable due to the conformation and the lack of freedom of rotation about the double bonds.
        \end{itemize}
    \end{itemize}
    \item Next week is all about the reactions of benzene.
\end{itemize}



\section{Chapter 14: Aromatic Compounds}
\emph{From \textcite{bib:SolomonsEtAl}.}
\begin{itemize}
    \item \marginnote{2/16:}Nomenclature.
    \begin{itemize}
        \item "In many simple compounds, \emph{benzene} is the parent name and the substituent is simply indicated by a prefix" \parencite[619]{bib:SolomonsEtAl}.
        \begin{itemize}
            \item Examples: Flurorbenzene, chlorobenzene, nitrobenzene.
        \end{itemize}
        \item Other simple compounds have commonly accepted parent names.
        \begin{itemize}
            \item Examples: Methylbenzene $\to$ toluene, hydroxybenzene $\to$ phenol, aminobenzene $\to$ aniline.
            \item Other compounds to be aware of: Benzenesulfonic acid (\ce{C6H5SO3H}), benzoic acid (\ce{C6H5COOH}), acetophenone (\ce{C6H5COCH3}), and anisole (\ce{C6H5OCH3}).
        \end{itemize}
        \item Covers ortho, meta, para naming.
        \item "When a substituent is one that together with the benzene ring gives a new base name, that substituent is assumed to be in position 1 and the new parent name is used" \parencite[620]{bib:SolomonsEtAl}.
        \begin{itemize}
            \item This means that molecules such as m-nitrobenzoic acid are named 3-nitrobenzoic acid.
        \end{itemize}
        \item Dimethylbenzenes are often known as xylenes.
        \item Phenylmethyl becomes benzyl.
    \end{itemize}
    \item The only alkene chemistry in which benzene participates is hydrogenation (in the presence of finely divided nickel, under high temperatures and pressures).
    \item \textbf{Benzene substitution}: The substitution of one of the hydrogens of benzene for a bromine, as initiated by the presence of a Lewis acid catalyst such as ferric bromide (\ce{FeBr3}).
    \begin{itemize}
        \item Explanation: All hydrogens are equivalent and replacing any one of them with bromine results in the same product.
        \item Possible explanation: Only one of benzene's hydrogens is reactive.
        \begin{itemize}
            \item Wrong, though --- ruled out by the structure of benzene but plausible when we didn't know its structure.
        \end{itemize}
    \end{itemize}
    \item The Kekul\'{e} structure for benzene (cyclohexatriene) satisfied the requirements but failed for the reason of Figure \ref{fig:brominationCyclohexatriene}.
    \begin{itemize}
        \item Kekul\'{e} proposed a rapid equilibrium between the structures (resonance), but today we prefer the explanation of delocalization.
    \end{itemize}
    \item A new meaning of aromaticity: Aromatic compounds are highly unsaturated compounds that prefer substitution chemistry to addition chemistry.
    \item Richard Willst\"{a}tter first synthesized cyclooctatetraene in 1911.
    \item \textbf{Resonance energy}: The difference between the amount of heat actually released and that calculated on the basis of the Kekul\'{e} structure.
    \item "Resonance contributors, we emphasize again, are not in equilibrium. They are not structures of real molecules. They are the closest we can get if we are bound by simple rules of valence, but they are very useful in helping us visualize the actual molecule as a hybrid" \parencite[625]{bib:SolomonsEtAl}.
    \item It was recently discovered that "crystalline benzene involves perpendicular interactions between benzene rings, so that the relatively positive periphery of one molecule associates with the relatively negative faces of the benzene molecules aligned above and below it" \parencite[627]{bib:SolomonsEtAl}.
    \item In 1931, Erich H\"{u}ckel carried out a series of quantum mechanical calculations that concluded that planar monocyclic rings containing $4n+2$ $\pi$-electrons have \textbf{closed shells} of delocalized electrons (like benzene) and therefore have substantial resonance energies.
    \item \textbf{Closed shell}: A set of molecular orbitals that are all either completely occupied or completely empty (i.e., no MO in the set contains only one electron).
    \begin{itemize}
        \item Molecules that lack closed shells have unpaired electrons (radicals) and are usually not stable.
    \end{itemize}
    \item The \textbf{polygon-and-circle method} was developed by C. A. Coulson of Oxford university as a simple method of deriving the same energy levels that the quantum mechanical calculations of H\"{u}ckel would furnish.
    \item \textbf{Polygon-and-circle method}: The following procedure.
    \begin{enumerate}
        \item We start by drawing a polygon corresponding to the number of carbons in the ring, placing a corner of the polygon at the bottom.
        \item Next, we surround the polygon with a circle that touches each corner of the polygon (the circumcircle).
        \item At the points where the polygon touches the circle, we draw short horizontal lines outside the circle. The height of each line represents the relative energy of each $\pi$ molecular orbital.
        \item Next, we draw a dashed horizontal line across and halfway up the circle. The energies of bonding $\pi$ molecular orbitals are below this line. The energies of antibonding $\pi$ molecular orbitals are above, and those for nonbonding orbitals are at the level of the dashed line.
        \item Based on the number of $\pi$ electrons in the ring, we then place electron arrows on the lines corresponding to the respective orbitals, beginning at the lowest energy level and working upward. In doing so, we fill degenerate orbitals each with one electron first, then add to each unpaired electron another with opposite spin if it is available.
    \end{enumerate}
    \item \textbf{Annulene}: A monocyclic compound that can be represented by a structure having alternating single and double bonds.
    \begin{itemize}
        \item The ring size of an annulene is indicated by a number in brackets.
        \item For example, benzene is [6]annulene and cyclooctatetraene is [8]annulene.
    \end{itemize}
    \item H\"{u}ckel's rule predicts that annulenes are aromatic iff they have $4n+2$ $\pi$-electrons.
    \begin{itemize}
        \item This prediction was verified in the 1960s (largely by F. Sondheimer) as numerous new annulenes became available for testing.
        \item Annulenes 14-24 satisfy H\"{u}ckel's prediction.
        \item Annulenes 10-12 are too strained to be planar, regardless of double bond placement (see Figure \ref{fig:specialConsiderationsb}).
    \end{itemize}
    \item The \ce{{}^1H} NMR spectrum of benzene supports both equivalent hydrogens (only a singlet appears) and the cyclic nature of the $\pi$-system (the high chemical shift is indicative of a ring current).
    \begin{itemize}
        \item {}[18]annulene has six hydrogens within its ring and twelve hydrogens at the periphery. Because of the shape of the ring current, the internal hydrogens are highly shielded ($\delta\ -3.0$) and the external hydrogens are highly deshielded ($\delta\ 9.3$).
        \item NMR spectroscopy provides direct physical evidence of whether or not the $\pi$-electrons are delocalized.
    \end{itemize}
    \item Cycloheptatriene is also commonly known as tropylidene.
    \item To evaluate the stabilization (or lack thereof) of a cyclic compound with delocalized $\pi$ electrons, we compare it to a conjugated linear model and consider what would happen (theoretically or experimentally) if we removed a hydrogen from each end of the linear compound to form a ring.
    \begin{itemize}
        \item Such calculations/experiments are beyond the scope of \textcite{bib:SolomonsEtAl}.
    \end{itemize}
    \item \textbf{Benzenoid polycyclic aromatic hydrocarbon}: A molecule having two or more benzene rings fused together.
    \item \textbf{Nonbenzenoid aromatic compound}: A compound that is either the cyclopentadienyl anion, the cycloheptatrienyl cation, \emph{trans}-15,16-dimethyldihydropyrene, or an aromatic annulene (except [6]annulene).
    \item \textcite{bib:SolomonsEtAl} briefly discusses fullerenes, such as buckyballs.
    \item \textcite{bib:SolomonsEtAl} discusses applications of aromatic compounds to biochemistry. In particular, it discusses NADH and \ce{NAD+}.
    \item Discusses the infrared absorptions of aromatic compounds (not covered in class, but potentially relevant?).
\end{itemize}




\end{document}