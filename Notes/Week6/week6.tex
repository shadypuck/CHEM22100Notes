\documentclass[../notes.tex]{subfiles}

\pagestyle{main}
\renewcommand{\chaptermark}[1]{\markboth{\chaptername\ \thechapter\ (#1)}{}}
\setcounter{chapter}{5}

\begin{document}




\chapter{Electrophilic Aromatic Substitution}
\section{Electrophilic Aromatic Substitution 1}
\begin{itemize}
    \item \marginnote{2/15:}Discusses the aromaticity of fluorescein as an example to review from last class.
    \item Reactions of aromatic compounds are divided into two classes: Electrophilic and nucleophilic aromatic substitutions.
    \item Example:
    \begin{itemize}
        \item \ce{C6H6 ->[H3O+] C6H6} means no reaction?
        \item \ce{C6H6 ->[D3O+] C6D6}; thus, a substitution is occurring.
    \end{itemize}
    \item Mechanism:
    \begin{figure}[h!]
        \centering
        \footnotesize
        \schemestart
            \chemfig{*6(-=-=[@{db1}]-=)}
            \arrow{->[\chemfig{@{D2}\charge{45=$\oplus$}{D}}]}[,1.3]
            \chemfig{*6(-=-(-[@{sb3a}:10]@{H3}H)(-[:50]D)-[@{sb3b}]\charge{90:3pt=$\oplus$}{}-=)}
            \arrow{->[\chemfig{@{O4}\charge{90=\:}{O}D_2}][-\ce{HD2O+}]}[,1.3]
            \chemfig{*6(-=-(-D)=-=)}
        \schemestop
        \chemmove{
            \draw [rex,semithick,shorten <=2pt,shorten >=2pt] (db1) to[bend left=60,looseness=1.5] (D2);
            \draw [rex,semithick,shorten <=6pt,shorten >=2pt] (O4) to[out=90,in=60,looseness=2] (H3);
            \draw [rex,semithick,shorten <=2pt,shorten >=2pt] (sb3a) to[bend left=90,looseness=3] (sb3b);
        }
        \caption{Electrophilic aromatic substitution mechanism.}
        \label{fig:EASmechanism}
    \end{figure}
    \begin{itemize}
        \item To begin, one of the $\pi$ bonds of benzene attacks \ce{D+}. This causes the loss of aromaticity, but the carbocation is highly resonance delocalized.
        \item Although we \emph{could} make an alcohol at this point, this would lead to the loss of aromaticity in the product, so we won't do that.
        \item Instead, we do an E1-type reaction.
        \item The first step is the RDS.
        \item The intermediate in this mechanism is called the \textbf{arenium ion}, the \textbf{Wheland intermediate}, or the \textbf{sigma complex}.
        \item Note that the electrophile used in this reaction has to be a very special, very reactive, very strong electrophile in order to make up the energy gap.
    \end{itemize}
    \item We know that the sigma complex exists because we can trap the intermediate.
    \item Whether or not we see the product react again depends on whether the product or starting material is more nucleophilic.
    \item Adding an EDG to the benzene makes the reaction proceed faster.
    \begin{itemize}
        \item A good EDG will stabilize the arenium ion, lowering the activation barrier of the first step (the RDS).
    \end{itemize}
    \item Halogenation.
    \item General form.
    \begin{equation*}
        \ce{PhH + Br2 ->[\text{cat.} FeBr3] PhBr + HBr}
    \end{equation*}
    \begin{itemize}
        \item \ce{Br2} is too unreactive to have chemistry with benzene on its own.
        \item In particular, when we say that \ce{Br2} is too unreactive, we mean that there is not enough \ce{Br+} character, i.e., it is not a good enough electrophile.
        \item To overcome the problem, we add \ce{Br2} to \ce{FeBr3}, a good Lewis acid with an open valence site. It follows that \ce{Br-Br^+-Fe^-Br3} is a super awesome electrophile!
    \end{itemize}
    \item Mechanism.
    \begin{figure}[h!]
        \centering
        \footnotesize
        \schemestart
            \chemfig{*6(-=-=-=)}
            \arrow{0}[,0.1]\+
            \chemfig{Br-@{Br2}\charge{90=\:}{Br}}
            \arrow{->[\chemfig{@{Fe3}FeBr_3}]}[,1.1]
            \chemfig{*6(-=-=[@{db4}]-=)}
            \arrow{0}[,0.1]\+
            \chemfig{@{Br5a}Br-[@{sb5}]@{Br5b}\charge{90:3pt=$\oplus$}{Br}-\charge{90:3pt=$\ominus$}{Fe}Br_3}
            \arrow
            \subscheme{
                \chemfig{*6(-=(-[:-50,,,,white]\phantom{Br})-(-[:10]H)(-[:50]Br)-\charge{90:3pt=$\oplus$}{}-=)}
                \arrow{0}[,0.1]\+
                \chemfig{@{Br7}Br-[@{sb7}]\charge{90:3pt=$\ominus$}{Fe}Br_3}
            }
            \arrow(.south--.north){->[*{0}-\ce{FeBr3}]}[-90]
            \subscheme{
                \chemfig{*6(-=(-[:-50,,,,white]\phantom{Br})-(-[@{sb8a}:10]@{H8}H)(-[:50]Br)-[@{sb8b}]\charge{90:3pt=$\oplus$}{}-=)}
                \arrow{0}[,0.1]\+
                \chemfig{@{Br9}\charge{45:2pt=$\ominus$}{Br}}
            }
            \arrow(.west--.east)[180]
            \subscheme{
                \chemfig{*6(-=(-[,,,,white]\phantom{Br})-(-Br)=-=)}
                \arrow{0}[,0.1]\+
                \chemfig{HBr}
            }
        \schemestop
        \chemmove{
            \draw [rex,semithick,shorten <=6pt,shorten >=2pt] (Br2) to[bend left=80,looseness=2.5] (Fe3);
            \draw [rex,semithick,shorten <=2pt,shorten >=2pt] (db4) to[bend left=60,looseness=1.5] (Br5a);
            \draw [rex,semithick,shorten <=2pt,shorten >=2pt] (sb5) to[bend right=70,looseness=2.5] (Br5b);
            \draw [rex,semithick,shorten <=2pt,shorten >=2pt] (sb7) to[bend right=90,looseness=3] (Br7);
            \draw [rex,semithick,shorten <=2pt,shorten >=2pt] (Br9) to[bend right=60,looseness=1.5] (H8);
            \draw [rex,semithick,shorten <=2pt,shorten >=2pt] (sb8a) to[bend left=90,looseness=3] (sb8b);
        }
        \caption{EAS halogenation mechanism.}
        \label{fig:EAShalogenationMechanism}
    \end{figure}
    \item For chlorination, we use catalytic \ce{AlCl3}.
    \item For iodination, we use catalytic \ce{CuI2}.
    \item Nitration.
    \item General form.
    \begin{equation*}
        \ce{PhH + HNO3 ->[\text{cat.} H2SO4] PhNO2 + H2O}
    \end{equation*}
    \begin{itemize}
        \item We start with nitric acid, but as before, the nitrogen is not electrophilic enough.
        \item Thus, we add catalytic sulfuric acid. Since \ce{H2SO4} is stronger than \ce{HNO3}, it protonates nitric acid to \ce{H2NO3+}, which quickly splits into \ce{H2O + NO2+}, where \ce{NO2+} is the \textbf{nitronium ion} (a super electrophile!).
    \end{itemize}
    \item Mechanism.
    \begin{figure}[H]
        \centering
        \begin{subfigure}[b]{\linewidth}
            \centering
            \footnotesize
            \schemestart
                \chemfig{\charge{90=\:,135:2pt=$\ominus$}{O}-[:30]\charge{-90:3pt=$\oplus$}{N}(=[2]O)-[:-30]@{O1}\charge{90=\:}{O}H}
                \arrow{->[\chemfig{@{H2}H-[@{sb2}]@{O2}O_4SH}][-\ce{HSO4-}]}[,1.8]
                \chemfig{@{O3a}\charge{90=\:,135:2pt=$\ominus$}{O}-[@{sb3a}:30]\charge{-90:3pt=$\oplus$}{N}(=[2]O)-[@{sb3b}:-30]@{O3b}\charge{90:3pt=$\oplus$}{O}H_2}
                \arrow{->[][-\ce{H2O}]}[,1.8]
                \chemfig{O=[2]\charge{45:2pt=$\oplus$}{N}=[2]O}
            \schemestop
            \chemmove{
                \draw [rex,semithick,shorten <=6pt,shorten >=2pt] (O1) to[out=90,in=90,out looseness=4,in looseness=2] (H2);
                \draw [rex,semithick,shorten <=2pt,shorten >=2pt] (sb2) to[bend left=80,looseness=3] (O2);
                \draw [rex,semithick,shorten <=6pt,shorten >=2pt] (O3a) to[bend left=70,looseness=2.5] (sb3a);
                \draw [rex,semithick,shorten <=2pt,shorten >=2pt] (sb3b) to[bend right=70,looseness=2.5] (O3b);
            }
            \caption{Nitronium ion formation.}
            \label{fig:EASnitrationMechanisma}
        \end{subfigure}
    \end{figure}
    \begin{figure}[H]
        \ContinuedFloat
        \begin{subfigure}[b]{\linewidth}
            \centering
            \footnotesize
            \schemestart
                \chemfig{*6(-=-=[@{db1}]-=)}
                \arrow{0}[,0.1]\+{,,2.7em}
                \chemfig{@{O2}O=[@{db2}2]@{N2}\charge{45:2pt=$\oplus$}{N}=[2]O}
                \arrow[,1.8]
                \chemfig{*6(-=(-[:-50,,,,white]\phantom{H})-(-[:10]NO_2)(-[@{sb3a}:50]@{H3}H)-[@{sb3b}]\charge{90:3pt=$\oplus$}{}-=)}
                \arrow{->[\chemfig{@{O4}\charge{90=\:,45:2pt=$\ominus$}{O}-SO_3H}][-\ce{H2SO4}]}[,1.8]
                \chemfig{*6(-=(-[,,,,white]\phantom{NO_2})-(-NO_2)=-=)}
            \schemestop
            \chemmove{
                \draw [rex,semithick,shorten <=2pt,shorten >=2pt] (db1) to[out=60,in=150,looseness=1.5] (N2);
                \draw [rex,semithick,shorten <=3pt,shorten >=2pt] (db2) to[bend left=80,looseness=3] (O2);
                \draw [rex,semithick,shorten <=6pt,shorten >=2pt] (O4) to[bend right=60,looseness=1.5] (H3);
                \draw [rex,semithick,shorten <=2pt,shorten >=2pt] (sb3a) to[bend right=70,looseness=2] (sb3b);
            }
            \caption{Nitration of benzene.}
            \label{fig:EASnitrationMechanismb}
        \end{subfigure}
        \caption{EAS nitration mechanism.}
        \label{fig:EASnitrationMechanism}
    \end{figure}
\end{itemize}



\section{Electrophilic Aromatic Substitution 2}
\begin{itemize}
    \item \marginnote{2/17:}Problem set 4 posted today.
    \begin{itemize}
        \item We will have all the material for it by the end of Tuesday.
    \end{itemize}
    \item Today:
    \begin{itemize}
        \item We run through a bunch of use cases of EAS (putting different functional groups on an aromatic ring).
        \item Regioselectivity, reaction rates, etc. at the end of class.
    \end{itemize}
    \item Sulfonation.
    \item General form.
    \begin{equation*}
        \ce{PhH + SO3 ->[\text{cat.} H2SO4] PhSO3H}
    \end{equation*}
    \begin{itemize}
        \item Important for making detergents --- sulfates are highly soluble, so we use them to solvate the constituent lipids.
    \end{itemize}
    \item Mechanism.
    \begin{figure}[h!]
        \centering
        \footnotesize
        \schemestart
            \chemfig{S(=[:-30]O)(=[2]O)(=[:210]O)}
            \arrow{0}[,0]\+
            \chemfig{H_2SO_4}
            \arrow
            \chemfig{@{S3}S(=[:210]O)(=[2]O)=[@{db3}:-30]@{O3}\charge{-90:3pt=$\oplus$}{O}-[:30]H}
            \arrow{0}[,0]\+{1em}
            \chemfig{HS@{O4}\charge{45:1pt=$\ominus$}{O}_4}
            \arrow(@c1--){0}[-100,1.5]
            \chemfig{*6(-=-=[@{db5}]-=)}
            \arrow
            \chemfig{*6(-=(-[0,,,,white]\phantom{S}=[6,,,,white]\phantom{O})-(-[0]S(=[2]O)(-[0]OH)(=[6]O))(-[@{sb6a}:80]@{H6}H)-[@{sb6b}]\charge{90:3pt=$\oplus$}{}-=)}
            \arrow
            \chemfig{[:-30]*6(-=-(-S(=[6]O)(=[2]O)-OH)=-=)}
        \schemestop
        \chemmove{
            \draw [rex,semithick,shorten <=2pt,shorten >=2pt] (db5) to[out=60,in=-90,in looseness=0.8] (S3);
            \draw [rex,semithick,shorten <=3pt,shorten >=2pt] (db3) to[out=60,in=90,looseness=3] (O3);
            \draw [rex,semithick,shorten <=2pt,shorten >=2pt] (O4) to[out=-90,in=60] (H6);
            \draw [rex,semithick,shorten <=2pt,shorten >=2pt] (sb6a) to[bend right=70,looseness=2.5] (sb6b);
        }
        \caption{EAS sulfonation mechanism.}
        \label{fig:EASsulfonationMechanism}
    \end{figure}
    \begin{itemize}
        \item As with nitration (Figure \ref{fig:EASnitrationMechanism}), we use sulfuric acid to protonate a species that will then interact with benzene.
    \end{itemize}
    \item Friedel-Crafts acylation.
    \item General form.
    \begin{equation*}
        \ce{PhH + RCOCl ->[AlCl3] PhCOR + HCl}
    \end{equation*}
    \begin{itemize}
        \item The acid chloride is a very strong electrophile, but it needs to be even stronger. We can make it stronger with the \ce{AlCl3} catalyst.
        \item This reaction is incredibly useful because it forms a new \ce{C-C} bond.
        \item Limitation: You cannot have an EWG on the ring (the ring needs to be nucleophilic).
    \end{itemize}
    \item Mechanism.
    \begin{figure}[h!]
        \centering
        \footnotesize
        \begin{subfigure}[b]{\linewidth}
            \centering
            \schemestart
                \chemfig{R-[:30](=[2]O)-[:-30]@{Cl1}\charge{90=\:}{Cl}}
                \arrow{->[\chemfig{@{Al2}AlCl_3}]}[,1.1]
                \chemfig{R-[:30](=[@{db3}2]@{O3}\charge{[extra sep=1.5pt]45=\:,135=\:}{O})-[@{sb3}:-30]@{Cl3}\charge{-90:3pt=$\oplus$}{Cl}-[:30]\charge{90:3pt=$\ominus$}{Al}Cl_3}
                \arrow
                \chemleft{[}
                    \subscheme{
                        \chemfig{R-~[@{tb4}]@{O4}\charge{90:3pt=$\oplus$}{O}}
                        \arrow{<->}[-90]\arrow{0}[-90,0.1]
                        \chemfig{R-[:30]\charge{90:3pt=$\oplus$}{}=_[:-30]O}
                    }
                \chemright{]}
                \arrow{0}[,0]\+{,,-3em}
                \chemleft{[}
                    \subscheme{
                        \chemfig{@{Cl6}Cl-[@{sb6}]\charge{90:3pt=$\ominus$}{Al}Cl_3}
                        \arrow{<->}[-90]
                        \subscheme{
                            \chemfig{\charge{45:1pt=$\ominus$}{Cl}}
                            \+{1em,,0em}
                            \chemfig{AlCl_3}
                        }
                    }
                \chemright{]}
            \schemestop
            \chemmove{
                \draw [rex,semithick,shorten <=6pt,shorten >=2pt] (Cl1) to[out=90,in=180,looseness=1.5] (Al2);
                \draw [rex,semithick,shorten <=6pt,shorten >=3pt] (O3) to[out=135,in=180,looseness=4] (db3);
                \draw [rex,semithick,shorten <=2pt,shorten >=2pt] (sb3) to[out=60,in=90,looseness=2.5] (Cl3);
                \draw [rex,semithick,shorten <=4pt,shorten >=2pt] (tb4) to[bend right=90,looseness=3] (O4);
                \draw [rex,semithick,shorten <=2pt,shorten >=2pt] (sb6) to[bend left=90,looseness=3] (Cl6);
            }
            \caption{Acylium ion formation.}
            \label{fig:FCacylationMechanisma}
        \end{subfigure}\\[2em]
        \begin{subfigure}[b]{\linewidth}
            \centering
            \schemestart
                \chemfig{*6(-=-=[@{db1}]-=)}
                \+{,,0.8em}
                \chemfig{R-[:30]@{C2}\charge{90:3pt=$\oplus$}{}=_[:-30]O}
                \arrow[,1.1]
                \chemfig{*6(-=-(-[0](=[:60]O)-[:-60]R)(-[@{sb3a}:80]@{H3}H)-[@{sb3b}]\charge{90:3pt=$\oplus$}{}-=)}
                \arrow{->[\chemfig{@{Cl4}\charge{45:2pt=$\ominus$,90=\:}{Cl}}][-\ce{HCl}]}[,1.1]
                \chemfig{*6(-=-(-(=[2]O)-[:-30]R)=-=)}
            \schemestop
            \chemmove{
                \draw [rex,semithick,shorten <=2pt,shorten >=3pt] (db1) to[out=60,in=170] (C2);
                \draw [rex,semithick,shorten <=6pt,shorten >=2pt] (Cl4) to[out=90,in=60,looseness=1.5] (H3);
                \draw [rex,semithick,shorten <=2pt,shorten >=2pt] (sb3a) to[bend right=70,looseness=2.5] (sb3b);
            }
            \caption{Acylation of benzene.}
            \label{fig:FCacylationMechanismb}
        \end{subfigure}
        \caption{Friedel-Crafts acylation mechanism.}
        \label{fig:FCacylationMechanism}
    \end{figure}
    \begin{itemize}
        \item The reaction with the catalyst makes the carbon center in \ce{O=C^+-R} extremely electrophilic.
    \end{itemize}
    \item Friedel-Crafts alkylation.
    \item General form.
    \begin{equation*}
        \ce{PhH + RCl ->[AlCl3] PhR + HCl}
    \end{equation*}
    \begin{itemize}
        \item Again, F-C alkylation is useful because it forms a \ce{C-C} bond.
    \end{itemize}
    \item Mechanism.
    \begin{itemize}
        \item Mostly analogous to Figure \ref{fig:FCacylationMechanism}.
        \item Problems:
        \begin{itemize}
            \item EWGs (you need a nucleophilic aromatic ring as with F-C acylation).
            \item Selectivity (you get a mixture of products due to hydride/methyl shifts since the mechanism proceeds through a carbocation intermediate).
        \end{itemize}
        \item Additional issue: Over-alkylation. The products are more reactive because the electron-donating alkyl groups increase the nucleophilicity of the aromatic ring. Thus, we get \emph{ortho}- and \emph{para}-dialkyl compounds in addition to the monosubstituted products.
        \begin{itemize}
            \item Note that this is not a problem with the other reactions we've learned so far (everything else added EWGs).
        \end{itemize}
        \item Note that since all we need to run the reaction is a carbocation, the other carbocation generation methods we've learned can also lead to F-C alkylation (if an aromatic compound is present in solution).
        \begin{itemize}
            \item For example, mixing 2-methylpropene and acid generates a tertiary CC+ that can react with benzene to yield $t$-butylbenzene.
            \item Intramolecular reactions can also occur this way --- if a CC+ is formed on a substituent in an aromatic molecule, it can react with the aromatic ring in a ring-closing mechanism.
        \end{itemize}
    \end{itemize}
    \item More on the selectivity issue with F-C alkylation.
    \begin{itemize}
        \item For example, reacting benzene with 1-chloropropane under F-C alkylation conditions will yield isopropylbenzene as the major product and propylbenzene as the minor product.
        \item Thus, don't use F-C alkylation for linear $n$-alkyl compounds. It should be reserved for if you want to add a $t$-butyl group or another alkyl group with symmetric hydrogens.
        \item If you \emph{do} want to create propylbenzene, make use of the much more controllable F-C acylation reaction. Indeed, react benzene with propionyl chloride under F-C acylation conditions, and then either hydrogenate (\ce{H2}/\ce{Pd}) or perform a \textbf{Clemmensen reduction}.
    \end{itemize}
    \item \textbf{Clemmensen reduction}: The selective hydrogenation of a ketone using \ce{Zn(Hg) + HCl}\footnote{Note that \ce{Zn(Hg)} is zinc amalgam.}.
    \begin{itemize}
        \item If there is an alkene in the acid chloride added to the benzene that you don't want to hydrogenate, you will have to use the Clemmensen reduction (as it will only hydrogenate the unwanted ketone).
    \end{itemize}
    \item Example: Ring-closing acylation/alkylation reaction.
    \begin{figure}[h!]
        \centering
        \footnotesize
        \schemestart
            \chemfig{*6(-=-=-=)}
            \arrow{0}[,0]\+{1em,,1em}
            \chemfig{Cl-[:30](=[2]O)-[:-30]-[:30]-[:-30](=[:30])-[6]}
            \arrow{->[\ce{AlCl3}][-\ce{HCl}]}[,1.1]
            \chemfig{*6(-=(-[,,,,white]=[6,,,,white]\phantom{O})-(-(=[2]O)-[:-30]--[:-30](=)-[6])=-=)}
            \arrow{->[\ce{Zn(Hg)}][\ce{HCl}]}[,1.2]
            \chemfig{*6(-=-(--[:-30]--[:-30](=)-[6])=-=)}
            \arrow{->[*{0}\ce{H2SO4}]}[-90]
            \subscheme{
                \chemfig{*6(-=(-H)-(--[:-30]--[:-30]\charge{90:3pt=$\oplus$}{}(--[:-30]H)-[6])=-=)}
                \arrow{0}[,0]\+{1em}
                \chemfig{HS\charge{45:1pt=$\ominus$}{O}_4}
            }
            \arrow{->[][-\ce{H2SO4}]}[180,1.3]
            \chemfig{*6(-=*6(-(-[:-70])(-[:-110])----)-=(-[:70,,,,white])-=)}
        \schemestop
        \caption{Ring-closing Friedel-Crafts mechanism.}
        \label{fig:ringClosingFC}
    \end{figure}
    \begin{itemize}
        \item The acylation product is formed. It can be hydrogenated with the Clemmensen reduction. Then we can turn the alkene into a carbocation with sulfuric acid and subject it to F-C alkylation conditions to yield a ring-closing reaction.
    \end{itemize}
    \item Forming a benzoic acid.
    \begin{equation*}
        \ce{PhR ->[KMnO4][H2O] PhCOOH}
    \end{equation*}
    \begin{itemize}
        \item It is necessary to have a benzylic hydrogen for the mechanism to proceed (for example, \ce{PhBu^{$t$}} wouldn't react).
        \item It is possible to convert multiple alkyl groups at the same time (for example, \ce{C6H4MePr -> C6H4(COOH)2} where the carboxylic acids wind up where the methyl and propyl groups originally were). 
    \end{itemize}
    \item Forming an amine (from a nitro group).
    \begin{equation*}
        \ce{PhNO2 ->[reagents] PhNH2}
    \end{equation*}
    \begin{itemize}
        \item A number of reductive reagents can work here. Explicitly, we may use \ce{H2 + Pd/C}, \ce{H+} (acid), or \ce{SnCl2 + H2O}.
        \item This reaction takes a strong EWG and turns it into an EDG.
        \item The amine is also a gateway to a number of other functional groups, so being able to get one is very helpful.
    \end{itemize}
    \item Diazotization.
    \item General form.
    \begin{equation*}
        \ce{PhNH2 ->[reagents] PhN2+ + X-}
    \end{equation*}
    \begin{itemize}
        \item We use either \ce{NaNO2}/\ce{HCl} or \ce{HNO2} as the reagent(s).
        \item Note that the product is a diazonium salt.
    \end{itemize}
    \item Mechanism.
    \begin{figure}[h!]
        \centering
        \footnotesize
        \begin{subfigure}[b]{\linewidth}
            \centering
            \schemestart
                \chemfig{\charge{45:1pt=$\oplus$}{Na}-[,0.8,,,white]@{O1}\charge{90=\:,135:1pt=$\ominus$}{O}-[:30]N=[:-30]O}
                \arrow{0}[,0]\+
                \chemfig{@{H2}H-[@{sb2}]@{Cl2}Cl}
                \arrow{->[][-\ce{NaCl}]}[,1.1]
                \chemfig{H-[:30]@{O3}\charge{90=\:}{O}-[:-30]N=[:30]O}
                \arrow{->[\chemfig{@{H4}\charge{45:1pt=$\oplus$}{H}}]}
                \chemfig{@{O5b}\charge{45:1pt=$\oplus$}{O}(-[:120]H)(-[:-120]H)-[@{sb5}]\charge{[extra sep=1.5pt]-45=\:}{N}=[@{db5}:60]@{O5a}\charge{0=\:,90=\:}{O}}
                \arrow{->[][-\ce{H2O}]}
                \chemfig{N~\charge{45:1pt=$\oplus$}{O}}
            \schemestop
            \chemmove{
                \draw [rex,semithick,shorten <=6pt,shorten >=2pt] (O1) to[bend left=90,looseness=1.5] (H2);
                \draw [rex,semithick,shorten <=2pt,shorten >=2pt] (sb2) to[bend left=90,looseness=3] (Cl2);
                \draw [rex,semithick,shorten <=6pt,shorten >=2pt] (O3) to[bend left=90,looseness=1.5] (H4);
                \draw [rex,semithick,shorten <=6pt,shorten >=3pt] (O5a) to[out=90,in=150,out looseness=5,in looseness=3] (db5);
                \draw [rex,semithick,shorten <=2pt,shorten >=2pt] (sb5) to[bend left=90,looseness=3] (O5b);
            }
            \caption{Nitrosonium ion formation.}
            \label{fig:EASdiazotizationMechanisma}
        \end{subfigure}\\[2em]
        \begin{subfigure}[b]{\linewidth}
            \centering
            \schemestart
                \chemfig{*6(-=(-[,,,,white]\phantom{N})-(-@{N1}\charge{90=\:}{N}H_2)=-=)}
                \arrow{->[\chemfig{@{N2}N~[@{tb2}]@{O2}\charge{45:1pt=$\oplus$}{O}}]}[,1.3]
                \chemfig{*6(-=(-[,,,,white]\phantom{N}-[6,,,,white]\phantom{N}=[,,,,white]\phantom{O})-(-\charge{-90:1pt=$\oplus$}{N}H_2-[2]N=[:30]O)=-=)}
                \arrow(--.-150){<=>[-\ce{H+}]}[,1.2]
                \chemfig{*6(-=-(-@{N4}\charge{[extra sep=1.5pt]180=\:}{N}(-[@{sb4a}:-30]@{H4}H)-[@{sb4b}2]N=[@{db4}:30]@{O4}\charge{[extra sep=1.5pt]45=\:,-45=\:}{O})=-=)}
                \arrow{<=>}[-90]
                \chemfig{*6(-=(-[,,,,white]\phantom{N}-[6,,,,white]\phantom{N}=[,,,,white]\phantom{O})-(-N=[2]N-[:30]OH)=-=)}
                \arrow{<=>[][*{0.90}\ce{H+}]}[180]
                \chemfig{*6(-=(-[,,,,white]\phantom{N}-[6,,,,white]\phantom{N}=[,,,,white]\phantom{O})-(-@{N6}\charge{0=\:}{N}=[@{db6}2]N-[@{sb6}:30]@{O6}\charge{90:3pt=$\oplus$}{O}H_2)=-=)}
                \arrow{->[][*{0.90}\ce{-H2O}]}[180]
                \chemfig{*6(-=(-[,,,,white]\phantom{N}~[,,,,white]\phantom{N})-(-\charge{135:1pt=$\oplus$}{N}~N)=-=)}
            \schemestop
            \chemmove{
                \draw [rex,semithick,shorten <=6pt,shorten >=2pt] (N1) to[out=90,in=90,looseness=2] (N2);
                \draw [rex,semithick,shorten <=4pt,shorten >=2pt] (tb2) to[bend left=90,looseness=3] (O2);
                \draw [rex,semithick,shorten <=6pt,shorten >=2pt] (O4) to[bend left=45] (H4);
                \draw [rex,semithick,shorten <=2pt,shorten >=2pt] (sb4a) to[bend left=90,looseness=3] (N4);
                \draw [rex,semithick,shorten <=6pt,shorten >=2pt] (N4) to[bend left=90,looseness=3] (sb4b);
                \draw [rex,semithick,shorten <=3pt,shorten >=2pt] (db4) to[bend left=90,looseness=3] (O4);
                \draw [rex,semithick,shorten <=6pt,shorten >=3pt] (N6) to[bend right=90,looseness=3] (db6);
                \draw [rex,semithick,shorten <=2pt,shorten >=2pt] (sb6) to[bend left=90,looseness=3] (O6);
            }
            \vspace{-3em}
            \caption{Diazotization of aniline.}
            \label{fig:EASdiazotizationMechanismb}
        \end{subfigure}
        \caption{EAS diazotization mechanism.}
        \label{fig:EASdiazotizationMechanism}
    \end{figure}
    \begin{itemize}
        \item Note that the \ce{HONO} intermediate in Figure \ref{fig:EASdiazotizationMechanisma} is called "HONO"\footnote{"HOE-NOE"}.
        \item Excess strong acid is used here (we need two equivalents of acid at least to form the nitrosonium ion).
        \item The nitrosonium ion is very unstable and reacts quickly with the relatively nucleophilic amine.
        \item The last step being irreversible serves as the driving force for this reaction.
    \end{itemize}
    \item \textbf{Sandmeyer reaction}: Reacting an aryl diazonium salt with electrophiles in the presence of a copper catalyst substitutes those electrophiles for the diazonium group.
    \item In particular, mixing \ce{PhN2+} with\dots
    \begin{itemize}
        \item \ce{Cu2O, H2O} makes \ce{PhOH}.
        \item \ce{CuCl} makes \ce{PhCl}.
        \item \ce{CuBr} makes \ce{PhBr}.
        \item \ce{CuI} makes \ce{PhI}.
    \end{itemize}
    \item Mechanism.
    \begin{itemize}
        \item Complex; sort-of S\textsubscript{N}1-like.
        \item The diazonium is a great leaving group, so it leaves, making a phenyl cation and \ce{N2}. At this point, a nucleophile can just swoop in and attack the phenyl cation.
        \item Note that the phenyl cation intermediate is still aromatic --- the electron removed was taken from an $sp^2$ orbital, not a $p$ orbital.
    \end{itemize}
    \item We can hydrogenate our diazonium phenyl compound with \ce{H3PO2}.
    \begin{equation*}
        \ce{PhN2+ ->[H3PO2] PhH}
    \end{equation*}
    \item Why F-C alkylations lead to over-alkylation but others do not.
    \begin{itemize}
        \item Consider nitration.
        \item Nitro groups are strongly electron withdrawing. Thus, they deactivate their host aromatic ring.
        \begin{itemize}
            \item Over a long time, however, we will see the formation of some \emph{meta}-dinitrobenzene (not \emph{ortho} or \emph{para}).
            \item This is because resonance gives us carbocations at the \emph{ortho} and \emph{para} positions. Thus, the molecule is \num{100000} times less reactive then benzene toward EAS overall, but the molecule is even less reactive (less nucleophilic) at the \emph{ortho} and \emph{para} positions.
            \item Additionally, if we substitute \emph{ortho} or \emph{para}, we will have a resonance structure in the transition state with a carbocation directly adjacent to the positive nitrogen of the nitro group. This is no good, and another reason why EWGs are \emph{meta}-directing.
        \end{itemize}
        \item Same for acyl groups and \ce{SO3H} groups.
    \end{itemize}
    \item With respect to \emph{ortho}/\emph{para} selectivity, sterics \emph{may} sometimes make \emph{ortho}-substitution less likely.
    \item Activators and deactivators.
    \begin{figure}[h!]
        \centering
        \footnotesize
        \begin{tikzpicture}
            \draw [stealth-stealth] (-9.5,0) node[above right,yshift=1mm]{deactivators} -- (6.5,0) node[above left,yshift=1mm]{activators};
    
            \draw
                (-9,0.1) -- ++(0,-0.2) node[below]{\ce{{}^+NR3}}
                (-8,0.1) -- ++(0,-0.2) node[below]{\ce{NO2}}
                (-7,0.1) -- ++(0,-0.2) node[below]{\ce{CN}}
                (-6,0.1) -- ++(0,-0.2) node[below]{\ce{SO3H}}
                (-5,0.1) -- ++(0,-0.2) node[below]{\chemfig[atom sep=1.4em]{O=(-[:60])(-[:-60]R)}}
                (-4,0.1) -- ++(0,-0.2) node[below]{\ce{CO2H}}
                (-3,0.1) -- ++(0,-0.2) node[below]{\ce{CHO}}
                (-2.25,0.1) -- ++(0,-0.2) node[below]{\ce{I}}
                (-1.75,0.1) -- ++(0,-0.2) node[below]{\ce{Br}}
                (-1.25,0.1) -- ++(0,-0.2) node[below]{\ce{Cl}}
                (-0.75,0.1) -- ++(0,-0.2) node[below]{\ce{F}}
                (0,0.2) -- ++(0,-0.4) node[below]{\chemfig[atom sep=1.4em]{*6(-=-=-=)}}
                (1,0.1) -- ++(0,-0.2) node[below]{aryl}
                (2,0.1) -- ++(0,-0.2) node[below]{alkyl}
                (3,0.1) -- ++(0,-0.2) node[below]{\chemfig[atom sep=1.4em]{HN-[:-60](=[::-60]O)-[::60]R}}
                (4,0.1) -- ++(0,-0.2) node[below]{\ce{OR}}
                (5,0.1) -- ++(0,-0.2) node[below]{\ce{OH}}
                (6,0.1) -- ++(0,-0.2) node[below]{\ce{NH2}}
            ;
    
            \draw [decorate,decoration={brace,mirror}] (-9.4,-1.5) -- node[below=1mm]{\emph{m}-directing} (-2.6,-1.5);
            \draw [decorate,decoration={brace,mirror}] (-2.4,-1.5) -- node[below=1mm]{\emph{o}/\emph{p}-directing} (-0.6,-1.5);
            \draw [decorate,decoration={brace,mirror}] (0.6,-1.5) -- node[below=1mm]{\emph{o}/\emph{p}-directing} (6.4,-1.5);
        \end{tikzpicture}
        \caption{Activators and deactivators.}
        \label{fig:DeActivators}
    \end{figure}
    \begin{itemize}
        \item Deactivators are \emph{meta}-directors (the \emph{meta} position is the most nucleophilic position in a deactivator-substituted molecule).
        \item Activators won't break aromaticity as in resonance structures, but said structures can indicate trends. They are \emph{ortho}/\emph{para} directors.
        \begin{itemize}
            \item Convince yourself using resonance structures that \emph{ortho}/\emph{para} addition leads to 1 extra resonance structure.
        \end{itemize}
        \item Alkyl and aryl groups are \emph{ortho}/\emph{para}-directing activators.
        \begin{itemize}
            \item \emph{Ortho}/\emph{para} addition allows us to access a resonance structure where the carbocation intermediate is tertiary.
            \item For example, toluene is about 25 times more reactive than benzene, and it is even more reactive at the \emph{ortho}/\emph{para} positions.
        \end{itemize}
        \item Halogens will be discussed next week.
    \end{itemize}
\end{itemize}



\section{Chapter 15: Reactions of Aromatic Compounds}
\emph{From \textcite{bib:SolomonsEtAl}.}
\begin{itemize}
    \item \marginnote{2/23:}"Kekul\'{e} structures are more appropriate for writing mechanisms such as electrophilic aromatic substitution because they permit the use of resonance theory, which, as we shall soon see, is invaluable as an aid to our understanding" \parencite[663]{bib:SolomonsEtAl}.
    \item Halogenation.
    \item Bromination and chlorination of benzene are analogous (see Figure \ref{fig:EAShalogenationMechanism}).
    \begin{itemize}
        \item Flourination of benzene occurs so rapidly that it is hard to limit it to monofluorination. There are indirect methods of producing fluorobenzene, though.
        \item Iodination of benzene requires incredibly forcing conditions, and is thus carried out in the presence of a strong oxidizing agent such as nitric acid. Biochemically, iodination of benzene is enzymatically catalyzed.
    \end{itemize}
    \item Sulfonation.
    \item \textbf{Fuming sulfuric acid}: Sulfuric acid that contains added sulfur trioxide.
    \item It is the reaction of benzene with fuming sulfuric acid that produces benzenesulfonic acid.
    \begin{itemize}
        \item Note that since \ce{H2SO4} partially decomposes into \ce{SO3} over time, the reaction will take place in the presence of concentrated sulfuric acid alone, but much more slowly.
        \item The presence of \ce{SO3} is essential to the mechanism.
    \end{itemize}
    \item Sulfonation is reversible; the equilibrium depends on the conditions.
    \begin{itemize}
        \item Fuming sulfuric acid promotes sulfonation.
        \item Dilute sulfuric acid with steam bubbled through (high $[\ce{H2O}]$) promotes desulfonation.
    \end{itemize}
    \item \textbf{Protecting group}: A functional group used to temporarily block a position from electrophilic aromatic substitution.
    \item Since sulfonation is reversible, sulfonate is often used as a protecting group.
    \item Friedel-Crafts alkylation.
    \item For primary halides, an aluminum chloride-alkyl halide complex forms instead of a carbocation. In this case, the \ce{C-Cl} bond is all but broken, making the carbon atom positive to the point that it acts as if it were a carbocation, but there is still some connecting electron density.
    \item Any process that creates a carbocation may be used as a precursor to F-C alkylation.
    \begin{itemize}
        \item A mixture of an alcohol and an acid (e.g., \ce{BF3}) can also be used.
    \end{itemize}
    \item Friedel-Crafts acylation.
    \item \textbf{Acetyl group}: The \ce{MeCO} group, bound through the \ce{C}. \emph{Also known as} \textbf{ethanoyl group}.
    \item \textbf{Benzoyl group}: The \ce{PhCO} group, bound through the \ce{C}.
    \item \textbf{Acyl halide}. \emph{Also known as} \textbf{acid halide}.
    \item Friedel-Crafts acylations can also be carried out using carboxylic acid anhydrides.
    \item Additional limitation: "Aryl and vinylic halides cannot be used as the halide component because they do not form carbocations readily" \parencite[672]{bib:SolomonsEtAl}.
    \item Side reaction with the Clemmensen reduction: Zinc and \ce{HCl} can also reduce nitro groups to amino groups.
    \item \textbf{Wolff-Kishner reduction}: The basic complement of the Clemmensen reduction, which hydrogenates a ketone with hydrazine (\ce{H2NNH2}), \ce{KOH}, and heat.
    \begin{itemize}
        \item More on this next quarter.
    \end{itemize}
    \item \textbf{Arene}: A hydrocarbon that consists of both aliphatic and aromatic groups.
    \item Covers side-chain reactions, stability, and transmutations.
\end{itemize}




\end{document}