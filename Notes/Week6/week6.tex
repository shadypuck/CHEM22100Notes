\documentclass[../notes.tex]{subfiles}

\pagestyle{main}
\renewcommand{\chaptermark}[1]{\markboth{\chaptername\ \thechapter\ (#1)}{}}
\setcounter{chapter}{5}

\begin{document}




\chapter{Electrophilic Aromatic Substitution}
\section{Electrophilic Aromatic Substitution 1}
\begin{itemize}
    \item \marginnote{2/15:}Discusses the aromaticity of fluorescein as an example to review from last class.
    \item Reactions of aromatic compounds are divided into two classes: Electrophilic and nucleophilic aromatic substitutions.
    \item Example:
    \begin{itemize}
        \item \ce{C6H6 ->[H3O+] C6H6} means no reaction?
        \item \ce{C6H6 ->[D3O+] C6D6}; thus, a substitution is occurring.
    \end{itemize}
    \item Mechanism:
    \begin{figure}[h!]
        \centering
        \footnotesize
        \schemestart
            \chemfig{*6(-=-=[@{db1}]-=)}
            \arrow{->[\chemfig{@{D2}\charge{45=$\oplus$}{D}}]}[,1.3]
            \chemfig{*6(-=-(-[@{sb3a}:10]@{H3}H)(-[:50]D)-[@{sb3b}]\charge{90:3pt=$\oplus$}{}-=)}
            \arrow{->[\chemfig{@{O4}\charge{90=\:}{O}D_2}][-\ce{HD2O+}]}[,1.3]
            \chemfig{*6(-=-(-D)=-=)}
        \schemestop
        \chemmove{
            \draw [rex,semithick,shorten <=2pt,shorten >=2pt] (db1) to[bend left=60,looseness=1.5] (D2);
            \draw [rex,semithick,shorten <=6pt,shorten >=2pt] (O4) to[out=90,in=60,looseness=2] (H3);
            \draw [rex,semithick,shorten <=2pt,shorten >=2pt] (sb3a) to[bend left=90,looseness=3] (sb3b);
        }
        \caption{Electrophilic aromatic substitution mechanism.}
        \label{fig:EASmechanism}
    \end{figure}
    \begin{itemize}
        \item To begin, one of the $\pi$-bonds of benzene attacks \ce{D+}. This causes the loss of aromaticity, but the carbocation is highly resonance delocalized.
        \item Although we \emph{could} make an alcohol at this point, this would lead to the loss of aromaticity in the product, so we won't do that.
        \item Instead, we do an E\textsubscript{1}-type reaction.
        \item The first step is the RDS.
        \item The intermediate in this mechanism is called the \textbf{arenium ion}, the \textbf{Wheland intermediate}, or the \textbf{sigma complex}.
        \item Note that the electrophile used in this reaction has to be a very special, very reactive, very strong electrophile in order to make up the energy gap.
    \end{itemize}
    \item We know that the sigma complex exists because we can trap the intermediate.
    \item Whether or not we see the product react again depends on whether the product or starting material is more nucleophilic.
    \item Adding an EDG to the benzene makes the reaction proceed faster.
    \begin{itemize}
        \item A good EDG will stabilize the arenium ion, lowering the activation barrier of the first step (the RDS).
    \end{itemize}
    \item Halogenation.
    \item General form.
    \begin{equation*}
        \ce{PhH + Br2 ->[\text{cat.} FeBr3] PhBr + HBr}
    \end{equation*}
    \begin{itemize}
        \item \ce{Br2} is too unreactive to have chemistry with benzene on its own.
        \item In particular, when we say that \ce{Br2} is too unreactive, we mean that there is not enough \ce{Br+} character, i.e., it is not a good enough electrophile.
        \item To overcome the problem, we add \ce{Br2} to \ce{FeBr3}, a good Lewis acid with an open valence site. It follows that \ce{Br-Br^+-Fe^-Br3} is a super awesome electrophile!
    \end{itemize}
    \item Mechanism.
    \begin{figure}[h!]
        \centering
        \footnotesize
        \schemestart
            \chemfig{*6(-=-=-=)}
            \arrow{0}[,0.1]\+
            \chemfig{Br-@{Br2}\charge{90=\:}{Br}}
            \arrow{->[\chemfig{@{Fe3}FeBr_3}]}[,1.1]
            \chemfig{*6(-=-=[@{db4}]-=)}
            \arrow{0}[,0.1]\+
            \chemfig{@{Br5a}Br-[@{sb5}]@{Br5b}\charge{90:3pt=$\oplus$}{Br}-\charge{90:3pt=$\ominus$}{Fe}Br_3}
            \arrow
            \subscheme{
                \chemfig{*6(-=(-[:-50,,,,white]\phantom{Br})-(-[:10]H)(-[:50]Br)-\charge{90:3pt=$\oplus$}{}-=)}
                \arrow{0}[,0.1]\+
                \chemfig{@{Br7}Br-[@{sb7}]\charge{90:3pt=$\ominus$}{Fe}Br_3}
            }
            \arrow(.south--.north){->[*{0}-\ce{FeBr3}]}[-90]
            \subscheme{
                \chemfig{*6(-=(-[:-50,,,,white]\phantom{Br})-(-[@{sb8a}:10]@{H8}H)(-[:50]Br)-[@{sb8b}]\charge{90:3pt=$\oplus$}{}-=)}
                \arrow{0}[,0.1]\+
                \chemfig{@{Br9}\charge{45:2pt=$\ominus$}{Br}}
            }
            \arrow(.west--.east)[180]
            \subscheme{
                \chemfig{*6(-=(-[,,,,white]\phantom{Br})-(-Br)=-=)}
                \arrow{0}[,0.1]\+
                \chemfig{HBr}
            }
        \schemestop
        \chemmove{
            \draw [rex,semithick,shorten <=6pt,shorten >=2pt] (Br2) to[bend left=80,looseness=2.5] (Fe3);
            \draw [rex,semithick,shorten <=2pt,shorten >=2pt] (db4) to[bend left=60,looseness=1.5] (Br5a);
            \draw [rex,semithick,shorten <=2pt,shorten >=2pt] (sb5) to[bend right=70,looseness=2.5] (Br5b);
            \draw [rex,semithick,shorten <=2pt,shorten >=2pt] (sb7) to[bend right=90,looseness=3] (Br7);
            \draw [rex,semithick,shorten <=2pt,shorten >=2pt] (Br9) to[bend right=60,looseness=1.5] (H8);
            \draw [rex,semithick,shorten <=2pt,shorten >=2pt] (sb8a) to[bend left=90,looseness=3] (sb8b);
        }
        \caption{EAS halogenation mechanism.}
        \label{fig:EAShalogenationMechanism}
    \end{figure}
    \item For chlorination, we use catalytic \ce{AlCl3}.
    \item For iodination, we use catalytic \ce{CuI2}.
    \item Nitration.
    \item General form.
    \begin{equation*}
        \ce{PhH + HNO3 ->[\text{cat.} H2SO4] PhNO2 + H2O}
    \end{equation*}
    \begin{itemize}
        \item We start with nitric acid, but as before, the nitrogen is not electrophilic enough.
        \item Thus, we add catalytic sulfuric acid. Since \ce{H2SO4} is stronger than \ce{HNO3}, it protonates nitric acid to \ce{H2NO3+}, which quickly splits into \ce{H2O + NO2+}, where \ce{NO2+} is the nitronium ion (a super electrophile!).
    \end{itemize}
    \item Mechanism.
    \begin{figure}[H]
        \centering
        \begin{subfigure}[b]{\linewidth}
            \centering
            \footnotesize
            \schemestart
                \chemfig{\charge{90=\:,135:2pt=$\ominus$}{O}-[:30]\charge{-90:3pt=$\oplus$}{N}(=[2]O)-[:-30]@{O1}\charge{90=\:}{O}H}
                \arrow{->[\chemfig{@{H2}H-[@{sb2}]@{O2}OSO_3H}][-\ce{HSO4-}]}[,1.8]
                \chemfig{@{O3a}\charge{90=\:,135:2pt=$\ominus$}{O}-[@{sb3a}:30]\charge{-90:3pt=$\oplus$}{N}(=[2]O)-[@{sb3b}:-30]@{O3b}\charge{90:3pt=$\oplus$}{O}H_2}
                \arrow{->[][-\ce{H2O}]}[,1.8]
                \chemfig{O=[2]\charge{45:2pt=$\oplus$}{N}=[2]O}
            \schemestop
            \chemmove{
                \draw [rex,semithick,shorten <=6pt,shorten >=2pt] (O1) to[out=90,in=90,out looseness=4,in looseness=2] (H2);
                \draw [rex,semithick,shorten <=2pt,shorten >=2pt] (sb2) to[bend left=80,looseness=3] (O2);
                \draw [rex,semithick,shorten <=6pt,shorten >=2pt] (O3a) to[bend left=70,looseness=2.5] (sb3a);
                \draw [rex,semithick,shorten <=2pt,shorten >=2pt] (sb3b) to[bend right=70,looseness=2.5] (O3b);
            }
            \caption{Nitronium ion formation.}
            \label{fig:EASnitrationMechanisma}
        \end{subfigure}
    \end{figure}
    \begin{figure}[H]
        \ContinuedFloat
        \begin{subfigure}[b]{\linewidth}
            \centering
            \footnotesize
            \schemestart
                \chemfig{*6(-=-=[@{db1}]-=)}
                \arrow{0}[,0.1]\+{,,2.7em}
                \chemfig{@{O2}O=[@{db2}2]@{N2}\charge{45:2pt=$\oplus$}{N}=[2]O}
                \arrow[,1.8]
                \chemfig{*6(-=(-[:-50,,,,white]\phantom{H})-(-[:10]NO_2)(-[@{sb3a}:50]@{H3}H)-[@{sb3b}]\charge{90:3pt=$\oplus$}{}-=)}
                \arrow{->[\chemfig{@{O4}\charge{90=\:,45:2pt=$\ominus$}{O}-SO_3H}][-\ce{H2SO4}]}[,1.8]
                \chemfig{*6(-=(-[,,,,white]\phantom{NO_2})-(-NO_2)=-=)}
            \schemestop
            \chemmove{
                \draw [rex,semithick,shorten <=2pt,shorten >=2pt] (db1) to[out=60,in=150,looseness=1.5] (N2);
                \draw [rex,semithick,shorten <=3pt,shorten >=2pt] (db2) to[bend left=80,looseness=3] (O2);
                \draw [rex,semithick,shorten <=6pt,shorten >=2pt] (O4) to[bend right=60,looseness=1.5] (H3);
                \draw [rex,semithick,shorten <=2pt,shorten >=2pt] (sb3a) to[bend right=70,looseness=2] (sb3b);
            }
            \caption{Nitration of benzene.}
            \label{fig:EASnitrationMechanismb}
        \end{subfigure}
        \caption{EAS nitration mechanism.}
        \label{fig:EASnitrationMechanism}
    \end{figure}
\end{itemize}




\end{document}