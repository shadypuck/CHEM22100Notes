\documentclass[../notes.tex]{subfiles}

\pagestyle{main}
\renewcommand{\chaptermark}[1]{\markboth{\chaptername\ \thechapter\ (#1)}{}}
\setcounter{chapter}{6}

\begin{document}




\chapter{Nucleophilic Aromatic Substitution}
\section{Office Hours (Salinas)}
\begin{itemize}
    \item \marginnote{2/21:}Setup drawings points for Eucalyptus Oil and Bromination of Vanillin notebook pages? No new glassware setups so no drawings needed, right?
    \begin{itemize}
        \item Nope, not needed --- we can just refer back.
    \end{itemize}
    \item Go over Eucalyptus Oil reagent table column entries; make sure I have everything I need.
    \begin{itemize}
        \item Any hydrocarbon component.
    \end{itemize}
    \item Crude yield and percent recovery calculations?
    \begin{itemize}
        \item I did them correctly.
    \end{itemize}
    \item Do we not need an entry for one of the substances this week? Perhaps \ce{EtOH} in \ce{H2O}? Since there are only 15 points on Canvas, that leads me to think you're only looking for 5 entries.
    \begin{itemize}
        \item Do all six reagents.
        \item Just treat $50\%$ \ce{EtOH} in \ce{H2O} as \ce{EtOH} (though there might be slight safety differences; check to make sure there isn't a separate MSDS for \ce{EtOH} in \ce{H2O}).
    \end{itemize}
\end{itemize}



\section{Nucleophilic Aromatic Substitution}
\begin{itemize}
    \item \marginnote{2/22:}Office hours Wednesday/Thursday at 4:00 PM.
    \item 1 page notes sheet for the exam next week.
    \item Alkyl groups activate aromatic rings via induction.
    \item Halogens.
    \begin{itemize}
        \item Electron-withdrawing due to induction.
        \item Unlike alkyl groups, however, they have lone pairs that can contribute to resonance once the electrophile is added (i.e., in the carbocation intermediate).
        \item Halogens with greater electronegativity are more strongly electron-withdrawing and thus more deactivating.
    \end{itemize}
    \item Two major problems:
    \begin{enumerate}
        \item Predicting the products based off of the substituents present on a ring.
        \item Synthesizing a ring with multiple substituents on it (the order you add them matters!).
    \end{enumerate}
    \item Practice problem takeaways.
    \begin{itemize}
        \item When you have an ortho/para-directing substituent, you don't have to indicate major/minor products.
        \item However, when doing a synthesis, try and make the reaction more selective by precluding one of the sites with another functional group. You could synthesize ortho/para products and then purify (throw away half of your yield), and while this is an acceptable answer, it is not the best answer when there are other options.
        \item $t$-butyl groups generate significant steric hindrance, so groups will avoid adding ortho to them (even though $t$-butyl is an ortho/para-director by induction).
        \item We will not see trick questions where something is so deactivating that we don't have a reaction; in these cases in reality, raising the temperature would suffice to force the reaction.
        \item Resonance donation outcompetes induction donation.
    \end{itemize}
    \item Rules.
    \begin{enumerate}
        \item If all substituents direct to the same place, EAS happens there.
        \item If not, the strongest \emph{activator} wins.
        \begin{itemize}
            \item This is because deactivators slow everything down (but just the meta site less) whereas activators specifically accelerate particular sites.
        \end{itemize}
        \item If one site is significantly more crowded than a second (out of two choices), sterics can play a role.
        \begin{itemize}
            \item You do need a really big $t$-butyl group (or something larger) though to see this effect.
        \end{itemize}
    \end{enumerate}
    \item Synthesis practice problem.
    \begin{itemize}
        \item Benzene to 3-bromoaniline.
        \item Preferentially use \ce{H2 + Pd/C} to form an amine from a nitro group. The others are less common.
    \end{itemize}
    \item \textbf{Nucleophilic aromatic substitution}. \emph{Also known as} \textbf{NAS}, \textbf{S\textsubscript{N}Ar}.
    \item Reacting various aromatic compounds with methoxide and methanol.
    \begin{itemize}
        \item Chlorobenzene: No reaction.
        \item para-chloronitrobenzene: The methoxide substitutes the chloride (at high temperatures).
        \item 1-chloro-2,4-dinitrobenzene: The methoxide substitutes the chloride (much faster at lower temperatures).
        \item meta-chloronitrobenzene: No reaction.
    \end{itemize}
    \item Mechanism.
    \begin{figure}[h!]
        \centering
        \footnotesize
        \schemestart
            \chemfig{*6(-(-NO_2)=-@{C1b}=[@{db1}]@{C1a}(-Cl)-=)}
            \arrow{->[\chemfig{@{Nu2}\charge{45:1pt=$\ominus$,90=\:}{Nu}}]}
            \chemleft{[}
                \subscheme{
                    \chemfig{*6(-(-\charge{45:1pt=$\oplus$}{N}(=[:-150]O)(-[:-30]\charge{45:1pt=$\ominus$}{O}))=-@{C3}\charge{45:1pt=$\ominus$}{}-[@{sb3a}](-[:70]Nu)(-[@{sb3b}:110]@{Cl3}Cl)-=)}
                    \arrow{<->}
                    \chemname{\chemfig{*6(-(=\charge{45:1pt=$\oplus$}{N}(-[:-150]\charge{135:1pt=$\ominus$}{O})(-[:-30]\charge{45:1pt=$\ominus$}{O}))-=-(-[,2.5,,,white])(-[:70]Nu)(-[:110]Cl)-=)}}{Meisenheimer\\complex}
                }
            \chemright{]}
            \arrow{->[][-\ce{Cl-}]}
            \chemfig{*6(-(-NO_2)=-=(-Nu)-=)}
        \schemestop
        \chemmove{
            \draw [rex,semithick,shorten <=6pt,shorten >=2pt] (Nu2) to[out=90,in=45,looseness=1.5] (C1a);
            \draw [rex,semithick,shorten <=3pt,shorten >=3pt] (db1) to[bend right=90,looseness=4] (C1b);
            \draw [rex,semithick,shorten <=10pt,shorten >=2pt] (C3) to[bend right=90,looseness=5] (sb3a);
            \draw [rex,semithick,shorten <=2pt,shorten >=2pt] (sb3b) to[bend left=90,looseness=3] (Cl3);
        }
        \caption{Nucleophilic aromatic substitution mechanism.}
        \label{fig:NASmechanism}
    \end{figure}
    \begin{itemize}
        \item The driving force for the reaction is having the better leaving group leave.
        \begin{itemize}
            \item Between methoxide and chloride, for example, chloride is the better LG.
        \end{itemize}
        \item To form the Meisenheimer complex, you need a strongly electron-withdrawing group (such as a nitro group) or an intramolecular kinetic driving force.
        \begin{itemize}
            \item You also need the EWG in the right position to be able to accept electron density through resonance.
        \end{itemize}
    \end{itemize}
    \item \textbf{Meisenheimer complex}: The intermediate with a double-bonded nitrogen in a nitrobenzene derivative undergoing S\textsubscript{N}Ar.
    \begin{itemize}
        \item Can be isolated at very low temperatures.
    \end{itemize}
    \item 1,2-dichloro-4-nitrobenzene becomes 2-chloro-1-methoxy-4-nitrobenzene due to the para-activation of the nitro group.
    \item Reduction of aromatic compounds.
    \begin{itemize}
        \item Useful when you want to create a cyclohexane derivative --- you can put on functional groups with EAS and NAS, and then reduce at the end.
    \end{itemize}
    \item High pressure catalyzed.
    \item General form.
    \begin{center}
        \footnotesize
        \setchemfig{atom sep=1.4em}
        \schemestart
            \chemfig{*6(-=(-CH_3)-=(-CH_3)-=)}
            \arrow{0}[,0]\+
            3 \chemfig{H_2}
            \arrow{->[\ce{Pt, Pd, Ru}][$>\num{1000}\,\text{psi}$]}[,1.5]
            \chemfig{*6(--(-CH_3)--(-CH_3)--)}
        \schemestop
    \end{center}
    \begin{itemize}
        \item Only one of the listed transition metals is needed.
        \item This is not practical because you don't want such high pressure bombs in the lab.
    \end{itemize}
    \item Birch reduction.
    \item General form.
    \begin{center}
        \footnotesize
        \setchemfig{atom sep=1.4em}
        \schemestart
            \chemfig{*6(-=-=-=)}
            \arrow{->[\ce{2Li}][\ce{NH3 / EtOH}]}[,1.6]
            \chemfig{*6(=--=--)}
            \arrow{0}[,0]\+
            2 \chemfig{LiOEt}
        \schemestop
    \end{center}
    \begin{itemize}
        \item Creates a singly-reduced, dearomatized system.
        \item Sodium and potassium metals can also be used (in place of lithium).
        \item This is similar to alkyne reduction.
    \end{itemize}
    \item Mechanism.
    \begin{figure}[h!]
        \centering
        \footnotesize
        \schemestart
            \chemfig{*6(-@{C1b}=[@{db1b}]-[@{sb1}]=[@{db1a}]@{C1a}-=)}
            \arrow{->[\chemfig{@{Li2}\charge{0=\.}{Li}}][-\ce{Li+}]}[,1.5]
            \chemfig{*6(-@{C3}\charge{-45=\:,-135:3pt=$\ominus$}{}(-H)-=-\charge{45=\.}{}(-H)-=)}
            \arrow{->[\chemfig{Et@{O4}O-[@{sb4}]@{H4}H}][-\ce{EtO-}]}[,1.5]
            \chemfig{*6(-(-[:-70]H)(-[:-110]H)-=-@{C5}\charge{45=\.}{}(-H)-=-)}
            \arrow{->[\chemfig{@{Li6}\charge{0=\.}{Li}}][-\ce{Li+}]}[,1.5]
            \chemfig{*6(-(-[:-70]H)(-[:-110]H)-=-@{C7}\charge{45=\:,135:3pt=$\ominus$}{}(-H)-=)}
            \arrow{->[\chemfig{Et@{O8}O-[@{sb8}]@{H8}H}][-\ce{EtO-}]}[,1.5]
            \chemfig{*6(--=--=)}
        \schemestop
        \chemmove{
            \draw [rex,semithick,shorten <=2pt,shorten >=2pt,arrows={-Stealth[harpoon,swap]}] ([xshift=3pt]Li2.east) to[bend right=60] (C1a);
            \draw [rex,semithick,shorten <=4pt,shorten >=2pt] (db1a) to[bend right=60,looseness=2] (sb1);
            \draw [rex,semithick,shorten <=4pt,shorten >=4pt] (db1b) to[bend right=90,looseness=4] (C1b);
            \draw [rex,semithick,shorten <=6pt,shorten >=2pt] (C3) to[out=-45,in=120,out looseness=1.5,in looseness=3.5] (H4);
            \draw [rex,semithick,shorten <=2pt,shorten >=2pt] (sb4) to[bend right=70,looseness=2.5] (O4);
            \draw [rex,semithick,shorten <=2pt,shorten >=6pt,arrows={-Stealth[harpoon,swap]}] ([xshift=3pt]Li6.east) to[bend right=60] (C5);
            \draw [rex,semithick,shorten <=6pt,shorten >=2pt] (C7) to[out=45,in=120,looseness=1.2] (H8);
            \draw [rex,semithick,shorten <=2pt,shorten >=2pt] (sb8) to[bend right=70,looseness=2.5] (O8);
        }
        \caption{Birch reduction mechanism.}
        \label{fig:Birchmechanism}
    \end{figure}
    \begin{itemize}
        \item Although we draw the lithium radical directly attacking benzene, in reality, lithium gives up one of its electrons to become a cation, and this electron is solvated by \ce{NH3}.
    \end{itemize}
    \item One more reaction.
    \item General form.
    \begin{equation*}
        \ce{PhBr ->[NaNH2][NH3] PhNH2}
    \end{equation*}
    \begin{itemize}
        \item Also works with other alkali metals.
    \end{itemize}
    \item A radiolabeling study.
    \begin{figure}[h!]
        \centering
        \footnotesize
        \schemestart
            \chemfig{*6(-(-[,,,,white]\phantom{Br})=-=@{C1}(-Br)-=)}
            \arrow
            \chemfig{*6(-(-[,,,,white]\phantom{N})=-=@{C2}(-NH_2)-=)}
            \arrow{0}[,0]\+{1em,1em,0.5em}
            \chemfig{*6(-=(-[,,,,white]\phantom{N})-(-NH_2)=@{C3}-=)}
        \schemestop
        \chemmove{
            \fill [blx,opacity=0.3]
                (C1) circle (2mm)
                (C2) circle (2mm)
                (C3) circle (2mm)
            ;
        }
        \vspace{-3em}
        \caption{Radiolabeling bromobenzene and transforming it into aniline.}
        \label{fig:radiolabelBromobenzene}
    \end{figure}
    \begin{itemize}
        \item When we radiolabel the carbon to which bromine is initially bonded, we see that two products are formed in equimolar ratios.
        \item This means that something other than S\textsubscript{N}Ar is occurring, and that whatever is happening is proceeding through some sort of symmetric intermediate.
    \end{itemize}
    \item Mechanism.
    \begin{figure}[h!]
        \centering
        \footnotesize
        \begin{tikzpicture}
            \node{
                \chemnameinit{}
                \schemestart
                    \chemfig{*6(=-(-[@{sb1}]@{H1}H)=[@{db1}](-Br)-=-)}
                    \arrow{->[\chemfig{@{N2}\charge{180=\:,90:3pt=$\ominus$}{N}H_2}][-\ce{NH3}]}
                    \chemname[1em]{\chemfig{*6(=-@{C3b}~@{C3a}-(-[,,,,white])=-)}}{Benzyne}
                \schemestop
            };
            \node at (6,1) {
                \chemfig{*6(=-=(-NH_2)-=-)}
            };
            \node at (6,-1) {
                \chemfig{*6(=-(-NH_2)=-=-)}
            };
    
            \draw [-CF] (2.7,0) -- ++(1,0) -- ++(0,1) -- node[above]{\chemfig{@{N4}\charge{180=\:,90:3pt=$\ominus$}{N}H_2}} ++(1,0);
            \draw [-CF] (3.7,0) -- ++(0,-1) -- node[below]{\chemfig{@{N5}\charge{180=\:,90:3pt=$\ominus$}{N}H_2}} ++(1,0);
        \end{tikzpicture}
        \chemmove{
            \draw [rex,semithick,shorten <=6pt,shorten >=2pt] (N2) to[out=180,in=90,looseness=1.5] (H1);
            \draw [rex,semithick,shorten <=2pt,shorten >=2pt] (sb1) to[bend right=60,looseness=1.5] (db1);
            \draw [rex,semithick,shorten <=6pt,shorten >=2pt] (N4) to[out=180,in=60] (C3a);
            \draw [rex,semithick,shorten <=6pt,shorten >=2pt] (N5) to[out=180,in=-60] (C3b);
        }
        \caption{Bromobenzene to aniline mechanism.}
        \label{fig:PhBrtoPhNH2}
    \end{figure}
    \begin{itemize}
        \item With a strong enough base, we can formally abstract a hydrogen from benzene to create an alkyne-like species.
        \item Orbitally, we can picture the triple bond in benzyne as a weak interaction (weak because of the nonlinearity/intense angle strain) between adjacent $p$ orbitals in the molecular plane.
    \end{itemize}
    \item Applications of this reaction and further mechanistic evidence.
    \begin{enumerate}
        \item Using the strong base \ce{KNH2}, we can generate the benzyne intermediate and then trap it with other nucleophiles, leading to an equimolar mixture of products.
        \item We can also trap benzyne by using it as the dieneophile in a Diels-Alder reaction.
        \item Lastly, we note that 2-bromo-1,3-dimethylbenzene does not react under these conditions, confirming the need for an $\alpha$-proton to make the benzyne intermediate.
    \end{enumerate}
\end{itemize}



\section{Review / Alcohols}
\begin{itemize}
    \item \marginnote{2/24:}Practice synthesis problems.
    \begin{center}
        \footnotesize
        \setchemfig{atom sep=1.4em}
        \schemestart
            \chemfig{*6(-=-=-=)}
            \arrow{->[\ce{HNO3, H2SO4}][-\ce{H2O}]}[,1.8]
            \chemfig{*6(-=-=(-NO_2)-=)}
            \arrow{->[\ce{Cl2}][\ce{AlCl3}]}
            \chemfig{*6(-=(-Cl)-=(-NO_2)-=)}
            \arrow{->[\ce{H2}][\ce{Pd/C}]}
            \chemfig{*6(-=(-Cl)-=(-NH_2)-=)}
            \arrow{->[\ce{NaNO2}][\ce{HCl}]}[,1.2]
            \chemfig{*6(-=(-Cl)-=(-\charge{45:1pt=$\oplus$}{N}_2)-=)}
            \arrow{->[\ce{Cu2O}][\ce{H2O}]}[,1.1]
            \chemfig{*6(-=(-Cl)-=(-OH)-=)}
        \schemestop
    \end{center}
    \begin{center}
        \footnotesize
        \setchemfig{atom sep=1.4em}
        \schemestart
            \chemfig{*6(-(-[,,,,white]\phantom{O})=-=(-OMe)-=)}
            \arrow{->[\ce{Cl2}][\ce{AlCl3}]}
            \chemfig{*6(-(-Cl)=-=(-OMe)-=)}
            \arrow{->[\ce{2HNO3, 2H2SO4}][-\ce{2H2O}]}[,2.1]
            \chemfig{*6(-(-Cl)=-(-NO_2)=(-OMe)-(-O_2N)=)}
            \arrow{->[\ce{H2}][\ce{Pd/C}]}
            \chemfig{*6(-(-Cl)=-(-NH_2)=(-OMe)-(-H_2N)=)}
        \schemestop
    \end{center}
    \item Takeaways.
    \begin{itemize}
        \item Don't be afraid to get another isomer than what you need and chuck it out if you have to.
        \item It's always better to activate first and deactivate later if possible.
    \end{itemize}
    \item One new reaction.
    \begin{figure}[h!]
        \centering
        \footnotesize
        \schemestart
            \chemfig{*6(-(-[,,,,white]\phantom{O})=-=(-OMe)-=)}
            \arrow{->[\ce{SO3}][\ce{H2SO4}]}[,1.1]
            \chemfig{*6(-(-SO_3H)=-=(-OMe)-=)}
            \arrow{->[\ce{HNO3}][\ce{H2SO4}]}[,1.2]
            \chemfig{*6(-(-SO_3H)=-(-NO_2)=(-OMe)-(-O_2N)=)}
            \arrow{->[\ce{H+ / H2O}][$\Delta$]}[,1.4]
            \chemfig{*6(-(-[,,,,white]\phantom{O})=-(-NO_2)=(-OMe)-(-O_2N)=)}
            % \arrow{->[\ce{Cl2}][\ce{AlCl3}]}
            % \chemfig{*6(-(-Cl)=-(-NO_2)=(-OMe)-(-O_2N)=)}
        \schemestop
        \caption{Protecting groups.}
        \label{fig:protectingGroup}
    \end{figure}
    \begin{itemize}
        \item Make use of a protecting group.\par
        \item Note that the sulfate group adds para due to sterics.
        \item To finish the synthesis, just chlorinate para (\ce{Cl2 / AlCl3}) and reduce the nitro groups (\ce{H2 + Pd/C}).
    \end{itemize}
    \item If they want us to draw all of the resonance structures, they'll ask. Most likely yes in a mechanism but no in a synthesis.
    \item Answer to PSet 4, Q6 given.
    \item For PSet 4, Q7a, recall that you can't run F-C alkylation when you have EWGs on the ring.
    \item PSet 4, Q7b's issue is not the reactivity. Guessing the actual product is p-bromoisopropylbenzene.
    \item Hint for PSet 4, Q8: You can create an aromatic ring by brominating one of the alkenes in cyclohexa-1,4-diene and then doing E2 twice.
    \begin{itemize}
        \item Looks like Dickinson may be asking us to invent new stuff on PSets and the exam.
    \end{itemize}
    \item Synthesis and reactivity of alcohols.
    \item Alcohols have unique properties.
    \begin{itemize}
        \item Boiling points: Alcohols significantly raise the boiling points of the compounds to which they're attached (because of hydrogen bonding).
    \end{itemize}
    \item General reactivity of alcohols.
    \begin{itemize}
        \item Adding acid to an alcohol makes water a leaving group, yielding an alkyl carbocation that can then react with nucleophiles.
        \item Adding a very strong base/good nucleophile (a Grignard reagent) leads to the creation of an alkoxide (and the fully protonated Grignard species as a side product).
    \end{itemize}
    \item Relative strengths of nucleophiles.
    \begin{equation*}
        \ce{{}^-NRH} > \ce{RO- / HO-}
        > \ce{Br-}
        > \ce{NR3}
        > \ce{Cl-}
        > \ce{F-}
        > \ce{H2O / ROH}
        > \text{alkene}
        > \text{benzene}
    \end{equation*}
    \item Acidity effects.
    \begin{itemize}
        \item $1^\circ>2^\circ>3^\circ$.
        \item More inductive donating effects (e.g., from alkyl groups) means more destabilization of the conjugate base.
        \item On the other hand, \ce{CF3CH2OH} has a much lower $\pKa$ because of the strong inductive withdrawing effects and resultant delocalization.
        \item Similarly, phenoxide is stabilized via resonance.
        \item At the extreme, \ce{(CF3)3COH} is a true acid (will be predominantly deprotonated in water).
        \item Inductive and resonance effects can be mixed, too: 2,4,6-trinitrophenol\footnote{Also known as picric acid.} is a very strong acid ($\pKa=0.6$).
    \end{itemize}
    \item Alkoxide generation.
    \begin{enumerate}
        \item \ce{EtOH + NaOH <=> NaOEt + H2O}.
        \item \ce{EtOH + Na${}^\circ$ -> NaOEt + \frac{1}{2}H2}.
        \begin{itemize}
            \item \ce{Na${}^\circ$} is sodium metal.
            \item This is a strongly exothermic reaction and a dangerous one (since \ce{H2} is explosive).
            \item It is common in laboratory use, though.
        \end{itemize}
        \item \ce{CyOH + NaNH2 <=>> NaOCy + NH3}.
        \begin{itemize}
            \item A more common form of this reaction uses LDA (lithium diisopropylamine), a sterically hindered strong base, instead of \ce{NaNH2}.
        \end{itemize}
        \item \ce{$i$-BuOH + NaH <=>> NaOBu^{$i$} + H2}.
        \item \ce{CH3OH + LiMe <=>> LiOMe + CH4}.
        \begin{itemize}
            \item We can also use \ce{LiBu}, \ce{MeMgBr}, etc. as other sources of carbanions.
        \end{itemize}
        \item \ce{PhOH + NaOH <=>> NaOPh + H2O}.
    \end{enumerate}
\end{itemize}



\section{Office Hours (Dickinson)}
\begin{itemize}
    \item Is the order of the deactivating halogens reversed?
    \begin{itemize}
        \item Yes --- fluorine should be the most deactivating. The way I have it drawn in Figure \ref{fig:DeActivators} is correct.
    \end{itemize}
    \item Why would the Clemmensen reduction work for reducing a nitro group to an amine --- isn't it for carbonyls?
    \begin{itemize}
        \item Don't get caught up on the name. The same reagents do the same thing in a few contexts; it's just using them to reduce ketones in particular that is termed the "Clemmensen reduction."
    \end{itemize}
    \item Electron flow in Figure \ref{fig:PhBrtoPhNH2}?
    \item Sulfonation vs. sulfation?
    \begin{itemize}
        \item Would have to ask the IUPAC, but he could have it backwards. There probably isn't any issue though.
    \end{itemize}
\end{itemize}




\end{document}