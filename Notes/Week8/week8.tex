\documentclass[../notes.tex]{subfiles}

\pagestyle{main}
\renewcommand{\chaptermark}[1]{\markboth{\chaptername\ \thechapter\ (#1)}{}}
\setcounter{chapter}{7}

\begin{document}




\chapter{Alcohols, Ethers, and Epoxides}
\section{Office Hours (Salinas)}
\begin{itemize}
    \item \marginnote{2/28:}Does \ce{H2 + Pd/C} hydrogenate ketones or not? Conflict between Lecture 11 and 2020 Exam 2A Q1e.
    \begin{itemize}
        \item Either way.
        \item \ce{H2 + Pd/C} hydrogenates \emph{benzylic} ketones only; it will leave ketones that are farther away from the benzene ring alone.
        \item \ce{Zn(Hg) + HCl} hydrogenates all ketones, but nothing else.
    \end{itemize}
    \item When do alkenes in PAHs get hydrogenated?
    \begin{itemize}
        \item Ones that are added onto the Rocks of Gibraltar molecules.
    \end{itemize}
    \item Do we have to know that aryl amines present a problem in F-C alkyl/arylations? It seems like there's a lot of content on this exam that BCD never went over.
    \begin{itemize}
        \item Things like this probably won't show up on the exam.
    \end{itemize}
    \item Can we use \ce{HCN + \ce{NaCN}} to substitute \ce{CN}?
    \begin{itemize}
        \item This would work, but Sandmeyer is the go-to.
    \end{itemize}
    \item How do you indicate you want to do something twice (e.g., bromination on 2020 Exam 2A Q3a)?
    \begin{itemize}
        \item Write (2x): For example, "\ce{Br2 / FeBr3} (2x)".
    \end{itemize}
    \item Is it \ce{KMnO4} (2020 Exam 2A answer key), \ce{KMnO4 / H2O} (class), \ce{KMnO4 / NaOH + $\Delta$} (PSet 4 key), or \ce{KMnO4 / NaOH + $\Delta$} followed by \ce{H3O+} (PSet 4 key) for benzoic acid formation?
    \begin{itemize}
        \item \ce{KMnO4 + H2O} is pretty solid.
    \end{itemize}
    \item 2020 Exam 2A Q3c: Is it preferable to use S\textsubscript{N}Ar or a novel Sandmeyer reaction? What are the limits of the Sandmeyer reaction?
    \begin{itemize}
        \item Note that we can achieve meta addition of an amine when an o/p-director is present by brominating para and then using the benzyne intermediate.
    \end{itemize}
    \item 2020 Exam 2A Q3d: Is \ce{SnCl2 / H2O} selective reduction of nitro groups?
    \begin{itemize}
        \item Perhaps, Omar will get back to me on whether to use \ce{SnCl2 / H2O} or \ce{H2 + Pd/C}.
    \end{itemize}
    \item When adding an alkane via F-C alkylation to later be transformed into a benzoic acid, is it preferable to use 2-chloropropane for some reason?
    \begin{itemize}
        \item Anything's fine.
    \end{itemize}
    \item PSet 4 2021 1f/h:
    \begin{figure}[H]
        \centering
        \begin{tikzpicture}
            \footnotesize
            \node at (0,0) {\chemfig{*6(-(-)=-=(-[,,,,white])-=)}};
            \draw (1,0) -- node[above]{\ce{Br2}} node[below]{\ce{FeBr3}} ++(1,0);
            \draw [CF-CF] (3,3) -- ++(-1,0) -- ++(0,-6) -- ++(1,0);
    
            \node at (4,3) {\chemfig{*6(-(-\phantom{Br})=(-Br)-=(-[,,,,white])-(-[,,,,white]\phantom{Br})=)}};
            \node at (4,-3) {\chemfig{*6(-(-\phantom{Br})=-=(-Br)-=)}};
            \draw (6,3) -- node[above]{\ce{HNO3}} node[below]{\ce{H2SO4}} ++(1,0);
            \draw [CF-CF] (8,4.5) -- ++(-1,0) -- ++(0,-3) -- ++(1,0);
            \draw (5,-3) -- node[above]{\ce{HNO3}} node[below]{\ce{H2SO4}} ++(2,0);
            \draw [CF-CF] (8,-1.5) -- ++(-1,0) -- ++(0,-3) -- ++(1,0);
    
            \node at (10,4.5) {\chemfig{O_2N-[:30]*6(-(-)=(-[,,,,white]\phantom{NO_2})(-Br)-=(-[,,,,white])-=)}};
            \node at (10,1.5) {\chemfig{\phantom{Br}-[:30,,,,white]*6(-(-[,,,,white]\phantom{NO_2})(-)=(-Br)-=(-NO_2)-=)}};
            \node at (10,-1.5) {\chemfig{*6(-(-\phantom{Br})=-(-NO_2)=(-Br)-(-[,,,,white]\phantom{O_2N})=)}};
            \node at (10,-4.5) {\chemfig{*6(-(-\phantom{Br})=-(-[,,,,white]\phantom{NO_2})=(-Br)-(-O_2N)=)}};
        \end{tikzpicture}
        \caption{Major and minor synthesis products.}
        \label{fig:majorMinor}
    \end{figure}
    \begin{itemize}
        \item When asked to determine major/minor when it could be kind of ambiguous, assume equimolar concentrations of reactants after the step before the last step.
        \item In the example above, notice how the two products on the bottom are identical, so they constitute the major product.
    \end{itemize}
\end{itemize}
\newpage



\section{Exam 2 Cheat Sheet}
\textbf{Reactions}:
\begin{itemize}
    \item \ce{C6H6 ->[D3O+] C6D6}
    \item \ce{PhH ->[Br2][FeBr3] PhBr}
    \begin{itemize}
        \item \ce{AlCl3}, \ce{CuI2}.
    \end{itemize}
    \item \ce{PhH ->[HNO3][H2SO4] PhNO2}
    \item \ce{PhH ->[SO3][H2SO4] PhSO3H}
    \item \ce{PhH ->[RCOCl][AlCl3] PhCOR}
    \item \ce{PhH ->[RCl][AlCl3] PhR}
    \item \ce{\text{benzylic carbonyl} ->[Zn(Hg)][HCl] \text{reduced carbon}}
    \item \ce{PhR ->[KMnO4][H2O] PhCOOH}
    \begin{itemize}
        \item Needs benzylic hydrogen.
    \end{itemize}
    \item \ce{PhNO2 ->[\text{reagents}] PhNH2}
    \begin{itemize}
        \item \ce{H2 + Pd/C} or \ce{SnCl2 + H2O} (selective).
    \end{itemize}
    \item \ce{PhNH2 ->[NaNO2][HCl] PhN2+ + X-}
    \begin{itemize}
        \item Mechanism has many equilibrium steps (only first and last are not).
    \end{itemize}
    \item \ce{PhN2+ ->[Cu2O][H2O] PhOH}
    \begin{itemize}
        \item \ce{PhN2+ ->[CuCl] PhCl}
        \item \ce{PhN2+ ->[CuBr] PhBr}
        \item \ce{PhN2+ ->[CuI] PhI}
        \item \ce{PhN2+ ->[CuCN] PhCN}
    \end{itemize}
    \item \ce{PhN2+ ->[D3PO2] PhD}
    \item \ce{PhBr ->[NaNH2][NH3] PhNH2}
    \item \ce{PhCl ->[NaNu][NuH] PhNu}
    \item \ce{PhH ->[Pd][$>1000\,\text{psi}$] CyH}
    \item \ce{\text{benzene} ->[2Li][NH3 / EtOH] \text{cyclohexa-1,4-diene} + 2LiOEt}
\end{itemize}

\textbf{Reminders}:
\begin{itemize}
    \item Aromatic stabilization of benzene: \SI[per-mode=symbol]{-36.5}{\kilo\calorie\per\mole}.
    \item Frost method: Point down, MOs at the carbons.
    \begin{itemize}
        \item 5-membered rings: 3 bonding / 2 antibonding. 7-membered: 3 bonding / 4 antibonding.
    \end{itemize}
    \item Aromaticity checklist: Flat, cyclic, conjugated, uninterrupted flow of $p$-orbitals, $(4n+2)$-rule.
    \item $(+/-)$ for Diels-Alder reactions!
    \item F-C reactions happen ONLY IF there is not an EWG on the ring.
    \item Add stronger EWGs later.
    \item Nucleophile strengths.
    \begin{equation*}
        \ce{{}^-NRH} > \ce{RO- / HO-}
        > \ce{Br-}
        > \ce{NR3}
        > \ce{Cl-}
        > \ce{F-}
        > \ce{H2O / ROH}
        > \text{alkene}
        > \text{benzene}
    \end{equation*}
    \item Breslow (1967), Faraday (1825), Kekul\'{e} (1865), Jack Roberts (benzyne).
\end{itemize}

\begin{figure}[h!]
    \centering
    \footnotesize
    \begin{tikzpicture}
        \draw [stealth-stealth] (-9.5,0) node[above right,yshift=1mm]{deactivators} -- (6.5,0) node[above left,yshift=1mm]{activators};

        \draw
            (-9,0.1) -- ++(0,-0.2) node[below]{\ce{{}^+NR3}}
            (-8,0.1) -- ++(0,-0.2) node[below]{\ce{NO2}}
            (-7,0.1) -- ++(0,-0.2) node[below]{\ce{CN}}
            (-6,0.1) -- ++(0,-0.2) node[below]{\ce{SO3H}}
            (-5,0.1) -- ++(0,-0.2) node[below]{\chemfig[atom sep=1.4em]{O=(-[:60])(-[:-60]R)}}
            (-4,0.1) -- ++(0,-0.2) node[below]{\ce{CO2H}}
            (-3,0.1) -- ++(0,-0.2) node[below]{\ce{CHO}}
            (-2.25,0.1) -- ++(0,-0.2) node[below]{\ce{I}}
            (-1.75,0.1) -- ++(0,-0.2) node[below]{\ce{Br}}
            (-1.25,0.1) -- ++(0,-0.2) node[below]{\ce{Cl}}
            (-0.75,0.1) -- ++(0,-0.2) node[below]{\ce{F}}
            (0,0.2) -- ++(0,-0.4) node[below]{\chemfig[atom sep=1.4em]{*6(-=-=-=)}}
            (1,0.1) -- ++(0,-0.2) node[below]{aryl}
            (2,0.1) -- ++(0,-0.2) node[below]{alkyl}
            (3,0.1) -- ++(0,-0.2) node[below]{\chemfig[atom sep=1.4em]{HN-[:-60](=[::-60]O)-[::60]R}}
            (4,0.1) -- ++(0,-0.2) node[below]{\ce{OR}}
            (5,0.1) -- ++(0,-0.2) node[below]{\ce{OH}}
            (6,0.1) -- ++(0,-0.2) node[below]{\ce{NH2}}
        ;

        \draw [decorate,decoration={brace,mirror}] (-9.4,-1.5) -- node[below=1mm]{m-directing} (-2.6,-1.5);
        \draw [decorate,decoration={brace,mirror}] (-2.4,-1.5) -- node[below=1mm]{o/p-directing} (-0.6,-1.5);
        \draw [decorate,decoration={brace,mirror}] (0.6,-1.5) -- node[below=1mm]{o/p-directing} (6.4,-1.5);
    \end{tikzpicture}
    \caption*{Activators and deactivators.}
\end{figure}

\begin{figure}[h!]
    \centering
    \footnotesize
    \schemestart
        \chemfig{*6(-@{C1b}=[@{db1b}]-[@{sb1}]=[@{db1a}]@{C1a}-=)}
        \arrow{->[\chemfig{@{Li2}\charge{0=\.}{Li}}][-\ce{Li+}]}[,1.5]
        \chemfig{*6(-@{C3}\charge{-45=\:,-135:3pt=$\ominus$}{}(-H)-=-\charge{45=\.}{}(-H)-=)}
        \arrow{->[\chemfig{Et@{O4}O-[@{sb4}]@{H4}H}][-\ce{EtO-}]}[,1.5]
        \chemfig{*6(-(-[:-70]H)(-[:-110]H)-=-@{C5}\charge{45=\.}{}(-H)-=-)}
        \arrow{->[\chemfig{@{Li6}\charge{0=\.}{Li}}][-\ce{Li+}]}[,1.5]
        \chemfig{*6(-(-[:-70]H)(-[:-110]H)-=-@{C7}\charge{45=\:,135:3pt=$\ominus$}{}(-H)-=)}
        \arrow{->[\chemfig{Et@{O8}O-[@{sb8}]@{H8}H}][-\ce{EtO-}]}[,1.5]
        \chemfig{*6(--=--=)}
    \schemestop
    \chemmove{
        \draw [rex,semithick,shorten <=2pt,shorten >=2pt,arrows={-Stealth[harpoon,swap]}] ([xshift=3pt]Li2.east) to[bend right=60] (C1a);
        \draw [rex,semithick,shorten <=4pt,shorten >=2pt] (db1a) to[bend right=60,looseness=2] (sb1);
        \draw [rex,semithick,shorten <=4pt,shorten >=4pt] (db1b) to[bend right=90,looseness=4] (C1b);
        \draw [rex,semithick,shorten <=6pt,shorten >=2pt] (C3) to[out=-45,in=120,out looseness=1.5,in looseness=3.5] (H4);
        \draw [rex,semithick,shorten <=2pt,shorten >=2pt] (sb4) to[bend right=70,looseness=2.5] (O4);
        \draw [rex,semithick,shorten <=2pt,shorten >=6pt,arrows={-Stealth[harpoon,swap]}] ([xshift=3pt]Li6.east) to[bend right=60] (C5);
        \draw [rex,semithick,shorten <=6pt,shorten >=2pt] (C7) to[out=45,in=120,looseness=1.2] (H8);
        \draw [rex,semithick,shorten <=2pt,shorten >=2pt] (sb8) to[bend right=70,looseness=2.5] (O8);
    }
    \caption*{Birch reduction mechanism.}
\end{figure}

\begin{figure}[h!]
    \centering
    \footnotesize
    \begin{subfigure}[b]{\linewidth}
        \centering
        \schemestart
            \chemfig{R-[:30](=[2]O)-[:-30]@{Cl1}\charge{90=\:}{Cl}}
            \arrow{->[\chemfig{@{Al2}AlCl_3}]}[,1.1]
            \chemfig{R-[:30](=[@{db3}2]@{O3}\charge{[extra sep=1.5pt]45=\:,135=\:}{O})-[@{sb3}:-30]@{Cl3}\charge{-90:3pt=$\oplus$}{Cl}-[:30]\charge{90:3pt=$\ominus$}{Al}Cl_3}
            \arrow
            \chemleft{[}
                \subscheme{
                    \chemfig{R-~[@{tb4}]@{O4}\charge{90:3pt=$\oplus$}{O}}
                    \arrow{<->}[-90]\arrow{0}[-90,0.1]
                    \chemfig{R-[:30]\charge{90:3pt=$\oplus$}{}=_[:-30]O}
                }
            \chemright{]}
            \arrow{0}[,0]\+{,,-3em}
            \chemleft{[}
                \subscheme{
                    \chemfig{@{Cl6}Cl-[@{sb6}]\charge{90:3pt=$\ominus$}{Al}Cl_3}
                    \arrow{<->}[-90]
                    \subscheme{
                        \chemfig{\charge{45:1pt=$\ominus$}{Cl}}
                        \+{1em,,0em}
                        \chemfig{AlCl_3}
                    }
                }
            \chemright{]}
        \schemestop
        \chemmove{
            \draw [rex,semithick,shorten <=6pt,shorten >=2pt] (Cl1) to[out=90,in=180,looseness=1.5] (Al2);
            \draw [rex,semithick,shorten <=6pt,shorten >=3pt] (O3) to[out=135,in=180,looseness=4] (db3);
            \draw [rex,semithick,shorten <=2pt,shorten >=2pt] (sb3) to[out=60,in=90,looseness=2.5] (Cl3);
            \draw [rex,semithick,shorten <=4pt,shorten >=2pt] (tb4) to[bend right=90,looseness=3] (O4);
            \draw [rex,semithick,shorten <=2pt,shorten >=2pt] (sb6) to[bend left=90,looseness=3] (Cl6);
        }
        \caption{Acylium ion formation.}
    \end{subfigure}\\[2em]
    \begin{subfigure}[b]{\linewidth}
        \centering
        \schemestart
            \chemfig{*6(-=-=[@{db1}]-=)}
            \+{,,0.8em}
            \chemfig{R-[:30]@{C2}\charge{90:3pt=$\oplus$}{}=_[:-30]O}
            \arrow[,1.1]
            \chemfig{*6(-=-(-[0](=[:60]O)-[:-60]R)(-[@{sb3a}:80]@{H3}H)-[@{sb3b}]\charge{90:3pt=$\oplus$}{}-=)}
            \arrow{->[\chemfig{@{Cl4}\charge{45:2pt=$\ominus$,90=\:}{Cl}}][-\ce{HCl}]}[,1.1]
            \chemfig{*6(-=-(-(=[2]O)-[:-30]R)=-=)}
        \schemestop
        \chemmove{
            \draw [rex,semithick,shorten <=2pt,shorten >=3pt] (db1) to[out=60,in=170] (C2);
            \draw [rex,semithick,shorten <=6pt,shorten >=2pt] (Cl4) to[out=90,in=60,looseness=1.5] (H3);
            \draw [rex,semithick,shorten <=2pt,shorten >=2pt] (sb3a) to[bend right=70,looseness=2.5] (sb3b);
        }
        \caption{Acylation of benzene.}
    \end{subfigure}
    \caption*{Friedel-Crafts acylation mechanism.}
\end{figure}




\end{document}