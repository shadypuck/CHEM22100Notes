\documentclass[../notes.tex]{subfiles}

\pagestyle{main}
\renewcommand{\chaptermark}[1]{\markboth{\chaptername\ \thechapter\ (#1)}{}}
\setcounter{chapter}{8}

\begin{document}




\chapter{Oxidation/Reduction and Organometallics}
\section{Reduction of Carbonyls}
\begin{itemize}
    \item \marginnote{3/8:}In general chemistry, oxidation and reduction referred to the loss and gain of electrons, respectively.
    \begin{itemize}
        \item In organic chemistry, we think about it differently.
    \end{itemize}
    \item \textbf{Organic oxidation}: Increasing the number of bonds to oxygen or decreasing the number of bonds to hydrogen.
    \item \textbf{Organic reduction}: Decreasing the bonds to oxygen or increasing the bonds to hydrogen.
    \item Example: Ethene to ethanol is neither an oxidation or reduction since the \ce{C-O} bond formed is cancelled by the \ce{C-H} bond formed.
    \item We now transition to carbonyl chemistry, which will also be really important next quarter.
    \item \textbf{Carbonyl}: Any carbon-oxygen double-bonded system.
    \begin{itemize}
        \item Important derivatives include aldehydes, ketones, carboxylic acids, esters, and amides.
        \item A defining character of carbonyls is their resonance, which we can formalize by representing them as an oxygen anion and a carbocation.
    \end{itemize}
    \item General reactivity of carbonyls.
    \begin{enumerate}
        \item Nucleophiles can add to the carbonyl carbon. A slightly acidic aqueous workup from here can form an alcohol.
        \item Oxidation/reduction. Alcohol to carbonyl and vice versa.
    \end{enumerate}
    \item Reduction of aldehydes and ketones.
    \item General form.
    \begin{equation*}
        \ce{RCOR$'$ ->[\text{reagents}] RC(OH)HR$'$}
    \end{equation*}
    \begin{itemize}
        \item This is a two-step process. We first need a source of \ce{H-}, and then an acidic workup.
        \begin{itemize}
            \item Possible hydride sources are \ce{NaBH4} (a weak source) and \ce{LiAlH4} (a strong source).
            \item The acidic workup reagents are always \ce{H3O+, H2O}.
        \end{itemize}
    \end{itemize}
    \item Mechanism.
    \begin{itemize}
        \item We use the hydride as a nucleophile to attack the carbonyl carbon, and then the acid to protonate the alkoxide intermediate.
    \end{itemize}
    \item Varying types of carbonyls.
    \begin{itemize}
        \item Aldehydes and ketones go through the full reaction with both reagents.
        \item Esters do not react with \ce{NaBH4} (not powerful enough), but do react with \ce{LiAlH4}. However, they form a primary alcohol in this case.
    \end{itemize}
    \item Reactivity of carbonyls.
    \begin{figure}[h!]
        \centering
        \footnotesize
        \begin{equation*}
            \schemestart[][.-165]
                \chemfig{R-[:30](=[2]O)-[:-30]H}
            \schemestop
            \quad>\quad
            \schemestart[][.-165]
                \chemfig{R-[:30](=[2]O)-[:-30]R'}
            \schemestop
            \quad>\quad
            \schemestart[][.-165]
                \chemfig{R-[:30](=[2]O)-[:-30]O-[:30]R'}
            \schemestop
            \quad>\quad
            \schemestart[][.-165]
                \chemfig{R-[:30](=[2]O)-[:-30]OH}
            \schemestop
        \end{equation*}
        \caption{Reactivity of carbonyls.}
        \label{fig:carbonylReactivity}
    \end{figure}
    \begin{itemize}
        \item \ce{NaBH4} stops working after ketones.
    \end{itemize}
    \item Reduction of esters.
    \item General form.
    \begin{center}
        \footnotesize
        \setchemfig{atom sep=1.4em}
        \schemestart
            \chemfig{R-[:30](=[2]O)-[:-30]O-[:30]R'}
            \arrow{->[1. \ce{LiAlH4}][2. \ce{H3O+, H2O}]}[,1.8]
            \chemfig{R-[:30]-[2]OH}
            \+
            \chemfig{HO-R'}
        \schemestop
    \end{center}
    \item Mechanism.
    \begin{figure}[h!]
        \centering
        \footnotesize
        \schemestart
            \chemfig{R-[:30]@{C1}(=[@{db1}2]O)-[:-30]O-[:30]R'}
            \arrow{->[*{0}\chemfig{@{Li2}\charge{45:1pt=$\oplus$}{Li}-[,0.35,,,white]-[2,0.5,,,white]}][*{0}\chemfig{@{H3}\charge{-90=\:,45=$\ominus$}{H}-[2,0.3,,,white]}]}[90]
            \chemfig{R-[:30](-[2]O-[:30]Li)(-[:-70]H)(-[:-30]OR')}
            \arrow{<=>}
            \chemfig{R-[:30](-[@{sb5a}2]@{O5a}\charge{180=\:,45:1pt=$\ominus$}{O}-[,0.5,,,white]\charge{45:1pt=$\oplus$}{Li})(-[:-70]H)(-[@{sb5b}:-30]@{O5b}OR')}
            \arrow{->[][-\ce{Li+}]}
            \chemfig{R-[:30]@{C6}(=[@{db6}2]@{O6}O)-[:-30]H}
            \arrow{0}[,0.1]\+
            \chemfig{\charge{135:1pt=$\ominus$}{O}R'}
            \arrow{->[\chemfig{@{H8}\charge{90=\:,45=$\ominus$}{H}}]}
            \subscheme{
                \chemfig{R-[:30](-[2]@{O9}\charge{180=\:,45:1pt=$\ominus$}{O})(-[:-70]H)(-[:-30]H)}
                \arrow{0}[,0.1]\+
                \chemfig{@{O10}\charge{-90=\:,135:1pt=$\ominus$}{O}R'}
            }
            \arrow{->[*{0}\setchemfig{atom sep=1.4em}\chemfig{@{H11}H-[@{sb11}]@{O11}\charge{90:3pt=$\oplus$}{O}H_2}][*{0}\setchemfig{atom sep=1.4em}\chemfig{@{H12}H-[@{sb12}]@{O12}\charge{90:3pt=$\oplus$}{O}H_2}]}[-90]
            \subscheme{
                \chemfig{R-[:30]-[2]OH}
                \arrow{0}[,0.1]\+
                \chemfig{HOR'}
                \+
                2\,\chemfig{H_2O}
            }
        \schemestop
        \chemmove{
            \draw [rex,semithick,shorten <=6pt,shorten >=3pt] (H3) to[out=-90,in=30] (C1);
            \draw [rex,semithick,shorten <=3pt,shorten >=2pt] (db1) to[bend left=60,looseness=1.4] (Li2);
            \draw [rex,semithick,shorten <=6pt,shorten >=2pt] (O5a) to[bend right=90,looseness=3] (sb5a);
            \draw [rex,semithick,shorten <=2pt,shorten >=2pt] (sb5b) to[bend left=90,looseness=3] (O5b.100);
            \draw [rex,semithick,shorten <=6pt,shorten >=3pt] (H8) to[out=90,in=30] (C6);
            \draw [rex,semithick,shorten <=3pt,shorten >=2pt] (db6) to[bend left=90,looseness=3] (O6);
            \draw [rex,semithick,shorten <=6pt,shorten >=2pt] (O9) to[out=180,in=90,out looseness=1.7] (H12);
            \draw [rex,semithick,shorten <=2pt,shorten >=2pt] (sb12) to[bend right=90,looseness=3] (O12);
            \draw [rex,semithick,shorten <=6pt,shorten >=2pt] (O10) to[out=-90,in=90] (H11);
            \draw [rex,semithick,shorten <=2pt,shorten >=2pt] (sb11) to[bend right=90,looseness=3] (O11);
        }
        \caption{Reduction of esters mechanism.}
        \label{fig:esterReduction}
    \end{figure}
    \begin{itemize}
        \item Positive lithium ions combine with the oxygen of the carbonyl in the first step. This activates the \ce{C=O} bond, making the carbon more electrophilic.
        \begin{itemize}
            \item Thus, by using \ce{LiAlH4}, we both make the electrophile stronger and introduce a stronger nucleophile.
        \end{itemize}
    \end{itemize}
    \item \textbf{Chemoselective} (reaction): React with one group in the presence of other "related" groups.
    \begin{itemize}
        \item For example, if we have a ketone and ester in the same molecule, reacting with \ce{NaBH4} / \ce{H3O+, H2O} will yield a chemoselective reduction of the ketone in the presence of an ester. (Reacting with \ce{LiAlH4} / \ce{H3O+, H2O} will alter both groups in a non-chemoselective fashion.)
    \end{itemize}
    \item Note that we can reduce alkyl halides to hydrocarbons with \ce{LiAlH4} / \ce{H3O+, H2O}.
    \item Reactivity of an $\alpha$-$\beta$ unsaturated compound.
    \item General form.
    \begin{center}
        \footnotesize
        \setchemfig{atom sep=1.4em}
        \begin{tikzpicture}
            \node{\chemfig{*6(--=-(=O)-(-=^[:-150])-)}};
            \draw (1,0) -- ++(1,0);
            \draw [CF-CF] (4,1.5) -- node[above]{1. \ce{NaBH4}} node[below]{2. \ce{H3O+, H2O}} ++(-2,0) -- ++(0,-3) -- node[above]{1. \ce{LiAlH4}} node[below]{2. \ce{H3O+, H2O}} ++(2,0);
    
            \node at (6.5,1.5) {
                \schemestart
                    \chemname{\chemfig{*6(--=-(-OH)-(-=^[:-150])-)}}{Minor product}
                    \+{1em,1em}
                    \chemname{\chemfig{*6(----(-OH)-(-=^[:-150])-)}}{Major product}
                \schemestop
            };
            \node at (6.5,-1.5) {
                \schemestart
                    \chemname{\chemfig{*6(--=-(-OH)-(-=^[:-150])-)}}{Major product}
                    \+{1em,1em}
                    \chemname{\chemfig{*6(----(-OH)-(-=^[:-150])-)}}{Minor product}
                \schemestop
            };
        \end{tikzpicture}
    \end{center}
    \begin{itemize}
        \item With \ce{NaBH4}, the major product has been reduced both at the ketone and the alkene.
        \item With \ce{LiAlH4}, the major product has been reduced at the ketone only.
        \item Note that the alkene that is not conjugated with the carbonyl is untouched.
    \end{itemize}
    \item Mechanism.
    \begin{figure}[h!]
        \centering
        \footnotesize
        \begin{subfigure}[b]{\linewidth}
            \centering
            \schemestart
                \chemfig{*6(--@{C1}\charge{-90:3pt=$\delta^+$}{}=[@{db1a}]-[@{sb1}](=[@{db1b}]@{O1}\charge{45:2pt=$\delta^-$}{O})-(-=^[:-150])-)}
                \arrow{->[\chemfig{@{H2}\charge{180=\:,45:1pt=$\ominus$}{H}}]}
                \chemleft{[}
                    \subscheme{
                        \chemfig{*6(---@{C3}=[@{db3}](-[@{sb3}]@{O3}\charge{180=\:,45:1pt=$\ominus$}{O})-(-=^[:-150]-[4,0.2,,,white])-)}
                        \arrow{<->}
                        \chemfig{*6(--(-[0,0.6,,,white])-\charge{45:1pt=$\ominus$}{}-@{C4}(=[@{db4}]@{O4}O)-(-=^[:-150])-)}
                    }
                \chemright{]}
                \arrow{->[\chemfig{@{H5}\charge{90=\:,45:1pt=$\ominus$}{H}}]}
                \chemfig{*6(---\charge{45:1pt=$\ominus$}{}-(-\charge{45:1pt=$\ominus$}{O})-(-=^[:-150])-)}
                \arrow{->[\ce{H3O+}][\ce{H2O}]}
                \chemfig{*6(----(-OH)-(-=^[:-150])-)}
            \schemestop
            \chemmove{
                \draw [rex,semithick,shorten <=6pt,shorten >=2pt] (H2) to[out=180,in=-30,looseness=2.5] (C1);
                \draw [rex,semithick,shorten <=4pt,shorten >=2pt] (db1a) to[bend left=60,looseness=2] (sb1);
                \draw [rex,semithick,shorten <=3pt,shorten >=2pt] (db1b) to[bend left=90,looseness=3] (O1);
                \draw [blx,semithick,shorten <=6pt,shorten >=2pt] (O3) to[bend right=90,looseness=3] (sb3);
                \draw [blx,semithick,shorten <=4pt,shorten >=4pt] (db3) to[bend right=90,looseness=5] (C3);
                \draw [rex,semithick,shorten <=6pt,shorten >=3pt] (H5) to[out=90,in=30] (C4);
                \draw [rex,semithick,shorten <=3pt,shorten >=2pt] (db4) to[bend left=90,looseness=3] (O4);
            }
            \caption{Reduction via \ce{NaBH4}.}
            \label{fig:alphaBetaa}
        \end{subfigure}\\[4em]
        \begin{subfigure}[b]{\linewidth}
            \centering
            \schemestart
                \chemfig{*6(--=-(=[@{db1}]O)-(-=^[:-150])-)}
                \arrow{->[\chemfig{@{Li2}\charge{45:1pt=$\oplus$}{Li}}]}
                \chemleft{[}
                    \subscheme{
                        \chemfig{*6(--=[@{db3}]-[@{sb3}]@{C3}\charge{135:3pt=$\oplus$}{}(-\charge{45:1pt=$\ominus$}{O}-[0,0.5,,,white]\charge{45:1pt=$\oplus$}{Li})-(-=^[:-150])-)}
                        \arrow{<->}
                        \chemfig{*6(--\charge{-45:3pt=$\oplus$}{}-(-[,0.6,,,white])=(-\charge{45:1pt=$\ominus$}{O}-[0,0.5,,,white]\charge{45:1pt=$\oplus$}{Li})-(-=^[:-150])-)}
                    }
                \chemright{]}
                \arrow{->[\chemfig{@{H5}\charge{90=\:,45:1pt=$\ominus$}{H}}][-\ce{Li+}]}
                \chemfig{*6(--=-(-\charge{45:1pt=$\ominus$}{O})-(-=^[:-150])-)}
                \arrow{->[\ce{H3O+}][\ce{H2O}]}
                \chemfig{*6(--=-(-OH)-(-=^[:-150])-)}
            \schemestop
            \chemmove{
                \draw [rex,semithick,shorten <=3pt,shorten >=2pt] (db1) to[bend left=45,looseness=1.2] (Li2);
                \draw [blx,semithick,shorten <=4pt,shorten >=2pt] (db3) to[bend left=60,looseness=2] (sb3);
                \draw [rex,semithick,shorten <=6pt,shorten >=2pt] (H5) to[out=90,in=30] (C3);
            }
            \caption{Reduction via \ce{LiAlH4}}
            \label{fig:alphaBetab}
        \end{subfigure}
        \caption{Reduction of an $\alpha$-$\beta$ unsaturated compound mechanism.}
        \label{fig:alphaBeta}
    \end{figure}
    \begin{itemize}
        \item On Figure \ref{fig:alphaBetaa}.
        \begin{itemize}
            \item In the leftmost molecule, resonance draws charge toward the electronegative oxygen, making the carbon at the end of the conjugated chain the most electrophilic site in the molecule. Thus, hydride attacks there.
            \item The resulting molecule has a ketone as one of its resonance structures, so since ketones are reactive to further hydride attacks, we take this to be the major contributor and react the molecule with hydride again.
            \item The $2-$ product can how be reduced with acid and water.
        \end{itemize}
        \item On Figure \ref{fig:alphaBetab}.
        \begin{itemize}
            \item When \ce{Li+} bonds to the oxygen, it creates a formal carbocation in the ring system that can be delocalized by resonance.
            \item However, the carbocation will preferentially exist as a $3^\circ$ carbocation, so the $\alpha$-carbon is the most electrophilic site in the molecule in this case, making hydride attack there.
        \end{itemize}
    \end{itemize}
    \item Grignard reagents provide a new way to form \ce{C-C} bonds.
    \item \textbf{Grignard reagent}: An alkyl magnesium halide compound.
    \begin{itemize}
        \item Creates carbanions that are both strong bases and strong nucleophiles.
    \end{itemize}
    \item Forming Grignard reagents.
    \begin{equation*}
        \ce{RBr ->[Mg${}^\circ$][Et2O] RMgBr}
    \end{equation*}
    \begin{itemize}
        \item We need an aprotic solvent such as diethyl ether to stabilize the positive \ce{Mg}.
        \begin{itemize}
            \item If there are acidic protons present, the Grignard will just deprotonate them.
        \end{itemize}
    \end{itemize}
    \item Common Grignard reagents.
    \begin{itemize}
        \item To add phenyl groups to systems, use phenylmagnesium chloride.
        \item To add alkenes to systems, use allylmagnesium bromide.
    \end{itemize}
    \item Making a Grignard reagent basically inverts the reactivity of the precursor: While the precursor alkyl halide is electrophilic, Grignards are very nucleophilic.
    \item Grignards can be made out of iodides, bromides, and chlorides.
    \begin{itemize}
        \item Iodides are more reactive than bromides, are more reactive than chlorides.
        \item We commonly find them as bromides, though.
    \end{itemize}
    \item We can use Grignards as nucleophiles in the reduction of formaldehyde.
    \begin{itemize}
        \item Creates primary alcohols.
    \end{itemize}
    \item Using an aldehyde makes a secondary alcohol.
    \item Using a ketone makes a tertiary alcohol.
    \item Using an ester adds the Grignard twice and kicks out an alcohol.
    \item Using a carboxylic acid protonates the alkyl part of the Grignard, releases a magnesium salt, and regenerates the carboxylic acid.
    \item Since Grignards deprotonate any acids present, we can't use them on molecules that contain alcohols, thiols, carboxylic acids, phenols, amines, and acetylenes.
    \item Organolithium reagents are conceptually identical to Grignards, but even more ionic/reactive.
    \item Forming organolithium reagents.
    \begin{equation*}
        \ce{RBr ->[2Li${}^\circ$][Et2O] RLi + LiBr}
    \end{equation*}
    \begin{itemize}
        \item Organolithium reagents are more ionic than Grignards.
        \begin{itemize}
            \item They are 40\% ionic; Grignards are much less.
        \end{itemize}
        \item Very reactive (nucleophile and base), but very dangerous, too.
    \end{itemize}
\end{itemize}



\section{Oxidation of Alcohols}
\begin{itemize}
    \item \marginnote{3/10:}Alcohol oxidation often occurs with the help of chromium (VI) reagents, of which there are three "flavors."
    \item \textbf{Collins reagent}: The compound \ce{CrO3}.
    \item General form.
    \begin{figure}[H]
        \centering
        \footnotesize
        \setchemfig{atom sep=1.4em}
        \begin{subfigure}[b]{\linewidth}
            \centering
            \schemestart
                \chemfig{R-[:30]-[:-30]OH}
                \arrow{->[\ce{CrO3, Py}][\ce{CH2Cl2}]}[,1.3]
                \chemfig{R-[:30](=[2]O)-[:-30]H}
            \schemestop
            \caption{Anhydrous oxidation of a primary alcohol.}
        \end{subfigure}\\[1em]
        \begin{subfigure}[b]{\linewidth}
            \centering
            \schemestart
                \chemfig{R-[:30](-[2]OH)-[:-30]R'}
                \arrow{->[\ce{CrO3, Py}][\ce{CH2Cl2}]}[,1.3]
                \chemfig{R-[:30](=[2]O)-[:-30]R'}
            \schemestop
            \caption{Anhydrous oxidation of a secondary alcohol.}
        \end{subfigure}\\[1em]
        \begin{subfigure}[b]{\linewidth}
            \centering
            \schemestart
                \chemfig{R-[:30]-[:-30]OH}
                \arrow{->[\ce{CrO3, Py}][\ce{H2O}]}[,1.3]
                \chemfig{R-[:30](=[2]O)-[:-30]OH}
            \schemestop
            \caption{Aqueous oxidation of a primary alcohol.}
        \end{subfigure}
    \end{figure}
    \item Mechanism.
    \begin{figure}[H]
        \centering
        \footnotesize
        \begin{subfigure}[b]{\linewidth}
            \centering
            \schemestart
                \subscheme{
                    \chemfig{R-[:30]-[:-30]@{O1}\charge{-90=\:}{O}-[:30]H}
                    \arrow{0}[,0.5]
                    \chemfig{@{Cr2}Cr(=[:-30]O)(=[@{db2}2]@{O2}O)(=[:-150]O)}
                }
                \arrow[-90]
                \chemfig{R-[:30]-[:-30]\charge{90:3pt=$\oplus$}{O}(-[6]H)-[:30]Cr(=[6]O)(=O)(-[2]\charge{45:1pt=$\ominus$}{O})}
                \arrow{<=>}
                \chemfig{R-[:30](-[@{sb4a}:70]@{H4}H)(-[:110]H)-[@{sb4b}:-30]O-[@{sb4c}:30]@{Cr4}Cr(=[6]O)(=O)(-[2]OH)}
                \arrow{-U>[\chemfig[atom sep=1.4em]{*6(-=-=@{N5}\charge{90=\:}{N}-=)}][\chemfig[atom sep=1.4em]{*6(-=-=\chemabove{\charge{45:2pt=$\oplus$}{N}}{H}-=)}]}[,1.5]
                \chemfig{R-[:30](=[2]O)-[:-30]H}
                \arrow{0}[,0.1]\+{,,0.5em}
                \chemfig{Cr(-[:-30]OH)(=[2]O)(=[:-150]O)}
            \schemestop
            \chemmove{
                \draw [rex,semithick,shorten <=6pt,shorten >=2pt] (O1) to[out=-90,in=150,looseness=1.3] (Cr2);
                \draw [rex,semithick,shorten <=3pt,shorten >=2pt] (db2) to[bend right=90,looseness=3] (O2);
                \draw [rex,semithick,shorten <=6pt,shorten >=2pt] (N5) to[out=90,in=90] (H4);
                \draw [rex,semithick,shorten <=2pt,shorten >=2pt] (sb4a) to[bend left=70,looseness=2] (sb4b);
                \draw [rex,semithick,shorten <=2pt,shorten >=2pt] (sb4c) to[bend left=70,looseness=2] (Cr4);
            }
            \caption{Anhydrous oxidation of a primary alcohol.}
            \label{fig:CollinsReagenta}
        \end{subfigure}\\[2em]
        \begin{subfigure}[b]{\linewidth}
            \centering
            \schemestart
                \subscheme{
                    \chemfig{R-[:30](=[2]O)-[:-30]H}
                    \arrow{0}[,0.5]
                    \chemfig{Cr(-[:-30]OH)(=[2]O)(=[:-150]O)}
                }
                \arrow{<=>[*{0}\ce{H2O}]}[-90]
                \chemfig{R-[:30](-[:70]OH)(-[:110]HO)-[:-30]H}
                \arrow{0}[,0.5]
                \chemfig{Cr(-[:-30]OH)(=[2]O)(=[:-150]O)}
                \arrow
                \chemfig{R-[:30](-[:-30]H)(-[:110]HO)-[:70]O-Cr(=[6]O)(-OH)(=[2]O)}
                \arrow{-U>[\chemfig[atom sep=1.4em]{*6(-=-=@{N5}\charge{90=\:}{N}-=)}][\chemfig[atom sep=1.4em]{*6(-=-=\chemabove{\charge{45:2pt=$\oplus$}{N}}{H}-=)}]}[,1.5]
                \chemfig{R-[:30](=[2]O)-[:-30]OH}
                \arrow{0}[,0.1]\+
                \chemfig{Cr(-[:-30]OH)(=[2]O)(=[:-150]O)}
            \schemestop
            \caption{Aqueous oxidation of a primary alcohol.}
            \label{fig:CollinsReagentb}
        \end{subfigure}
        \caption{Alcohol oxidation via Collins reagent mechanism.}
        \label{fig:CollinsReagent}
    \end{figure}
    \begin{itemize}
        \item On Figure \ref{fig:CollinsReagenta}.
        \begin{itemize}
            \item The reversible proton shift may be a 1- or 2-step process.
            \item Having a leaving group on the oxygen makes the protons on the $\alpha$ carbon weakly acidic, so pyridine can attack them.
        \end{itemize}
        \item On Figure \ref{fig:CollinsReagentb}.
        \begin{itemize}
            \item This reaction picks up directly where the other one left off. Essentially, if water is absent, the mechanism will stop after the sequence of steps in Figure \ref{fig:CollinsReagenta}, and if water is present, the mechanism will continue through the sequence of steps in Figure \ref{fig:CollinsReagenta}.
            \item In the beginning, we note that aldehydes are one of the most electrophilic carbon compounds.
            \item Thus, in the presence of water, aldehydes exist in equilibrium with \textbf{acetals}.
            \item Any acetal that is generated can react with chromium again and another equivalent of pyridine as in the previous mechanism, but this time to generate a carboxylic acid.
        \end{itemize}
    \end{itemize}
    \item Since water has such an effect on the mechanism, we should be sure specify in the case of oxidation reactions whether the reaction is run under aqueous or anhydrous conditions.
    \item \textbf{Jones reagent}: The mixture \ce{CrO3 + H2SO4_{(aq)}}.
    \item The general form is the same as with Collins reagent, except obviously for the reagents used. In particular\dots
    \begin{itemize}
        \item $1^\circ$ alcohols go to carboxylic acids (water is present).
        \item $2^\circ$ alcohols will go to ketones.
    \end{itemize}
    \item Mechanism.
    \begin{figure}[h!]
        \centering
        \footnotesize
        \begin{subfigure}[b]{\linewidth}
            \centering
            \schemestart
                \chemfig{Cr(=[:-30]O)(=[2]O)(=[@{db1}:-150]O)}
                \arrow{->[\chemfig[atom sep=1.4em]{@{H2}H-[@{sb2}]@{O2}OSO_3H}][-\ce{HSO4-}]}[,1.6]
                \chemfig{@{Cr3}\charge{135:1pt=$\oplus$}{Cr}(=[:-30]O)(=[2]O)(-[:-150]HO)}
                \arrow{->[\chemfig{@{O4}\charge{90=\:}{O}H_2}]}
                \chemfig{HO-Cr(=[6]O)(=[2]O)-\charge{135:1pt=$\oplus$}{O}(-[:60]H)(-[:-60]H)}
            \schemestop
            \chemmove{
                \draw [rex,semithick,shorten <=3pt,shorten >=2pt] (db1) to[out=120,in=90,looseness=2] (H2);
                \draw [rex,semithick,shorten <=2pt,shorten >=2pt] (sb2) to[bend left=90,looseness=3] (O2);
                \draw [rex,semithick,shorten <=6pt,shorten >=2pt] (O4) to[out=90,in=30,looseness=1.5] (Cr3);
            }
            \caption{Protonated chromic acid formation.}
            \label{fig:JonesReagenta}
        \end{subfigure}\\[2em]
        \begin{subfigure}[b]{\linewidth}
            \centering
            \schemestart
                \chemfig{R-[:30]-[:-30]@{O1}\charge{90=\:}{O}H}
                \arrow{->[\chemfig[atom sep=1.4em]{HO-@{Cr2}Cr(=[6]O)(=[2]O)-[@{sb2}]@{O2}\charge{90:3pt=$\oplus$}{O}H_2}][-\ce{H2O, H+}]}[,2]
                \chemfig{R-[:30](-[@{sb3a}2]@{H3}H)-[@{sb3b}:-30]O-[@{sb3c}:30]@{Cr3}Cr(=[6]O)(=[2]O)-OH}
                \arrow{->[\chemfig[atom sep=1.4em]{HS@{O4}\charge{90=\:,45:1pt=$\ominus$}{O}_4}][-\ce{HCrO3}]}[,1.2]
                \chemfig{R-[:30](=[2]O)-[:-30]H}
                \arrow{->[\ce{H2O}]}
                \chemfig{R-[:30](=[2]O)-[:-30]OH}
            \schemestop
            \chemmove{
                \draw [rex,semithick,shorten <=6pt,shorten >=2pt] (O1) to[out=90,in=-135] (Cr2);
                \draw [rex,semithick,shorten <=2pt,shorten >=2pt] (sb2) to[bend right=90,looseness=3] (O2);
                \draw [rex,semithick,shorten <=6pt,shorten >=2pt] (O4) to[bend right=70] (H3);
                \draw [rex,semithick,shorten <=2pt,shorten >=2pt] (sb3a) to[bend left=60,looseness=1.5] (sb3b);
                \draw [rex,semithick,shorten <=2pt,shorten >=2pt] (sb3c) to[bend left=70,looseness=2] (Cr3);
            }
            \caption{Oxidation of a $1^\circ$ alcohol.}
            \label{fig:JonesReagentb}
        \end{subfigure}
        \caption{Alcohol oxidation via Jones reagent mechanism.}
        \label{fig:JonesReagent}
    \end{figure}
    \item \textbf{Pyridinium chlorochromate}: The mixture \ce{Py, HCl, CrO3}. \emph{Also known as} \textbf{PCC}.
    \begin{itemize}
        \item Mixing these three compounds together yields \ce{PyH+} and \ce{CrO3Cl-}.
        \begin{itemize}
            \item Note that the chloride is bonded to the chromium center and one of the oxygens adopts the negative charge.
        \end{itemize}
        \item We've essentially suped up chromium by adding chloride as a leaving group.
        \item Running this in anhydrous conditions allows us to control reactivity (DCM is a good anhydrous solvent here).
    \end{itemize}
    \item As before, we take $1^\circ$ alcohols to carboxylic acids and $2^\circ$ alcohols to ketones.
    \item \textbf{Oxalyl chloride}: The following compound. \emph{Structure}
    \begin{figure}[H]
        \centering
        \footnotesize
        \chemfig{Cl-[:30](=[2]O)-[:-30](=[6]O)-[:30]Cl}
        \caption{Oxalyl chloride.}
        \label{fig:oxalylChloride}
    \end{figure}
    \item \textbf{Swern oxidation}: An eco-friendly alcohol oxidation mechanism that does away with toxic metal chromium.
    \item General form.
    \begin{center}
        \footnotesize
        \setchemfig{atom sep=1.4em}
        \schemestart
            \chemfig{R-[:30]-[:-30]OH}
            \arrow{->[1. DMSO, \ce{(COCl)2}][2. \ce{NEt3}\rule{5.1em}{0em}]}[,2.2]
            \chemfig{R-[:30](=[2]O)-[:-30]H}
        \schemestop
    \end{center}
    \item Mechanism.
    \begin{figure}[h!]
        \centering
        \footnotesize
        \begin{subfigure}[b]{\linewidth}
            \centering
            \schemestart
                \chemfig{Me-[:30]S(=[2]O)-[:-30]Me}
                \arrow{0}[,0.1]\+{,,0.5em}
                \chemfig{Cl-[:30](=[2]O)-[:-30](=[6]O)-[:30]Cl}
                \arrow
                \chemfig{Me-[:30]\charge{45:1pt=$\oplus$}{S}(-[2]Cl)-[:-30]Me}
            \schemestop
            \caption{Chlorodimethylsulfonium ion formation.}
            \label{fig:swernOxidationa}
        \end{subfigure}\\[2em]
        \begin{subfigure}[b]{\linewidth}
            \centering
            \schemestart
                \chemfig{R-[:30]-[:-30]@{O1}\charge{90=\:}{O}H}
                \arrow{->[\chemfig[atom sep=1.4em]{Me-[:30]@{S2}\charge{-90:3pt=$\oplus$}{S}(-[@{sb2}2]@{Cl2}Cl)-[:-30]Me}]}[,1.6]
                \chemfig{R-[:30]-[:-30]@{O3}\charge{90:3pt=$\oplus$}{O}(-[@{sb3}6]@{H3}H)-[:30]\charge{-90:3pt=$\oplus$}{S}(-[2]Me)-[:-30]Me}
                \arrow{0}[,0.1]\+
                \chemfig{@{Cl4}\charge{-90=\:,45:1pt=$\ominus$}{Cl}}
                \arrow{->[][-\ce{HCl}]}
                \chemfig{R-[:30](-[@{sb5a}2]@{H5}H)-[@{sb5b}:-30]O-[@{sb5c}:30]@{S5}\charge{-90:3pt=$\oplus$}{S}(-[2]Me)-[:-30]Me}
                \arrow{->[\chemfig{@{N6}\charge{90=\:}{N}Et_3}][-\ce{HNEt3+, Me2S}]}[,1.9]
                \chemfig{R-[:30](=[2]O)-[:-30]H}
            \schemestop
            \chemmove{
                \draw [rex,semithick,shorten <=6pt,shorten >=2pt] (O1) to[out=90,in=150,looseness=1.3] (S2);
                \draw [rex,semithick,shorten <=2pt,shorten >=2pt] (sb2) to[bend right=90,looseness=3] (Cl2);
                \draw [rex,semithick,shorten <=6pt,shorten >=2pt] (Cl4) to[out=-90,in=0] (H3);
                \draw [rex,semithick,shorten <=2pt,shorten >=2pt] (sb3) to[bend left=90,looseness=3] (O3);
                \draw [rex,semithick,shorten <=6pt,shorten >=2pt] (N6) to[out=90,in=90] (H5);
                \draw [rex,semithick,shorten <=2pt,shorten >=2pt] (sb5a) to[bend left=70,looseness=2] (sb5b);
                \draw [rex,semithick,shorten <=2pt,shorten >=2pt] (sb5c) to[bend left=80,looseness=2.5] (S5);
            }
            \caption{Oxidation of a $1^\circ$ alcohol.}
            \label{fig:swernOxidationb}
        \end{subfigure}
        \caption{Swern oxidation mechanism.}
        \label{fig:swernOxidation}
    \end{figure}
    \begin{itemize}
        \item We won't worry about how the first step proceeds because it's pretty complicated. However, know that it generates an electrophilic sulfur analogous to the chromium.
    \end{itemize}
    \item A side note on biology.
    \begin{itemize}
        \item Alcohol dehydrogenase (ADH1), an enzyme in our body, deals with \ce{EtOH} and other harmful alcohols by transforming them into acetaldehyde.
        \item Acetaldehyde is very toxic, though, but in the presence of ADH2 and \ce{H2O}, it will form acetic acid (vinegar, which is relatively nontoxic).
        \item Being flushed when you drink is a result of having a deficiency of ADH2.
        \begin{itemize}
            \item Many of the problems associated with drinking come from a buildup of acetaldehyde!
            \item You can take ADH1 inhibitors to keep the ethanol around for longer because that's safer than letting acetaldehyde build up.
        \end{itemize}
    \end{itemize}
    \item Protecting groups.
    \begin{itemize}
        \item Trimethylsilyl chloride (\ce{TMSCl}) is a common one.
        \item Adding it and then a weak base such as \ce{NEt3} with DCM as a solvent leads to the formation of a \textbf{silyl-protected alcohol} (see Figure \ref{fig:tosylate}).
    \end{itemize}
    \item \textbf{Silyl-protected alcohol}: A very stable form of an alcohol which allows the addition of Grignards, etc. to react with the rest of the compound in question. \emph{Given by}
    \begin{figure}[h!]
        \centering
        \footnotesize
        \chemfig{R-[:30]O-[:-30]Si(-[6]Me)(-Me)(-[2]Me)}
        \caption{Silyl-protected alcohol.}
        \label{fig:silylProtectedAlcohol}
    \end{figure}
    \begin{itemize}
        \item The silyl protecting group can be removed by acid (\ce{H3O+, H2O}). This kicks out TMSOH and the alcohol.
        \item It can also be removed by fluoride (\ce{F-}) followed by acid. This kicks out TMSF and \ce{RO-} which is protonated to become \ce{ROH}.
    \end{itemize}
    \item Example of using protecting groups in a synthesis:
    \begin{center}
        \footnotesize
        \setchemfig{atom sep=1.4em}
        \schemestart
            \chemfig{H-[:-30](=[6]O)-[:30]*6(---(-OH)---)}
            \arrow{->[*{0}\ce{TMSCl, NEt3}][*{0}\ce{CH2Cl2}]}[-90]
            \chemfig{H-[:-30](=[6]O)-[:30]*6(---(-OTMS)---)}
            \arrow{->[1. \ce{MeMgBr, Et2O}][2. \ce{H3O+, H2O}\rule{1.5em}{0pt}]}[,2]
            \chemfig{-[:-30](-[6]OH)-[:30]*6(---(-OTMS)---)}
            \arrow{->[PCC]}
            \chemfig{-[:-30](=[6]O)-[:30]*6(---(-OTMS)---)}
            \arrow{->[*{0}\ce{F-}][*{0}\ce{H3O+}]}[90]
            \chemfig{-[:-30](=[6]O)-[:30]*6(---(-OH)---)}
        \schemestop
    \end{center}
    \begin{itemize}
        \item Remember: Grignards are bases (look out for acidic protons), and make sure there are no other reactive sites on your molecule.
        \item You can get the products you want, though, via protection and deprotection.
    \end{itemize}
    \item Practice problem: Synthesize the end product using only carbon atoms from alcohols with 5 or fewer carbons.
    \begin{figure}[H]
        \centering
        \footnotesize
        \setchemfig{atom sep=1.4em}
        \begin{tikzpicture}
            \node at (0,2) {
                \schemestart
                    \chemfig{HO-[:30]-[:-30]-[:30]-[:-30]}
                    \arrow{->[1. \ce{PBr3, Py}\rule{0.8em}{0pt}][2. \ce{Mg${}^\circ$, Et2O}]}[,1.7]
                    \chemfig{BrMg-[:30]-[:-30]-[:30]-[:-30]}
                \schemestop
            };
            \node at (0,0) {
                \schemestart
                    \chemfig{[:15]*4(-(--[::-60]OH)---)}
                    \arrow{->[PCC]}
                    \chemfig{[:15]*4(-(-(=[::-60]O)-[::60]H)---)}
                \schemestop
            };
            \draw (2.6,1.7) to[out=-85,in=180] (4,0);
            \draw [-CF] (2.5,0) -- node[below]{2. \ce{H3O+, H2O}} ++(2,0);
    
            \node at (8,0) {
                \schemestart
                    \chemfig{OH-[2](-[:150]*4(----))-[:30]-[:-30]-[:30]-[:-30]}
                    \arrow{->[\ce{H2CrO4}]}[,1.2]
                    \chemfig{O=[2](-[:150]*4(----))-[:30]-[:-30]-[:30]-[:-30]}
                \schemestop
            };
            \node at (13,0.5) {
                \schemestart
                    \chemfig{[:18]*5(--(-OH)---)}
                    \arrow{->[*{0}2. \ce{Mg${}^\circ$, Et2O}][*{0}1. \ce{PBr3}]}[-90]
                    \chemfig{[:18]*5(--(-MgBr)---)}
                \schemestop
            };
            \draw (11.3,-0.8) to[out=-170,in=90] (10.07,-1.8);
            \draw [-CF] (10.07,-1) -- node[left]{2. \ce{H3O+, H2O}} ++(0,-1);
    
            \node at (6,-3) {
                \schemestart
                    \chemfig{OH-[2](-[:-150]*5(-----))(-[2]*4(----))-[:-30]-[:30]-[:-30]-[:30]}
                    \arrow{->[*{0.-90}\ce{POCl3}][*{0.90}\ce{Py}]}[180]
                    \chemfig{(=[:-150]*5(-----))(-[2]*4(----))-[:-30]-[:30]-[:-30]-[:30]}
                    \arrow{->[*{0.-90}mCPBA]}[180]
                    \chemfig{[:-60]*3(([::180]*5(-----))-O-(-[2]*4(----))(-[:-30]-[:30]-[:-30]-[:30])-)}
                \schemestop
            };
        \end{tikzpicture}
    \end{figure}
    \begin{itemize}
        \item Zaitsev's rule eliminates one alkene and ring strain eliminates the other. Thus, the alkene that's formed is the most stable one.
    \end{itemize}
\end{itemize}
\newpage



\section{Exam 3 Cheat Sheet}
\begin{table}[h!]\marginnote{3/15:}
    \centering
    \small
    \renewcommand{\arraystretch}{1.2}
    \begin{tabular}{|l|l|}
        \hline
        \multicolumn{2}{|c|}{\textbf{COMMON ABSORPTIONS}}\\ \hline
        Aromatic \ce{C-C} & Two peaks usually in the range of $\SIrange{1500}{1600}{\per\centi\meter}$\\ \hline
        \ce{C=C} & $\sim\SI{1650}{\per\centi\meter}$\\ \hline
        \ce{C=O} & $\sim\SI{1710}{\per\centi\meter}$ (shifts to $\sim\SI{1735}{\per\centi\meter}$ for esters)\\ \hline
        \ce{C#C} & $\SIrange{2100}{2300}{\per\centi\meter}$\\ \hline
        \ce{C#N} & $\SIrange{2100}{2300}{\per\centi\meter}$\\ \hline
        \ce{C-H} (aldehyde) & Two peaks at $\SI{2170}{\per\centi\meter}$ and $\SI{2810}{\per\centi\meter}$\\ \hline
        $sp^3$ \ce{C-H} & Just to the right of $\SI{3000}{\per\centi\meter}$\\ \hline
        $sp^2$ \ce{C-H} & Just to the left of $\SI{3000}{\per\centi\meter}$\\ \hline
        $sp$ \ce{C-H} & $\sim\SI{3300}{\per\centi\meter}$\\ \hline
        \ce{N-H} & $\sim\SI{3300}{\per\centi\meter}$ (one peak for \ce{-NH-}, two peaks for \ce{-NH2})\\ \hline
        \ce{O-H} (alcohol) & $\sim\SI{3400}{\per\centi\meter}$ (a broad, smooth peak)\\ \hline
        \ce{O-H} (acid) & $\sim\SIrange{2500}{3500}{\per\centi\meter}$ (a very broad, ugly [not smooth] peak)\\ \hline
    \end{tabular}
    \caption*{Common IR spectroscopy absorptions.}
\end{table}
\begin{table}[h!]
    \centering
    \small
    \renewcommand{\arraystretch}{1.4}
    \begin{tabular}{|lc|lc|}
        \hline
        \rule{0pt}{2em}\textbf{Type of Proton} & \textbf{\shortstack{Chemical Shift\\($\bm{\delta}$, ppm)}} & \textbf{Type of Proton} & \textbf{\shortstack{Chemical Shift\\($\bm{\delta}$, ppm)}}\\
        $\ang{1}$ Alkyl, {\sf\ce{RC{\color{rex}H}3}} & \numrange{0.8}{1.2} & Alkyl bromide, {\sf\ce{RC{\color{rex}H}2}Br} & \numrange{3.4}{3.6}\\
        $\ang{2}$ Alkyl, {\sf\ce{RC{\color{rex}H}2R}} & \numrange{1.2}{1.5} & Alkyl chloride, {\sf\ce{RC{\color{rex}H}2}Cl} & \numrange{3.6}{3.8}\\
        $\ang{3}$ Alkyl, {\sf\ce{R3C{\color{rex}H}}} & \numrange{1.4}{1.8} & Vinylic, {\sf\ce{R2C=C{\color{rex}H}2}} & \numrange{4.6}{5.0}\\
        Allylic, {\sf\ce{R2C=CR-C{\color{rex}H}3}} & \numrange{1.6}{1.9} & Vinylic, {\sf\ce{R2C=CR{\color{rex}H}}} & \numrange{5.2}{5.7}\\
        Ketone, {\sf\ce{RCOC{\color{rex}H}3}} & \numrange{2.1}{2.6} & Aromatic, {\sf\ce{Ar{\color{rex}H}}} & \numrange{6.0}{8.5}\\
        Benzylic, {\sf\ce{ArC{\color{rex}H}3}} & \numrange{2.2}{2.5} & Aldehyde, {\sf\ce{RCO{\color{rex}H}}} & \numrange{9.5}{10.5}\\
        Acetylenic, {\sf\ce{RC#C{\color{rex}H}}} & \numrange{2.5}{3.1} & Alcohol hydroxyl, {\sf\ce{RO{\color{rex}H}}} & \numrange{0.5}{6.0}\textsuperscript{*}\\
        Alkyl iodide, {\sf\ce{RC{\color{rex}H}2I}} & \numrange{3.1}{3.3} & Amino, {\sf\ce{R-N{\color{rex}H}2}} & \numrange{1.0}{5.0}\textsuperscript{*}\\
        Ether, {\sf\ce{ROC{\color{rex}H}2R}} & \numrange{3.3}{3.9} & Phenolic, {\sf\ce{ArO{\color{rex}H}}} & \numrange{4.5}{7.7}\textsuperscript{*}\\
        Alcohol, {\sf\ce{HOC{\color{rex}H}2R}} & \numrange{3.3}{4.0} & Carboxylic, {\sf\ce{RCOO{\color{rex}H}}} & \numrange{10}{13}\textsuperscript{*}\\
        \hline
        \multicolumn{4}{l}{\footnotesize\textsuperscript{*}The chemical shifts of these protons vary in different solvents and with temperature and concentration.}
    \end{tabular}
    \caption*{Approximate proton chemical shifts.}
\end{table}
\begin{table}[h!]
    \centering
    \small
    \renewcommand{\arraystretch}{1.4}
    \begin{tabular}{|lc|}
        \hline
        \rule{0pt}{2em}\textbf{Type of Carbon} & \textbf{\shortstack{Chemical Shift\\($\bm{\delta}$, ppm)}}\\
        $\ang{1}$ Alkyl, {\sf\ce{R{\color{rex}C}H3}} & \numrange{0}{40}\\
        $\ang{2}$ Alkyl, {\sf\ce{R{\color{rex}C}H2R}} & \numrange{10}{50}\\
        $\ang{3}$ Alkyl, {\sf\ce{R{\color{rex}C}HR2}} & \numrange{15}{50}\\
        Alkyl halide or amine, {\sf\ce{R3{\color{rex}C}X}} ($\ce{X}=\ce{Cl},\ce{Br},\ce{NR$'$2}$) & \numrange{10}{65}\\
        Alcohol or ether, {\sf\ce{R3{\color{rex}C}OR$'$}} & \numrange{50}{90}\\
        Alkyne, {\sf\ce{R{\color{rex}C}#R$'$}} & \numrange{60}{90}\\
        Alkene, {\sf\ce{R2{\color{rex}C}=R$'$}} & \numrange{100}{170}\\
        Aryl, {\renewcommand*\printatom[1]{\ensuremath{\mathsf{#1}}}\chemfig[atom sep=1.4em]{[:30]**6(--{\color{rex}C}(-R)----)}} & \numrange{100}{170}\\
        Nitrile, {\sf\ce{R{\color{rex}C}#N}} & \numrange{120}{130}\\
        Amide, {\sf\ce{R{\color{rex}C}ONR$'$2}} & \numrange{150}{180}\\
        Carboxylic acid or ester, {\sf\ce{R{\color{rex}C}OOR$'$}} & \numrange{160}{185}\\
        Aldehyde or ketone, {\sf\ce{R{\color{rex}C}OR$'$}} & \numrange{182}{215}\\
        \hline
    \end{tabular}
    \caption*{Approximate carbon-13 chemical shifts.}
\end{table}

\textbf{Reminders}:
\begin{itemize}
    \item Alkene reactions to know: hydrogenation (\ce{H2 + Pd}/\ce{C}), dihydroxylation (\ce{1. OsO4}, \ce{2. NaHSO3}), ozonolysis (\ce{O3 + Me2S}), hydrobromination (\ce{HBr}), and bromination (\ce{Br2}).
    \begin{itemize}
        \item Extra possibles: Acid-catalyzed hydration (\ce{H2SO4 + H2O}), oxymercuration/demercuration\\(\ce{1. Hg(OAc)2, H2O}, \ce{2. NaBH4}), hydroboration/oxidation (\ce{1. BH3}, \ce{2. H2O2, NaOH}), hydrogenation (\ce{H2 + \text{Lindlar's catalyst}} for alkynes to \emph{cis}-alkenes, \ce{2Na + 2NH3} for alkynes to \emph{trans}-alkenes), alkyne synthesis (terminal alkyne + \ce{1. NaNH2}, \ce{2. RBr}).
        \item Make ketones/aldehydes with ozonolysis, acid-catalyzed hydration of alkynes, and hydroboration of alkynes (with \ce{(sia)2BH} for terminal alkynes to aldehydes).
        \item Alkene to diene: \ce{1. Br2}, \ce{2. NaOH}.
        \begin{itemize}
            \item Alkene to alkyne: \ce{1. Br2}, \ce{2. 3NaNH2}.
        \end{itemize}
    \end{itemize}
    \item Frost method: Point down, MOs at the carbons.
    \begin{itemize}
        \item 5-membered rings: 3 bonding / 2 antibonding. 7-membered: 3 bonding / 4 antibonding.
    \end{itemize}
    \item Aromaticity checklist: Flat, cyclic, conjugated, uninterrupted flow of $p$-orbitals, $(4n+2)$-rule.
    \item $(+/-)$ for Diels-Alder reactions!
    \item F-C reactions happen ONLY IF there is not an EWG on the ring.
    \item Add stronger EWGs later.
    \item Nucleophile strengths.
    \begin{equation*}
        \ce{{}^-NRH} > \ce{RO- / HO-}
        > \ce{Br-}
        > \ce{NR3}
        > \ce{Cl-}
        > \ce{F-}
        > \ce{H2O / ROH}
        > \text{alkene}
        > \text{benzene}
    \end{equation*}
\end{itemize}

\textbf{Reactions}:
\begin{itemize}
    \item \ce{=- ->[Br2][h\nu] =--Br}
    \begin{itemize}
        \item Chlorination problems: Polychlorination, selectivity.
    \end{itemize}
    \item \ce{-= ->[HBr, h\nu][air] ---Br}
    \item \ce{C6H6 ->[D3O+] C6D6}
    \item \ce{PhH ->[Br2][FeBr3] PhBr}
    \begin{itemize}
        \item \ce{AlCl3}, \ce{CuI2}.
    \end{itemize}
    \item \ce{PhH ->[HNO3][H2SO4] PhNO2}
    \item \ce{PhH ->[SO3][H2SO4] PhSO3H}
    \item \ce{PhH ->[RCOCl][AlCl3] PhCOR}
    \item \ce{PhH ->[RCl][AlCl3] PhR}
    \item \ce{\text{benzylic carbonyl} ->[Zn(Hg)][HCl] \text{reduced carbon}}
    \item \ce{PhR ->[KMnO4][H2O] PhCOOH}
    \begin{itemize}
        \item Needs benzylic hydrogen.
    \end{itemize}
    \item \ce{PhNO2 ->[\text{reagents}] PhNH2}
    \begin{itemize}
        \item \ce{H2 + Pd/C} or \ce{SnCl2 + H2O} (selective).
    \end{itemize}
    \item \ce{PhNH2 ->[NaNO2][HCl] PhN2+ + X-}
    \begin{itemize}
        \item Mechanism has many equilibrium steps (only first and last are not).
    \end{itemize}
    \item \ce{PhN2+ ->[Cu2O][H2O] PhOH}
    \begin{itemize}
        \item \ce{PhN2+ ->[CuCl] PhCl}
        \item \ce{PhN2+ ->[CuBr] PhBr}
        \item \ce{PhN2+ ->[CuI] PhI}
        \item \ce{PhN2+ ->[CuCN] PhCN}
    \end{itemize}
    \item \ce{PhN2+ ->[D3PO2] PhD}
    \item \ce{PhBr ->[NaNH2][NH3] PhNH2}
    \item \ce{PhCl ->[NaNu][NuH] PhNu}
    \item \ce{PhH ->[Pd][$>1000\,\text{psi}$] CyH}
    \item \ce{\text{benzene} ->[2Li][NH3 / EtOH] \text{cyclohexa-1,4-diene} + 2LiOEt}
    \begin{figure}[h!]
        \centering
        \footnotesize
        \begin{tikzpicture}
            \draw [stealth-stealth] (-9.5,0) node[above right,yshift=1mm]{deactivators} -- (6.5,0) node[above left,yshift=1mm]{activators};
    
            \draw
                (-9,0.1) -- ++(0,-0.2) node[below]{\ce{{}^+NR3}}
                (-8,0.1) -- ++(0,-0.2) node[below]{\ce{NO2}}
                (-7,0.1) -- ++(0,-0.2) node[below]{\ce{CN}}
                (-6,0.1) -- ++(0,-0.2) node[below]{\ce{SO3H}}
                (-5,0.1) -- ++(0,-0.2) node[below]{\chemfig[atom sep=1.4em]{O=(-[:60])(-[:-60]R)}}
                (-4,0.1) -- ++(0,-0.2) node[below]{\ce{CO2H}}
                (-3,0.1) -- ++(0,-0.2) node[below]{\ce{CHO}}
                (-2.25,0.1) -- ++(0,-0.2) node[below]{\ce{I}}
                (-1.75,0.1) -- ++(0,-0.2) node[below]{\ce{Br}}
                (-1.25,0.1) -- ++(0,-0.2) node[below]{\ce{Cl}}
                (-0.75,0.1) -- ++(0,-0.2) node[below]{\ce{F}}
                (0,0.2) -- ++(0,-0.4) node[below]{\chemfig[atom sep=1.4em]{*6(-=-=-=)}}
                (1,0.1) -- ++(0,-0.2) node[below]{aryl}
                (2,0.1) -- ++(0,-0.2) node[below]{alkyl}
                (3,0.1) -- ++(0,-0.2) node[below]{\chemfig[atom sep=1.4em]{HN-[:-60](=[::-60]O)-[::60]R}}
                (4,0.1) -- ++(0,-0.2) node[below]{\ce{OR}}
                (5,0.1) -- ++(0,-0.2) node[below]{\ce{OH}}
                (6,0.1) -- ++(0,-0.2) node[below]{\ce{NH2}}
            ;
    
            \draw [decorate,decoration={brace,mirror}] (-9.4,-1.5) -- node[below=1mm]{m-directing} (-2.6,-1.5);
            \draw [decorate,decoration={brace,mirror}] (-2.4,-1.5) -- node[below=1mm]{o/p-directing} (-0.6,-1.5);
            \draw [decorate,decoration={brace,mirror}] (0.6,-1.5) -- node[below=1mm]{o/p-directing} (6.4,-1.5);
        \end{tikzpicture}
        \caption*{Activators and deactivators.}
    \end{figure}
    \item \ce{ROH ->[HBr] RBr}
    \item \ce{ROH + SOCl2 ->[Py] RCl + SO2 + Cl- + PyH+}
    \begin{itemize}
        \item \ce{PBr3}, \ce{PI3}.
    \end{itemize}
    \item \ce{ROH + Nu ->[TsCl][NEt3] RNu + HCl}
    \item {
        \footnotesize
        \setchemfig{atom sep=1.4em}
        \schemestart
            \chemfig{-[:30]-[:-30](-[:-110])(-[:-70]OH)-[:30]}
            \arrow{->[\ce{H+}]}
            \chemfig{-[:30]=_[:-30](-[6])-[:30]}
        \schemestop
    }
    \item {
        \footnotesize
        \setchemfig{atom sep=1.4em}
        \schemestart
            \chemfig{-[:30]-[:-30](-[6]OH)-[:30]}
            \arrow{->[\ce{POCl3}][\ce{Py}]}[,1.2]
            \chemfig{-[:30]=_[:-30]-[:30]}
            \+
            \chemfig{HCl}
            \+
            \chemfig{PO_2Cl}
            \+
            \chemfig{\charge{45:1pt=$-$}{Cl}}
            \+
            \chemfig{Py\charge{45:1pt=$+$}{H}}
        \schemestop
    }
    \item \ce{ROH ->[NaH][MeI] ROMe + H2 + NaI} --- Williamson Ether Synthesis.
    \item \ce{ROR$'$ ->[HBr] RBr + R$'$OH}
    \begin{itemize}
        \item Must use \ce{HBr} or \ce{HI}, not \ce{HCl}.
    \end{itemize}
    \item {
        \footnotesize
        \setchemfig{atom sep=1.4em}
        \schemestart
            \chemfig{-[:30]-[:-30]=_[:30]-[:-30]}
            \arrow{->[mCPBA][-mCBA]}[,1.3]
            \chemfig[atom sep=1.6em]{-[:30,0.85]-[:-30,0.85]*3(-O-(-[,0.85])-)}
        \schemestop
    }
    \item {
        \footnotesize
        \setchemfig{atom sep=1.4em}
        \schemestart
            \chemfig[bond offset=1pt]{*6(--*3(<O>)----)}
            \arrow{->[\ce{H+}][\ce{H2O}]}
            \chemfig{*6(--(<OH)-(<:OH)---)}
        \schemestop
    }
    \begin{itemize}
        \item Can be trapped by \ce{HCl} or \ce{HBr}.
        \item Acidic conditions $\to$ CC+ stability is important; basic $\to$ sterics.
    \end{itemize}
    \item {
        \footnotesize
        \setchemfig{atom sep=1.4em}
        \schemestart
            \chemfig{*6(--(<:Cl)-(<OH)---)}
            \arrow{->[\ce{NaOH}][-\ce{H2O, NaCl}]}[,1.6]
            \chemfig[bond offset=1pt]{*6(--*3(<O>)----)}
        \schemestop
    }
    \item \ce{RCOR$'$ ->[\text{reagents}] RC(OH)HR$'$}
    \begin{itemize}
        \item \ce{NaBH4} or \ce{LiAlH4}.
    \end{itemize}
    \item {
        \footnotesize
        \setchemfig{atom sep=1.4em}
        \schemestart
            \chemfig{R-[:30](=[2]O)-[:-30]O-[:30]R'}
            \arrow{->[1. \ce{LiAlH4}][2. \ce{H3O+, H2O}]}[,1.8]
            \chemfig{R-[:30]-[2]OH}
            \+
            \chemfig{HO-R'}
        \schemestop
    }
    \item \ce{RBr ->[Mg${}^\circ$][Et2O] RMgBr}
    \item \ce{RBr ->[2Li${}^\circ$][Et2O] RLi + LiBr}
    \item Collins reagent:
    \begin{figure}[H]
        \centering
        \footnotesize
        \setchemfig{atom sep=1.4em}
        \begin{subfigure}[b]{\linewidth}
            \centering
            \schemestart
                \chemfig{R-[:30]-[:-30]OH}
                \arrow{->[\ce{CrO3, Py}][\ce{CH2Cl2}]}[,1.3]
                \chemfig{R-[:30](=[2]O)-[:-30]H}
            \schemestop
            \caption{Anhydrous oxidation of a primary alcohol.}
        \end{subfigure}\\[1em]
        \begin{subfigure}[b]{\linewidth}
            \centering
            \schemestart
                \chemfig{R-[:30](-[2]OH)-[:-30]R'}
                \arrow{->[\ce{CrO3, Py}][\ce{CH2Cl2}]}[,1.3]
                \chemfig{R-[:30](=[2]O)-[:-30]R'}
            \schemestop
            \caption{Anhydrous oxidation of a secondary alcohol.}
        \end{subfigure}\\[1em]
        \begin{subfigure}[b]{\linewidth}
            \centering
            \schemestart
                \chemfig{R-[:30]-[:-30]OH}
                \arrow{->[\ce{CrO3, Py}][\ce{H2O}]}[,1.3]
                \chemfig{R-[:30](=[2]O)-[:-30]OH}
            \schemestop
            \caption{Aqueous oxidation of a primary alcohol.}
        \end{subfigure}
    \end{figure}
    \begin{itemize}
        \item Jones reagent: \ce{CrO3 + H2SO4_{(aq)}}; PCC: \ce{Py, HCl, CrO3}; Swern oxidation: \ce{1. DMSO, (COCl)2}, \ce{2. NEt3}.
    \end{itemize}
\end{itemize}

% \begin{itemize}
%     \item An epoxide and a Grignard adds a primary alcohol.
%     \item Grignards always preferentially deprotonate hydrogens.
%     \item To convert an alkene to a diene, brominate and then eliminate twice.
% \end{itemize}
\newpage



\section{Chapter 12: Alcohols from Carbonyl Compounds}
\emph{From \textcite{bib:SolomonsEtAl}.}
\begin{itemize}
    \item Together, reduction of carbonyls and modification by Grignards and organolithium reagents fall under the category of \textbf{nucleophilic addition}.
    \item There exist lowest and highest oxidation states of an organic compound.
    \begin{figure}[H]
        \centering
        \footnotesize
        \schemestart
            \chemfig{RCH_3}
            \arrow{<=>[[O]][[H]]}
            \chemfig{R-[:30](-[2]OH)(-[:-30]H)(-[:-60]H)}
            \arrow{<=>[[O]][[H]]}
            \chemfig{R-[:30](=[2]O)(-[:-30]H)}
            \arrow{<=>[[O]][[H]]}
            \chemfig{R-[:30](=[2]O)(-[:-30]OH)}
        \schemestop
        \caption{Oxidation state spectrum.}
        \label{fig:spectrumOxidationStates}
    \end{figure}
    \begin{itemize}
        \item Note that we use [H] to indicate in a general way that a molecule has been reduced and vice versa for [O].
    \end{itemize}
    \item "Oxidation of an organic compound may be more broadly defined as a reaction that increases its content of any element more electronegative than carbon" \parencite[537]{bib:SolomonsEtAl}.
    \item \ce{LiAlH4} is also denoted by the acronym LAH.
    \item Since LAH reacts violently with proton donors to release hydrogen gas, \ce{NaBH4} is a much safer (and therefore preferable) reagent for reducing aldehydes and ketones.
    \begin{itemize}
        \item Importantly, it can be used along with protic solvents.
        \item LAH is typically used in \ce{Et2O}. After the reaction is complete, \ce{EtAc} is added cautiously to decompose remaining LAH and then water to decompose the alumina complex, rendering it inert.
    \end{itemize}
    \item Aldehydes and ketones can also be reduced via \ce{H2 + Pd/C} (hydrogen and a metal catalyst) and \ce{Na${}^\circ$ + ROH} (sodium metal in an alcohol solvent).
    \item Almost all types of alkyl halides can be reduced by \ce{LiAlH4} in ether followed by sulfuric acid in water.
    \begin{itemize}
        \item Note that the proton comes from \ce{LiAlH4}, so we may use \ce{LiAlD4} to replace the halide with deuterium.
    \end{itemize}
    \item Primary and secondary alcohols can be reduced to carbonyl compounds, but tertiary ones cannot.
    \begin{itemize}
        \item This is because we need a hydrogen on the $\alpha$-carbon to lose along with the hydrogen from the alcohol group.
    \end{itemize}
    \item Oxidation of alcohols.
    \begin{itemize}
        \item \marginnote{3/11:}"Primary alcohols can be oxidized to aldehydes, and aldehydes can be oxidized to carboxylic acids" \parencite[542]{bib:SolomonsEtAl}.
        \item "Secondary alcohols can be oxidized to ketones" \parencite[542]{bib:SolomonsEtAl}.
        \item "Tertiary alcohols cannot be oxidized to carbonyl compounds" \parencite[542]{bib:SolomonsEtAl}.
    \end{itemize}
    \item The common mechanistic theme of alcohol oxidation by elimination.
    \begin{itemize}
        \item We attach a leaving group to the hydroxyl oxygen and deprotonate.
        \item Attacking an $\alpha$-hydrogen subsequently causes elimination of that hydrogen and the leaving group; the hydrogen's electrons become the double bond.
    \end{itemize}
    \item The Swern oxidation is usually carried out at low temperatures.
    \item Chromic acid (\ce{H2CrO4}) oxidation is discussed, but in a mechanistically different manner to that presented in class.
    \item Chromic acid is orange-red, but Cr(III) (in the product mixture) is greenish blue. Thus, reagents like Jones reagent can serve as a color-based test for the presence of functional groups including primary and secondary alcohols and aldehydes.
    \begin{itemize}
        \item This color change was the basis of the original brethalyzer test.
    \end{itemize}
    \item Since PCC is soluble in solvents other than water (e.g., \ce{CH2Cl2}), it can be used for the necessarily anhydrous monooxidations of alcohols.
    \item \marginnote{3/10:}\textbf{Organometallic compound}: A compound that contains a carbon-metal bond.
    \item \ce{C-M} bonds are largely ionic when $\ce{M}=\ce{Na, K}$, are largely covalent when $\ce{M}=\ce{Pb, Sn, Hg, Ti}$, and are in between when $\ce{M}=\ce{Mg, Li}$.
    \item Reactivity of organometallics increases with increasing ionic character.
    \begin{itemize}
        \item Alkylsodium an alkylpotassium compounds are among the most powerful of bases, but also react explosively with water and burst into flame when exposed to air.
        \item The more stable ones may only be volatile in air, but are still highly poisonous (e.g., \ce{Et4Pb}, the infamous antiknock compound formerly used in leaded gasoline).
    \end{itemize}
    \item Most Grignards exist in equilibrium between an alkylmagnesium halide and a dialkyl magnesium.
    \begin{equation*}
        \ce{2RMgX <=> R2Mg + MgX2}
    \end{equation*}
    \item Grignards in their alkylmagnesium halide state also form a complex with their aprotic solvent, attracting electron pairs in two partial bonds to their positive magnesium.
    \item A Grignard reagent behaves like a strong base and reacts to form a weak conjugate acid (such as its protonated or otherwise alkylated form).
    \item Grignard reagents can even deprotonate terminal alkynes.
    \begin{itemize}
        \item This serves as a method of production of alkynylmagnesium halides and alkynyllithiums, though.
    \end{itemize}
\end{itemize}




\end{document}